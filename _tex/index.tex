% Options for packages loaded elsewhere
% Options for packages loaded elsewhere
\PassOptionsToPackage{unicode}{hyperref}
\PassOptionsToPackage{hyphens}{url}
\PassOptionsToPackage{dvipsnames,svgnames,x11names}{xcolor}
%
\documentclass[
  10pt,
  letterpaper,
  DIV=11,
  numbers=noendperiod]{scrartcl}
\usepackage{xcolor}
\usepackage{amsmath,amssymb}
\setcounter{secnumdepth}{5}
\usepackage{iftex}
\ifPDFTeX
  \usepackage[T1]{fontenc}
  \usepackage[utf8]{inputenc}
  \usepackage{textcomp} % provide euro and other symbols
\else % if luatex or xetex
  \usepackage{unicode-math} % this also loads fontspec
  \defaultfontfeatures{Scale=MatchLowercase}
  \defaultfontfeatures[\rmfamily]{Ligatures=TeX,Scale=1}
\fi
\usepackage{lmodern}
\ifPDFTeX\else
  % xetex/luatex font selection
\fi
% Use upquote if available, for straight quotes in verbatim environments
\IfFileExists{upquote.sty}{\usepackage{upquote}}{}
\IfFileExists{microtype.sty}{% use microtype if available
  \usepackage[]{microtype}
  \UseMicrotypeSet[protrusion]{basicmath} % disable protrusion for tt fonts
}{}
\makeatletter
\@ifundefined{KOMAClassName}{% if non-KOMA class
  \IfFileExists{parskip.sty}{%
    \usepackage{parskip}
  }{% else
    \setlength{\parindent}{0pt}
    \setlength{\parskip}{6pt plus 2pt minus 1pt}}
}{% if KOMA class
  \KOMAoptions{parskip=half}}
\makeatother
% Make \paragraph and \subparagraph free-standing
\makeatletter
\ifx\paragraph\undefined\else
  \let\oldparagraph\paragraph
  \renewcommand{\paragraph}{
    \@ifstar
      \xxxParagraphStar
      \xxxParagraphNoStar
  }
  \newcommand{\xxxParagraphStar}[1]{\oldparagraph*{#1}\mbox{}}
  \newcommand{\xxxParagraphNoStar}[1]{\oldparagraph{#1}\mbox{}}
\fi
\ifx\subparagraph\undefined\else
  \let\oldsubparagraph\subparagraph
  \renewcommand{\subparagraph}{
    \@ifstar
      \xxxSubParagraphStar
      \xxxSubParagraphNoStar
  }
  \newcommand{\xxxSubParagraphStar}[1]{\oldsubparagraph*{#1}\mbox{}}
  \newcommand{\xxxSubParagraphNoStar}[1]{\oldsubparagraph{#1}\mbox{}}
\fi
\makeatother


\usepackage{longtable,booktabs,array}
\usepackage{calc} % for calculating minipage widths
% Correct order of tables after \paragraph or \subparagraph
\usepackage{etoolbox}
\makeatletter
\patchcmd\longtable{\par}{\if@noskipsec\mbox{}\fi\par}{}{}
\makeatother
% Allow footnotes in longtable head/foot
\IfFileExists{footnotehyper.sty}{\usepackage{footnotehyper}}{\usepackage{footnote}}
\makesavenoteenv{longtable}
\usepackage{graphicx}
\makeatletter
\newsavebox\pandoc@box
\newcommand*\pandocbounded[1]{% scales image to fit in text height/width
  \sbox\pandoc@box{#1}%
  \Gscale@div\@tempa{\textheight}{\dimexpr\ht\pandoc@box+\dp\pandoc@box\relax}%
  \Gscale@div\@tempb{\linewidth}{\wd\pandoc@box}%
  \ifdim\@tempb\p@<\@tempa\p@\let\@tempa\@tempb\fi% select the smaller of both
  \ifdim\@tempa\p@<\p@\scalebox{\@tempa}{\usebox\pandoc@box}%
  \else\usebox{\pandoc@box}%
  \fi%
}
% Set default figure placement to htbp
\def\fps@figure{htbp}
\makeatother


% definitions for citeproc citations
\NewDocumentCommand\citeproctext{}{}
\NewDocumentCommand\citeproc{mm}{%
  \begingroup\def\citeproctext{#2}\cite{#1}\endgroup}
\makeatletter
 % allow citations to break across lines
 \let\@cite@ofmt\@firstofone
 % avoid brackets around text for \cite:
 \def\@biblabel#1{}
 \def\@cite#1#2{{#1\if@tempswa , #2\fi}}
\makeatother
\newlength{\cslhangindent}
\setlength{\cslhangindent}{1.5em}
\newlength{\csllabelwidth}
\setlength{\csllabelwidth}{3em}
\newenvironment{CSLReferences}[2] % #1 hanging-indent, #2 entry-spacing
 {\begin{list}{}{%
  \setlength{\itemindent}{0pt}
  \setlength{\leftmargin}{0pt}
  \setlength{\parsep}{0pt}
  % turn on hanging indent if param 1 is 1
  \ifodd #1
   \setlength{\leftmargin}{\cslhangindent}
   \setlength{\itemindent}{-1\cslhangindent}
  \fi
  % set entry spacing
  \setlength{\itemsep}{#2\baselineskip}}}
 {\end{list}}
\usepackage{calc}
\newcommand{\CSLBlock}[1]{\hfill\break\parbox[t]{\linewidth}{\strut\ignorespaces#1\strut}}
\newcommand{\CSLLeftMargin}[1]{\parbox[t]{\csllabelwidth}{\strut#1\strut}}
\newcommand{\CSLRightInline}[1]{\parbox[t]{\linewidth - \csllabelwidth}{\strut#1\strut}}
\newcommand{\CSLIndent}[1]{\hspace{\cslhangindent}#1}



\setlength{\emergencystretch}{3em} % prevent overfull lines

\providecommand{\tightlist}{%
  \setlength{\itemsep}{0pt}\setlength{\parskip}{0pt}}



 


\usepackage{booktabs}
\usepackage{caption}
\usepackage{longtable}
\usepackage{colortbl}
\usepackage{array}
\usepackage{anyfontsize}
\usepackage{multirow}
\usepackage{pdfpages}
\KOMAoption{captions}{tableheading}
\makeatletter
\@ifpackageloaded{caption}{}{\usepackage{caption}}
\AtBeginDocument{%
\ifdefined\contentsname
  \renewcommand*\contentsname{Table of contents}
\else
  \newcommand\contentsname{Table of contents}
\fi
\ifdefined\listfigurename
  \renewcommand*\listfigurename{List of Figures}
\else
  \newcommand\listfigurename{List of Figures}
\fi
\ifdefined\listtablename
  \renewcommand*\listtablename{List of Tables}
\else
  \newcommand\listtablename{List of Tables}
\fi
\ifdefined\figurename
  \renewcommand*\figurename{Figure}
\else
  \newcommand\figurename{Figure}
\fi
\ifdefined\tablename
  \renewcommand*\tablename{Table}
\else
  \newcommand\tablename{Table}
\fi
}
\@ifpackageloaded{float}{}{\usepackage{float}}
\floatstyle{ruled}
\@ifundefined{c@chapter}{\newfloat{codelisting}{h}{lop}}{\newfloat{codelisting}{h}{lop}[chapter]}
\floatname{codelisting}{Listing}
\newcommand*\listoflistings{\listof{codelisting}{List of Listings}}
\makeatother
\makeatletter
\usepackage{pdflscape}
\makeatother
\makeatletter
\makeatother
\makeatletter
\@ifpackageloaded{caption}{}{\usepackage{caption}}
\@ifpackageloaded{subcaption}{}{\usepackage{subcaption}}
\makeatother
\usepackage{bookmark}
\IfFileExists{xurl.sty}{\usepackage{xurl}}{} % add URL line breaks if available
\urlstyle{same}
\hypersetup{
  pdftitle={Supplementary materials},
  colorlinks=true,
  linkcolor={blue},
  filecolor={Maroon},
  citecolor={Blue},
  urlcolor={Blue},
  pdfcreator={LaTeX via pandoc}}


\title{Supplementary materials}
\author{}
\date{}
\begin{document}
\maketitle

\renewcommand*\contentsname{Table of contents}
{
\hypersetup{linkcolor=}
\setcounter{tocdepth}{3}
\tableofcontents
}

\section{Trial registry}\label{trial-registry}

\includepdf[pages=-, fitpaper=true]{resources/GermanClinicalTrialsRegister.pdf}

\newpage{}

\section{The @myTabu framework}\label{the-mytabu-framework}

Please note, that the following descriptions are partly adapted from
Fromberger et al.~(2021).\textsuperscript{1}

\subsection{Structure of WBI}\label{structure-of-wbi}

The WBI @myTabu consists of three isolated sections, one only for
participants, one only for the online coaches, and one section for
community supervisors (see Figure~\ref{fig-sitemap}). The access to
these sections was secured by username and password full-filling the
criteria of the DSGVO. On the landing page of the coach section, a
dashboard informs the coach on tasks he has to perform (e.g.~unanswered
messages by a client, open guided tasks). Each coach had only access to
the participants he was responsible for. The landing page links to the
in-app messenger and a page for working on guided tasks together with
the participant. The supervisor section, only accessible by
participating community supervisors, provides a landing page which
summarized important key aspects (treatment progress, critical events)
of all participants, which have been supervised by this supervisor. Due
to security reasons, the supervisor was able to see the raw values of
the primary endpoint. If a item of the primary endpoint has been graded
as a potential hint to recidivism by lawyers within the research team
during development of the app, additional judicial information are
provided specifically for the federal state the supervision officer is
working at.

\begin{figure}

\centering{

\includegraphics[width=1\linewidth,height=\textheight,keepaspectratio]{_diagrams/mtDiagrams-Sitemap.png}

}

\caption{\label{fig-sitemap}Sitemap of the WBI @myTabu. WBI = Web-Based
Intervention.}

\end{figure}%

The participant section, only accessible by participants, provides on
the landing page key aspects of their progress, e.g.~current saldo,
gudied exercises the have to work on, a help page, a first aid kit as
well as a section for the WBI. The participants are remembered on open
guided tasks or being late with the intervention by in-app alerts. The
WBI starts with two introductory sessions, in order to explain the
participants how the app works. After finishing the instruction section,
the participants is linked to the first lection of the WBI. The
structure and organization of the content was identical across both
trial arms. The WBI was structured into six modules, each module into
four sessions and each session into six to eight lections. A lection was
divided into different content blocks: guided exercises,
psychoeducation, examples (virtual participants), and automated
exercises. Virtual participants accompanied the participant, narrate
about their experiences, and make suggestions from the perspective of
someone who is also affected (scored by professional actor)At the
beginning of each session, key elements from the previous session or
module were recapped. At the end of each session, the most important
points of that session or module were summarized. Content created during
exercises was partly dynamic integrated in consecutive sessions. For
example, at the start of the intervention participants created a list of
positive activities, which was later used to suggest ways in which they
could reward themselves for completing a session or module. In both
trial arms, the program included digital psycho-educational blocks
incorporating multimedia elements such as videos and graphics, as well
as guided and unguided exercises. A total of eighteen different online
exercises (e.g., fill‑in‑the‑gap tasks, multiple‑choice questions,
four-field-tasks) were integrated into the WBI. These exercises were
either offered as self‑guided tasks with automated feedback or as
coach‑guided tasks in which feedback was provided by online coaches. In
the intervention arm, the most relevant aspects of each session were
delivered as guided tasks. In the placebo arm, the allocation of tasks
as guided or unguided was determined based on a predefined matching
procedure to replicate the distribution of guided and unguided tasks in
the intervention arm. The feedback of the online coaches was manualized:
Within a special section of the WBI, the coach was able to choose
pre-build templates for each guided task (see
Figure~\ref{fig-act-guided-task}). Furthermore, an in-app messaging
system allowed the user to communicate with the online coach as well as
the supervision officer in a secured environment. It was ensured that
messages and feedback for guided tasks has been provided within 36 hours
by the coach. In order to enhance the extrinsic motivation of
participants, an automatic reward system is integrated based on a
gamification approach. That is, participants receive coins and virtual
awards for each finished content block. The coins can be changed into
Euros, if the participant has collected at least 15 Euros, he can
request the payout online. The participant can receive a maximum
compensation of 120 Euro. Finally, an online first aid kit was provided
for each user and filled with individual skills prepared by the user
itself during guided tasks in the intervention arm. In the placebo arm,
the first add kit was not individualized but pre-filled with basic
hints. Throughout all modules, the speech was friendly, empathetic,
simple, and easy to understand (e.g., no technical terms, examples of
everyday life) in order to account for the different cognitive
functioning level of participants.

\begin{figure}

\centering{

\includegraphics[width=1\linewidth,height=\textheight,keepaspectratio]{_diagrams/mtDiagrams-act_Guided_Task.png}

}

\caption{\label{fig-act-guided-task}Process flow of guided exercises. 29
guided exercises have been developed for each trial arm. The figure
shows the process of reviewing a guided task by study therapists.}

\end{figure}%

\subsection{Theoretical foundation}\label{theoretical-foundation}

Already established therapeutic concepts from in-face community-based
treatment programs for individuals who committed sexual offenses against
children\textsuperscript{2,3} as well as cognitive-behavioral treatment
concepts build the theoretical ground of the WBI; see\textsuperscript{1}
for more details. The in-person programs have been adapted for online
use. Both programs have been designed for CSA as well as CSEM offenders.
Parts relevant to the meaningful risk factors of both in-face programs
have been digitalized and parts on risk factors not supported by
empirical evidence have been skipped in order to shorten the WBI. The
content of the intervention arm focused exclusively on psychological
meaningful risk factors\textsuperscript{4,5}, see
Figure~\ref{fig-wbi-overview} for an overview of the theoretical concept
of the WBI. To achieve a reduction in re-offenses, it seems to be most
promising to target dynamic factors, which are changeable and strong
associated with recidivism\textsuperscript{5}. Mann et al.~(2010)
introduced the term meaningful psychological risk factors for sexual
offending by identifying those risk factors in the literature, which are
dynamic and therefore changeable by treatment and are empirically proved
in enhancing the risk for recidivism. Recently, Seto et al.~(2023)
replicated the findings and proofed the meaningful risk factors with a
broader and more recent literature basis. The emprirical basis for risk
factors for CSAM-exclusive offences is not as broad as for sexual
offending in the sense of in-person offences.\textsuperscript{6}
elaborated in their review risk factors for CSEM-exclusive offenders.
Atypical sexual interests, sexual pre-occupation, sexual arousal, access
to the internet and children, and distorted cognitions are assumed to
play a key role in recidivism. Thus, there is a overlap between risk
factors for CSA offenses and risk factors for CSAM
offenses\textsuperscript{7}. But it is assumed, that e.g.~the content of
offense-supportive attitudes differs from the content of
offense-supportive attitudes for mixed offenders or CSA-exclusive
offenders. Access to children or the internet is a so-called
facilitating factor in the sense of the Motivation-Facilitation Model of
sexual offending and therefore only indirect changeable by
treatment\textsuperscript{8}. In consequence, addressed risk factore
were the same for participants with CSA offences and participants with
CSEM offences. Virtual participants (four male and one female offender
with different offense histories) accompanied the participant, narrate
about their experiences, and make suggestions from the perspective of
someone who is also affected (scored by professional actor). Guided
exercises are used at the core of cognitive-behavioral treatment within
the WBI. Guided exercises have been developed to train important new
concepts online. They ensure, that the participants have to think about
the previous contents to answer them correctly, since the online coach
reviews the answers. Additionally, guided exercises are the most
probable content block in which the online coach gets in touch with the
participant (see Figure~\ref{fig-act-guided-task}). All modules within
the intervention arm comprised techniques of motivational
interviewing\textsuperscript{9} to enhance or maintain the motivation of
the participants. The WBI is designed for female as well as male
participants, but due to a lack of knowledge with regard to specific
risk factors for female sex offenders at development of the WBI, the
content remains identical for women and men. In order to make the WBI
suitable for female participants, the used language was adapted and one
female virtual participant has been designed and integrated. The placebo
arm was developed to ensure the same mode, dose, and amount of support
and attention by supporting online coaches as in the intervention arm -
but the content of the modules was unrelated to meaningful dynamic risk
factors. The content of the placebo condition comprised information on
healthy living, e.g., healthy nutrition, physical activities, or healthy
sleep. The amount of guided exercises was the same as in the
intervention arm (n = 29 guided exercises), but the exercises have been
specifically designed to not induce any sustainable effect. In the
following paragraphs, the theoretical backgorund of each module is
described in detail.

\begin{figure}

\centering{

\includegraphics[width=1\linewidth,height=\textheight,keepaspectratio]{_diagrams/mtDiagrams-act_Wbi_Overview.png}

}

\caption{\label{fig-wbi-overview}Overview of the WBI. The figure shows
the correspondence between meaningful risk-factors, secondary outcomes
of the clinical trial, and the content of the trial arm. Abbreviations:
BIS-15 = Barratt Impulsiveness Scale-15, BMS = Bumby Molest Scale, CUSI
= Coping Using Sex Inventory, CVTRQ = Corrections Victoria Treatment
Readiness Questionnaire, DERS = Difficulties in Emotion Regulation
Scale, EKK-R = Questionnaire on Emotional Congruence with
Children-Revised, ESIQ = Explicit Sexual Interest Questionnaire, F-Soz-U
= Seven-item short version of the Social Support Questionnaire, HBI-19 =
Hypersexual Behavior Inventory-19, NARQ = Negative Affect Repair
Questionnaire, OQMPR = Questionnaire for the Measurement of
Psychological Reactance, RCQ = Readiness to Change Questionnaire -
German version, SOI-R = Sexual Outlet Inventory revised, subscale desire
for sexual activity with children, SPSI-R = Social Problem-Solving
Inventory Revised, SSIC = Specific self-efficacy for modifying Sexual
Interest in Children, UCLA = UCLA Loneliness Scale - German short
version.}

\end{figure}%

\subsubsection{Module 1: Motivation}\label{module-1-motivation}

Ward et\,al.~identify motivation to change and subsequent treatment
engagement as intermediate targets that must be addressed before
criminogenic needs can be modified.\textsuperscript{10} Empirical
evidence indicates that higher treatment motivation is associated with
greater therapeutic success among individuals who committed CSA offences
and with programme dropout in web‑based interventions for alcohol use
disorder.\textsuperscript{11,12} Reviews of web‑based interventions
suggest that most treatment discontinuation occurs before the active
treatment phase begins and that the timing of dropout is comparable
between in‑person and therapist‑guided web‑based
treatments.\textsuperscript{13} As motivation to change is likely to
influence treatment engagement, outcomes, and dropout among ICCO, these
findings highlight the importance of addressing motivation at the outset
of the study.\textsuperscript{11,12} The motivation module is therefore
grounded in motivational interviewing principles, which conceptualise
behavioural change as a process marked by
ambivalence.\textsuperscript{9} Therapists support participants in
exploring reasons for and against change, with a focus on personal
values and goal setting to foster intrinsic motivation. The approach is
guided by four core principles---expressing empathy, developing
discrepancy, avoiding argumentation, and supporting
self‑efficacy---implemented through open‑ended questioning, reflective
listening, affirmations, change‑talk strategies, resistance management,
confidence‑building techniques, and summaries.

\subsubsection{Module 2: Supervision and social
relationships}\label{module-2-supervision-and-social-relationships}

Theoretical and empirical models of criminal behaviour underscore the
relevance of social influences for recidivism.\textsuperscript{14}
Negative social environments and resistance to rules and supervision
have been identified as meaningful risk factors for sexual
reoffending.\textsuperscript{5,15} Accordingly, interventions for ICCO
under CS should promote the development of protective social networks.
Such networks extend beyond family and peers to include therapists,
supervision officers, and social workers. The supervision and social
relationships module of @myTabu therefore aims to strengthen motivation
to establish positive social connections and to reduce resistance to
supervision through psychoeducational and cognitive‑behavioural
techniques, including the illustration of associations between negative
social networks and reoffending and the provision of strategies to
disengage from criminogenic influences. A further established risk
factor for sexual reoffending is the absence of emotionally intimate
adult relationships.\textsuperscript{5} Individuals who have committed
sexual offences and maintain stable adult partnerships show lower
recidivism rates than those who do not, and deficits in emotionally
intimate adult relationships are linearly associated with sexual
reoffending.\textsuperscript{15,16} Consequently, this module provides
information on these associations and offers guidance to support the
development of emotionally intimate adult relationships. To date, the
effectiveness of the specific supervision‑related content described here
has not been empirically evaluated, either in in‑person or in web‑based
interventions.

\subsubsection{Module 3: Emotion
management}\label{module-3-emotion-management}

Lifestyle impulsiveness as a pattern characterised by low self‑control,
instability in employment and housing, absence of meaningful daily
routines, irresponsible decision‑making, and limited or unrealistic
long‑term goals is associated with recdivism. Lifestyle impulsiveness is
a well‑established risk factor for sexual reoffending with a broad
empirical base.\textsuperscript{5} For dysfunctional coping, Knight and
Thornton demonstrated that external coping---defined as a tendency to
respond to stressful and negatively valenced emotional events in a
reckless and impulsive manner---predicts sexual
recidivism.\textsuperscript{17} To address these risk factors, @myTabu
integrates skills training derived from Dialectical Behaviour Therapy
(DBT\textsuperscript{18}) and Acceptance and Commitment Therapy
(ACT\textsuperscript{19}) with the aim of reducing impulsivity and
strengthening coping capacities in response to stressful or negatively
experienced life events. DBT skills training targets emotion regulation
and distress tolerance by teaching alternative behavioural strategies to
replace external or sexualised coping. DBT has previously been shown to
reduce impulsive behaviour in forensic populations, and web‑based
interventions based on DBT principles have demonstrated effectiveness,
particularly in reducing stress, across several psychiatric
conditions.\textsuperscript{20--22} ACT focuses on fostering a
meaningful life by promoting acceptance of negative internal experiences
and providing strategies for adaptive responses to emotional
distress.\textsuperscript{19} ACT has been successfully applied in
offender treatment settings and has shown promising results when
delivered via web‑based interventions for various psychiatric
disorders.\textsuperscript{21--23}

\subsubsection{Module 4: Problem
solving}\label{module-4-problem-solving}

Meta‑analyses have demonstrated a significant association between poor
social problem‑solving abilities and sexual
recidivism.\textsuperscript{4,5} Social problem solving is defined as a
self‑directed cognitive‑behavioural process through which individuals
attempt to identify and implement effective solutions to problems
encountered in everyday life.\textsuperscript{24} Within the widely
accepted cognitive model proposed by D'Zurilla and Goldfried, problem
solving comprises two dimensions: problem orientation and
problem‑solving style. Problem orientation may be positive (PPO) or
negative (NPO). PPO is associated with adaptive coping and reduced
emotional distress, whereas NPO is linked to avoidance and ineffective
coping strategies.\textsuperscript{25} According to this model, three
problem‑solving styles can be distinguished: rational problem solving
(RPS), impulsive coping style (ICS), and avoidance style (AS). RPS
involves a systematic and planned application of skills to achieve
adaptive solutions. In contrast, ICS is characterised by unsystematic
responses, often involving the implementation of the first available
option, while AS is marked by avoidance of problems and reliance on
external solutions. Empirical evidence indicates that ICCO exhibit
higher levels of NPO and lower levels of PPO than the general population
and show a greater tendency towards ICS and AS compared with
RPS.\textsuperscript{26} Moreover, poor problem‑solving skills in this
population have been shown to be amenable to change through
psychological interventions.\textsuperscript{27,28} The problem‑solving
module therefore aims to improve deficient problem‑solving skills by
drawing on the well‑established principles of Problem‑Solving Therapy
(PST\textsuperscript{26}). The module focuses on fostering a positive
problem orientation and strengthening rational problem‑solving skills.

\subsubsection{Module 5: Offense‑supportive
cognitions}\label{module-5-offensesupportive-cognitions}

Offense‑supportive attitudes among ICCO can be summarised as beliefs
that children are capable of sexually mature relationships, are not
harmed by sexual contact with adults, or actively provoke sexual
behaviour in adults.\textsuperscript{5,29} These attitudes must be
clearly distinguished from attempts to excuse or justify one's own
specific offences, which should not be the focus of cognitive
treatment.\textsuperscript{4} A meta‑analysis by Helmus
et\,al.~demonstrated that only offense‑supportive attitudes, as defined
above, constitute a psychologically meaningful risk factor in terms of
predicting sexual recidivism among ICCO.\textsuperscript{30} In
contrast, offence‑specific justifications have not been shown to be
predictive and are therefore not considered primary targets of
intervention. The offense‑supportive attitudes module applies cognitive
restructuring to modify offense‑supportive beliefs. Cognitive
restructuring is the most widely used and best‑evaluated technique for
changing maladaptive cognitions in this population. Modifying implicit
theories, attitudes, and beliefs is challenging, particularly among
ICCO, as such beliefs may not be experienced as harmful by the
individual and may instead serve to minimise perceived harm resulting
from their own offences.\textsuperscript{29} Nevertheless, several
studies have demonstrated that cognitive restructuring can effectively
reduce implicit offense‑supportive beliefs and
attitudes.\textsuperscript{31}

\subsubsection{Module 6: Sexuality}\label{module-6-sexuality}

Individuals who have committed sexual offences have been shown to be
more sexually active and more strongly interested in sexual topics than
non‑offending individuals.\textsuperscript{32} Sexual preoccupation is
characterised by intrusive sexual thoughts and behaviours that are
difficult or impossible to control (84) and has been shown to be
strongly associated with sexual reoffending.\textsuperscript{15}
Accordingly, a primary aim of the sexuality module is to provide basic
knowledge about sexuality and to support the reduction of excessive
sexual thoughts and behaviours. The module further aims to increase
awareness of high‑risk situations that may trigger sexual thoughts and
behaviours. Participants are provided with strategies to manage and,
where possible, reduce sexual preoccupation by limiting sexual thoughts
and activities. Emotional congruence with children represents another
central risk factor addressed in the sexuality module. ICCO under CS may
experience a perceived emotional closeness to children and, when
identifying themselves with children, may find it easier to initiate
contact with them.\textsuperscript{33} Emotional congruence with
children is therefore considered a risk factor strongly associated with
sexual reoffending.\textsuperscript{34} The sexuality module seeks to
reduce emotional congruence with children through psychoeducational
approaches, emphasising that emotionally equal relationships between
adults and children are not possible and providing strategies to
establish greater emotional distance from children. Sexual interest in
children, including sexual fantasies involving children, constitutes one
of the most important risk factors for sexual reoffending among
ICCO\textsuperscript{5} and has been described as a potential
motivational factor underlying sexual abuse of
children.\textsuperscript{35} Reducing sexual interest in children while
strengthening non‑deviant sexual interests may therefore contribute to
risk reduction. The extent to which sexual interests can change across
the lifespan remains debated, and empirical evidence is currently
insufficient to draw definitive conclusions.\textsuperscript{36}
Clinically, both changes in sexual interests and highly persistent
pedophilic interests have been observed, supporting the
conceptualisation of sexual interest in children as a dynamic risk
factor that may change over time.\textsuperscript{36} Within the
sexuality module, cognitive‑behavioural techniques are applied to
influence sexual interest in and, where possible, sexual fantasies about
children, while promoting sexual interest in adults. The techniques
employed in this module have demonstrated effectiveness in in‑person
treatments and, at least among younger individuals, have also been
applied in web‑based interventions.\textsuperscript{37--40}

\begin{figure}

\centering{

\includegraphics[width=1\linewidth,height=\textheight,keepaspectratio]{_diagrams/mtDiagrams-act_Critical_Event.png}

}

\caption{\label{fig-act-critical-event}Action diagram for critical
events. Predefined standard operating procedures have been used in order
to ensure the security of society during the clinical trial. The figure
shows exemplary the process for the occurence of a critical event (hints
on concrete preparations of a re-offence). Please note, no critical
events occured during the clinical trial.}

\end{figure}%

\newpage{}

\section{Outcomes and measurement
timepoints}\label{outcomes-and-measurement-timepoints}

\subsection{Description of trial endpoints}\label{endpoint-description}

Note that the following description of the primary and secondary outcome
moeasures is adapted from\textsuperscript{1}.

\subsubsection{SSQ-P}\label{ssq-p}

The Sample Specifications Questionnaire for the Participant (SSQ-P) is a
self developed online questionnaire asking the participants about their
demographics, lifestyle, current and former therapeutic treatments and
offending history.

\subsubsection{SSQ-SO}\label{ssq-so}

The Sample Specifications Questionnaire for the SO (SSQ-SO) is a
self-developed online questionnaire for the supervision officer
concerning the current living condition, offending and clinical history,
and index offense of the participant.

\subsubsection{SSQ-CF}\label{ssq-cf}

The Sample Specifications Questionnaire based on the Court File (SSQ-CF)
is a self-developed checklist for coding court files. Information from
the court files are coded by the study investigators to determine
whether the participant meets the study inclusion criteria. This
includes information e.g on the index offense, and former offenses. In
case that items of the SSQ-CF cannot be filled out adequately by
utilizing court files, the participants was interviewed directly.

\subsubsection{PHQ-D}\label{phq-d}

The Patient Health Questionnaire (PHQ-D\textsuperscript{41}) is a
screening tool for mental disorders. Classification of results generally
takes place on a syndrome level. Syndromes examined are: Somatoform
disorders, major depressive syndrome, other depressive syndrome, panic
syndrome, other anxiety syndromes, eating disorder syndrome, alcohol
abuse. Tests are scored using a stencil. Utilizing a sum score for each
syndrome, conclusions pertaining to the severity of each syndrome can be
drawn. The measure shows good to very good diagnostic validity,
especially regarding panic disorder and major
depression\textsuperscript{42}.

\subsubsection{SSPI-2}\label{sspi-2}

The Revised Scale for Pedophilic Interests (SSPI-2\textsuperscript{43})
is a structured rating scale of assessing pedophilic interests based on
the offending behavior. It comprises five items (number, age, gender,
relationship of victims, and child pornography) and is significantly
associated with phallometrically assessed sexual arousal to children.
The SSPI-2 is derived from information of the participants as well as
the supervision officer and is only defined for hands-on
offenders\textsuperscript{43}.

\subsubsection{ICD-11-Screener}\label{icd-11-screener}

The ICD-11-Screener\textsuperscript{44} is a self-assessment procedure
to ascertain indications for the presence of sexual dysfunction, gender
dysphoria, or paraphilic disorder. However, the instrument is not
suitable for diagnosis. It is assumed that the ICD-11-Screener can
identify individuals who have the disorders described above. The
ICD-11-Screener is not validated yet. Due to a mistake by the PI, this
questionnaire was not mentioned in the SAP.

\subsubsection{CARES}\label{cares}

The Child Sexual Abuse Risk Evaluation Self-Report (CARES) is a
composite instrument integrating the CARES-A, which assesses acute
dynamic risk factors, and the CARES-S, which measures stable dynamic
risk factors.\textsuperscript{45} The CARES total score is calculated by
summing the CARES-A and CARES-S total scores, producing a range from 0
to 102, with higher scores indicating greater severity of both acute and
stable dynamic risk factors. The CARES-A scale evaluates seven acute
dynamic risk factors associated with sexual reoffending in individuals
convicted of at least one (contact or non-contact) sexual offense
against children: (1) Preoccupation with Potential Victims, (2)
Hostility, (3) Hypersexuality, (4) Rejection of Control Measures, (5)
Emotional Crisis, (6) Loss of Social Support, and (7) Substance Use.
Comprising 21 items, the CARES-A yields scores from 0 to 42; higher
scores denote more pronounced acute dynamic risk. Reliability
coefficients for the CARES-A range from \(\alpha\) = .79 to \(\alpha\) =
.83. Analyses of social desirability revealed no significant
correlations between CARES-A scores and either the exaggeration of
positive qualities or negative qualities.\textsuperscript{45} The
CARES-S scale assesses six stable dynamic predictors of sexual
recidivism. It consists of 30 items covering: (1) Problematic Sexual
Interests, (2) Problematic Emotion-Regulation, (3) Lack of
Problem-Solving Skills, (4) Offense-Supportive Cognitions, and (5) Lack
of Social Support. Five of these subscales correspond to empirically
supported psychological risk factors for sexual
recidivism\textsuperscript{4,5}. The CARES-S total score ranges from 0
to 60; higher scores indicate stronger stable dynamic risk factors. The
reduced CARES-S achieved an internal consistency of \(\alpha\) = .53. As
with the CARES-A, no significant correlations with social desirability
measures were found in the evaluation study.\textsuperscript{45}

\subsubsection{Official re-offences}\label{official-re-offences}

Officially registered re-offenses will be assessed five years after
last-patient-out (September 2030).

\subsubsection{CVTRQ}\label{cvtrq}

The Corrections Victoria Treatment Readiness Questionnaire
(CVTRQ\textsuperscript{46}) is a self-report measure designed to assess
treatment readiness in offenders who have been referred to a cognitive
skills program. It comprises 20 items and consists of four scales: (1)
Attitudes and motivation, (2) emotional reaction, (3) offending beliefs,
and (4) efficacy. Each item has to be answered on a five-point
Likert-scale. The CVTRQ shows an acceptable convergent validity,
discriminant validity as well as predictive
validity\textsuperscript{46}. To the best of our knowledge, there is no
validated German version of the CVTRQ. @myTabu translated the CVTRQ. The
translation was checked and translated back into English. Resulting
differences due to adaptation were inspected and discussed in terms of
item content.

\subsubsection{RCQ}\label{rcq}

The Readiness to Change Questionnaire (RCQ\textsuperscript{47}) is a 12
item questionnaire originally designed to identify the stage of change
reached by individuals who excessively drink alcohol. The German version
(RCQ-D)\textsuperscript{48} was adapted within the project by changing
the questions regarding drinking of alcohol in questions regarding CSA
and CSEM offenses. Responses are made on a five-point Likert-scale. The
RCQ-D as well as the RCQ shows good psychometric
properties.\textsuperscript{48} Psychometric properties of the version
adapted for ISAC and ICCSEM are not available yet.

\subsubsection{OQMPR}\label{oqmpr}

The Questionnaire for the Measurement of Psychological Reactance
(QMPR\textsuperscript{49}) is a questionnaire for the assessment of
psychological reactance defined as the theory that people resist
attempts to constrain either their thoughts or their behaviors. The
Optimized Questionnaire for the Measurement of Psychological Reactance
(OQMPR)\textsuperscript{50} is a German variant of the QMPR. The OQMPR
consists of 12 statements on a five-point Likert-scale. It has a
test-retest-reliability of rtt = 0.85.\textsuperscript{50}

\subsubsection{F-SozU}\label{f-sozu}

Seven-item short version of the Social Support Questionnaire
(F-SozU\textsuperscript{51}) is an efficient questionnaire to assess
perceived social support. Each item comprises a five-point Likertscale.
Internal consistency of the original long version showed Cronbach's
\(\alpha\) between 0.81 and 0.93.\textsuperscript{52}

\subsubsection{UCLA}\label{ucla}

The UCLA Loneliness Scale (UCLA\textsuperscript{53}) consists of 20
items in order to assess subjective feelings of loneliness. The German
short version\textsuperscript{54} consists of 12 items on a four-point
Likert-scale. A German study with CSA and CSEM offenders demonstrated a
high reliability of \(\alpha\) = 0.92.\textsuperscript{55}

\subsubsection{DERS}\label{ders}

The Difficulties in Emotion Regulation Scale (DERS\textsuperscript{56})
is a self-report questionnaire to assess emotion dysregulation. The
sub-scale ``impulse control difficulties'' is used as outcome measure
and comprises five items on a five-point Likert-scale. The German
version is used, which shows a good internal consistency, construct and
predictive validity.\textsuperscript{57}

\subsubsection{NARQ}\label{narq}

The Negative Affect Repair Questionnaire (NARQ\textsuperscript{58}) is a
self-report questionnaire to assess strategies to regulate negative
affect in a systematic manner. It consists of 17 items on a five-point
Likert-scale and provides a good construct validity. Reliability scores
(Cronbach's \(\alpha\)) for the three NARQ scales ranged between 0.71
and 0.80.\textsuperscript{58}

\subsubsection{BIS-15}\label{bis-15}

The Barratt Impulsiveness Scale-15 (BIS-15\textsuperscript{59}) is a
questionnaire developed to assess the personality/behavioral construct
of impulsiveness. It consists of 30 items on a four-point Likert-scale.
The German short version BIS-15\textsuperscript{60} consisting of 15
items is used in the clinical trial. The BIS-15 is an efficient measure
of impulsiveness with good internal consistency of \(\alpha\) =
0.81.\textsuperscript{60}

\subsubsection{CUSI}\label{cusi}

The Coping Using Sex Inventory (CUSI\textsuperscript{61}) is a
questionnaire to assess the presence of and the degree to which sex was
used to deal with problematic situations. It consists of 16 items on a
fivepoint Likert-scale with a satisfying internal
consistency.\textsuperscript{61} The translation was checked and
translated back into English by the authors. Resulting differences due
to adaptation were inspected and discussed in terms of item content.

\subsubsection{SPSI-R}\label{spsi-r}

The Social Problem-Solving Inventory Revised
(SPSI-R\textsuperscript{24}) is a self-report questionnaire for the
assessment of the five dimensions in the social problem-solving model:
(1) positive problem orientation, (2) negative problem orientation, (3)
rational problem solving, (4) impulsivity/carelessness style, and (5)
avoidance style. The SPSI-R consists of 52 items on a five-point
Likert-scale. It shows good psychometric properties for individuals who
have sexually offended\textsuperscript{27}. The validated German version
is a short form consisting of 25 items\textsuperscript{62}.

\subsubsection{BMS}\label{bms}

The Bumby Molest Scale (BMS\textsuperscript{63}) is a self-report
questionnaire which assesses offense supportive cognitions of CSA or
CSEM offenders (German Version:\textsuperscript{64}). It consists of 38
items on a four-point Likert-scale. The German version shows good
construct validity, internal consistency, and test-retest
reliability\textsuperscript{64}.

\subsubsection{HBI-19}\label{hbi-19}

Sexual preoccupation is assessed by the Hypersexual Behavior
Inventory-19 (HBI-19\textsuperscript{65}). The HBI-19 is a three-factor
measure (coping, control, and consequences) developed to assess
hypersexual behavior. The instrument consists of 19 items (e.g., `I use
sex to forget sorrows of everyday life.') answered on a scale from 1
(never) to 5 (very often). The maximum score is 95, with higher scores
indicating a higher level of sexual preoccupation. The questionnaire was
shown to have good reliability (\(\alpha\) = 0.90) and
validity.\textsuperscript{65}

\subsubsection{SSIC}\label{ssic}

The Specific self-efficacy for modifying Sexual Interest in Children
(SSIC\textsuperscript{66}) comprises six items on the participant's
conviction regarding the ability to change their sexual interest in
children (e.g., `I can succeed in reducing my sexual interest in
children') were answered on a scale from 1 (do not agree at all) to 5
(totally agree). The maximum score is 30, with higher scores indicating
a higher level of self-efficacy. The instrument was shown to have good
reliability (\(\alpha\) = 0.87) and validity\textsuperscript{66}.

\subsubsection{SOI-R}\label{soi-r}

The Item 2a of the Sexual Outlet Inventory revised
(SOI-R\textsuperscript{67}) assesses the desire for sexual activity
involving children on a visual analog scale from 0 (desire is absent) to
100 (``I have to act to satisfy the desire)''\,``. Higher values on the
scale indicate a stronger sexual interest in
children.\textsuperscript{67}

\subsubsection{EKK-R}\label{ekk-r}

Emotional congruence with children is assessed by the Questionnaire on
Emotional Congruence with Children-Revised (EKK-R\textsuperscript{68})
including three factors (special relationship to children, immaturity,
and emotional closeness to children). Twenty items are answered on a
four-point scale. The questionnaire demonstrated good reliability
(\(\alpha\) = 0.80) and validity.\textsuperscript{68}

\subsubsection{ESIQ}\label{esiq}

The Explicit Sexual Interest Questionnaire (ESIQ\textsuperscript{69})
directly assesses pedophilic interest. It consists of two scales
measuring sexual behavior (20 items, e.g., `I enjoyed orally stimulating
a man.') and sexual fantasies (20 items, e.g., `I find it attractive to
imagine a little boy sexually stimulating me.'). All items are answered
on a scale from 1 (totally disagree) to 5 (totally agree). The
reliability of the instrument ranges between 0.86 and
0.97\textsuperscript{69}.

\subsubsection{WHO-5}\label{who-5}

The WHO-5 Well-Being Index (WHO-5\textsuperscript{70}) is a
questionnaire that measures current mental well-being with five items.
The items are rated on a six-point Likert-scale ranging from 0 (at no
time) to 5 (all of the time). The WHO-5 has shown good validity in
measuring subjective well-being in clinical studies\textsuperscript{71}.
The norm in Germany is mean 65.7. The threshold for a clinically
relevant change is defined as a change of at least 10 points on the
total score.\textsuperscript{71} Due to a mistake by the PI, this
questionnaire was not mentioned in the SAP.

\subsubsection{QNP}\label{qnp}

The Questionnaire of Non-Participation (QNP) was developed to assess
reasons for not taking part in the study. It comprises eleven items on a
six-point Likert-scale. Items are e.g., `I will not take part in the
study because I do not need any further help.' or `I will not take part
in the study because I am afraid of negative consequences if my SO gets
additional information about me.' The QNP will only be administered if a
potential participant will not take part in the study but has given
informed consent to fill out this questionnaire. Due to a mistake by the
PI, this questionnaire was not mentioned in the SAP.

\subsubsection{MRQ}\label{mrq}

The Mood and Risk Questionnaire (MRQ) consists of ten self-developed
items, of which six are asking about thoughts and behaviors associated
with potential re-offenses and four asking about the psychological and
emotional state of the participant. The MRQ was developed by the @myTabu
team in order to assess adverse events in a structured manner and with
adapted adverse events for individuals who have a history of CSA or
CSEM. The MRQ has been assessed before and after each session of the WBI
and differs between the items in the pre and post measurements: all
items of the pre measurement refer explicitly to the time between
finishing last post measurement and before starting the new session
(time not working on the WBI). The post measurement explicitly asks for
the time during working on the session. The MRQ has the following
one-item subscales: (1) Contact planning (``\ldots have you thought
about how to best make contact with a child?''); (2) Contact preparation
(``\ldots have you made preparations to be able to make contact with a
child?''); (3) Urge for CSA (``\ldots have you felt that you must commit
a sexual act with a child?''); (4) Urge for CSEM (``\ldots have you felt
that you might soon (again) watch child sexual abuse material?''); (5)
Sexual tension (``\ldots have you felt that sexual tension has built up
within you?)''; (6) Control of sexual thoughts and activity
(``\ldots have you felt that it is difficult for you to control your
sexual thoughts and activities?''); (7) Unable to cope with the mental
burden (``\ldots have you felt that you can no longer endure
burdens?''); (8) Suicidal ideations (``\ldots have you thought about
taking your own life?''); (9) Mental crisis (``At present, I feel that
my behavior and experience are impaired due to mental problems.''); (10)
Very bad negative mood (``How do you feel at the moment?''). An adverse
event was defined as the highest possible score on each one-item
self-report subscale (`Often' for subscales 1--8, `Strongly agree' for
subscale 9, and `Very bad' for subscale 10). The MRQ post questionnaire
explicitly asks for the time during working on the WBI, whereas the MRQ
pre questionnaire eclusiviely asks for the last week before measurement
timepoint. Please note, that the occurence of a report of the highest
possible score was defined as adverse event post-hoc. Due to a mistake
by the PI, this questionnaire was not mentioned in the SAP.

\subsubsection{WAI-SR}\label{wai-sr}

The Working Alliance Inventory-Short Revised
(WAI-SR\textsuperscript{72,73}) measures three dimensions of therapeutic
alliance: (a) bond, (b) task, and (c) goal. For the purpose of this
study, the original items of the German version\textsuperscript{72} were
adapted to be suitable for a web-based intervention. The items have to
be rated on a five-point Likertscale ranging from 1 (never) to 5
(always). It was adapted to suit to the online-coach (WAI-SR COACH) as
well as the supervision officer (WAI-SR SUPERVISOR) resulting in 16
items. Internal consistency (Cronbach's \(\alpha\)) of the original
questionnaire ranges from 0.82 to 0.90\textsuperscript{72}. Due to a
mistake by the PI, this questionnaire was not mentioned in the SAP.

\subsubsection{EQ}\label{eq}

The Exclusion Questionnaire (EQ) is a questionnaire to assess the
exclusion reasons of the supervision officer. The EQ was specifically
developed for this clinical trial by the project team. Only supervision
officers were able to exclude a participant from the trial by clicking a
button in the WBI. After clicking the button, the supervision officer
had to fill in the EQ before the PI was informed to exclude the
participant by Email. The EQ asks about eight possible exclusion reasons
(e.g.~re-offences, withdraw of informed consent) and one additional free
text from. Multiple answers on the eight exclusion reasons are possible.
Due to a mistake by the PI, this questionnaire was not mentioned in the
SAP.

\begin{landscape}

\begin{table}

\caption{\label{tbl-measurement-timepoints}Measurement timepoints for
all trial endpoints. Please note, that the Mood and Risk Questionnaire
(MRQ) is not listed in the table. The MRQ Pre was assessed before each
session, the MRQ Post after each session. The measurement time-points of
the WAI-SR COACH and WAI-SR SUPERVISOR have been administered before
session two in each module. Please see Fromberger et
al.\textsuperscript{1} for table of measurement timepoints with a higher
time resolution.}

\centering{

\fontsize{8.0pt}{9.0pt}\selectfont
\begin{tabular*}{\linewidth}{@{\extracolsep{\fill}}l|llrlllllllllllll}
\toprule
 &  & \multicolumn{2}{c}{Introduction} & \multicolumn{2}{c}{Module 1} & \multicolumn{2}{c}{Module 2} & \multicolumn{2}{c}{Module 3} & \multicolumn{2}{c}{Module 4} & \multicolumn{2}{c}{Module 5} & \multicolumn{2}{c}{Module 6} &  \\ 
\cmidrule(lr){3-4} \cmidrule(lr){5-6} \cmidrule(lr){7-8} \cmidrule(lr){9-10} \cmidrule(lr){11-12} \cmidrule(lr){13-14} \cmidrule(lr){15-16}
 & Allocation & pre & post & pre & post & pre & post & pre & post & pre & post & pre & post & pre & post & Follow-Up \\ 
\midrule\addlinespace[2.5pt]
\multicolumn{17}{l}{BASELINE CHARACTERISTICS} \\[2.5pt] 
\midrule\addlinespace[2.5pt]
SSQ-P &  & X &  &  &  &  &  &  &  &  &  &  &  &  &  &  \\ 
SSQ-SO &  &  &  &  &  &  &  &  &  &  &  &  &  &  &  &  \\ 
SSQ-CF & X &  &  &  &  &  &  &  &  &  &  &  &  &  &  &  \\ 
PHQ-D &  & X &  &  &  &  &  &  &  &  &  &  &  &  &  &  \\ 
SSPI-2 & X &  &  &  &  &  &  &  &  &  &  &  &  &  &  &  \\ 
ICD-11-Screener &  & X &  &  &  &  &  &  &  &  &  &  &  &  &  &  \\ 
CARES &  &  &  & X &  & X &  & X &  & X &  & X &  & X &  &  \\ 
\midrule\addlinespace[2.5pt]
\multicolumn{17}{l}{PRIMARY OUTCOMES} \\[2.5pt] 
\midrule\addlinespace[2.5pt]
Official re-offences &  &  &  &  &  &  &  &  &  &  &  &  &  &  &  & X \\ 
CVTRQ & X &  &  & X & X &  &  &  &  &  &  &  &  &  &  &  \\ 
\midrule\addlinespace[2.5pt]
\multicolumn{17}{l}{SECONDARY OUTCOMES} \\[2.5pt] 
\midrule\addlinespace[2.5pt]
RCQ &  &  &  & X & X &  &  &  &  &  &  &  &  &  &  &  \\ 
OQMPR &  &  &  &  &  & X & X &  &  &  &  &  &  &  &  &  \\ 
F-SozU &  &  &  &  &  & X & X &  &  &  &  &  &  &  &  &  \\ 
UCLA &  &  &  &  &  & X & X &  &  &  &  &  &  &  &  &  \\ 
DERS &  &  &  &  &  &  &  & X & X &  &  &  &  &  &  &  \\ 
NARQ &  &  &  &  &  &  &  & X & X &  &  &  &  &  &  &  \\ 
BIS-15 &  &  &  &  &  &  &  & X & X &  &  &  &  &  &  &  \\ 
CUSI &  &  &  &  &  &  &  & X & X &  &  &  &  &  &  &  \\ 
SPSI-R &  &  &  &  &  &  &  &  &  & X & X &  &  &  &  &  \\ 
BMS &  &  &  &  &  &  &  &  &  & X & X &  &  &  &  &  \\ 
HBI-19 &  &  &  &  &  &  &  &  &  &  &  & X & X &  &  &  \\ 
SSIC &  &  &  &  &  &  &  &  &  &  &  & X & X &  &  &  \\ 
SOI-R &  &  &  &  &  &  &  &  &  &  &  &  &  & X & X &  \\ 
EKK-R &  &  &  &  &  &  &  &  &  &  &  &  &  & X & X &  \\ 
ESIQ &  &  &  &  &  &  &  &  &  &  &  &  &  &  &  &  \\ 
SBV-R &  & X &  &  &  &  &  &  &  &  &  &  &  &  &  &  \\ 
\midrule\addlinespace[2.5pt]
\multicolumn{17}{l}{ANCILLARY OUTCOMES} \\[2.5pt] 
\midrule\addlinespace[2.5pt]
ATQ & X &  &  &  &  &  &  &  &  &  &  &  &  &  &  &  \\ 
WHO-5 &  &  &  & X & X & X & X & X & X & X & X & X & X & X & X &  \\ 
QNP & X &  &  &  &  &  &  &  &  &  &  &  &  &  &  &  \\ 
WAI-SR COACH &  &  &  & X &  & X &  & X &  & X &  & X &  & X &  &  \\ 
WAI-SR SUPERVISOR &  &  &  & X &  & X &  & X &  & X &  & X &  & X &  &  \\ 
\bottomrule
\end{tabular*}
\begin{minipage}{\linewidth}
\vspace{.05em}
\parbox{\linewidth}{\raggedright {Abbreviations: BIS-15 = Barratt Impulsiveness Scale-15, BMS = Bumby Molest Scale, CUSI = Coping Using Sex Inventory, CVTRQ = Corrections Victoria Treatment Readiness Questionnaire, DERS = Difficulties in Emotion Regulation Scale, EKK-R = Questionnaire on Emotional Congruence with Children-Revised, ESIQ = Explicit Sexual Interest Questionnaire, F-Soz-U = Seven-item short version of the Social Support Questionnaire, HBI-19 = Hypersexual Behavior Inventory-19, NARQ = Negative Affect Repair Questionnaire, OQMPR = Questionnaire for the Measurement of Psychological Reactance, RCQ = Readiness to Change Questionnaire - German version, SOI-R = Sexual Outlet Inventory revised, subscale desire for sexual activity with children, SPSI-R = Social Problem-Solving Inventory Revised, SSIC = Specific self-efficacy for modifying Sexual Interest in Children, UCLA = UCLA Loneliness Scale - German short version.\\
}}\end{minipage}

}

\end{table}%

\end{landscape}

\newpage{}

\section{Baseline characteristics}\label{baseline-characteristics}

\subsection{SSPI-2}\label{sspi-2-1}

\begin{table}

\caption{\label{tbl-sspi-2}Baseline values of the SSPI-2. Note, that the
SSPI-2 is only defined for participants who have at least one hands-on
offences.\textsuperscript{43} Abbreviations: Q1 = 25th percentile, Q3 =
75th percentile.}

\centering{

\fontsize{8.0pt}{9.0pt}\selectfont
\begin{tabular*}{\linewidth}{@{\extracolsep{\fill}}lccc}
\toprule
\textbf{Characteristic} & \textbf{Overall}  N = 221\textsuperscript{\textit{1}} & \textbf{Intervention}  N = 108\textsuperscript{\textit{1}} & \textbf{Placebo}  N = 113\textsuperscript{\textit{1}} \\ 
\midrule\addlinespace[2.5pt]
{\bfseries Any hands-off offences} &  &  &  \\ 
    Yes & 186 (84\%) & 90 (83\%) & 96 (85\%) \\ 
    No & 35 (16\%) & 18 (17\%) & 17 (15\%) \\ 
{\bfseries More than one hands-on victims} &  &  &  \\ 
    Yes & 43 (45\%) & 20 (43\%) & 23 (46\%) \\ 
    No & 53 (55\%) & 26 (57\%) & 27 (54\%) \\ 
    Missing & 125 & 62 & 63 \\ 
{\bfseries Any hands-on victims under 15 years old} &  &  &  \\ 
    Yes & 43 (44\%) & 21 (44\%) & 22 (45\%) \\ 
    No & 54 (56\%) & 27 (56\%) & 27 (55\%) \\ 
    Missing & 124 & 60 & 64 \\ 
{\bfseries Any hands-on victims under 12 years old} &  &  &  \\ 
    Yes & 68 (72\%) & 31 (66\%) & 37 (77\%) \\ 
    No & 27 (28\%) & 16 (34\%) & 11 (23\%) \\ 
    Missing & 126 & 61 & 65 \\ 
{\bfseries Any extra-familiar hands-on victims} &  &  &  \\ 
    Yes & 75 (77\%) & 35 (74\%) & 40 (80\%) \\ 
    No & 22 (23\%) & 12 (26\%) & 10 (20\%) \\ 
    Missing & 124 & 61 & 63 \\ 
{\bfseries SSPI-2 score} &  &  &  \\ 
    Median (Q1 – Q3) & 3·00 (2·00 – 4·00) & 3·00 (2·00 – 4·00) & 3·00 (2·00 – 4·00) \\ 
    Missing & 136 & 66 & 70 \\ 
\bottomrule
\end{tabular*}
\begin{minipage}{\linewidth}
\vspace{.05em}
\textsuperscript{\textit{1}} n (\%)\\
\end{minipage}

}

\end{table}%

\newpage{}

\section{Timing (protocol deviations)}\label{timing-protocol-deviations}

\subsection{Cumulative time from baseline to finish
module}\label{cumulative-time-from-baseline-to-finish-module}

\begin{figure}

\centering{

\includegraphics[width=1\linewidth,height=\textheight,keepaspectratio]{_figures/fig-time-to-post-module-rainplot.png}

}

\caption{\label{fig-time-to-post-module-rainplot}Cumulative time from
baseline to finish module. The cumulative time from baseline to finish
module is defined as time from beginning of module one to the end of the
respective module. Note, that the cumulative time from baseline to
finish module consist of the time of working on the content and time for
filling in questionnaires and time needed to revise guided tasks (and an
undefined error term for doing things not related to the WBI).}

\end{figure}%

\begin{table}

\caption{\label{tbl-time-to-post-module-mmrm}Mixed model for repeated
measures (MMRM) for the cumulative time from baseline to finish module.
The cumulative time from baseline to finish module is defined as time
from beginning of module one to the end of the respective module. Note,
that the cumulative time from baseline to finish module consist of the
time of working on the content and time for filling in questionnaires
and time needed to revise guided tasks (and an undefined error term for
doing things not related to the WBI).}

\centering{

\fontsize{8.0pt}{9.0pt}\selectfont
\begin{tabular*}{\linewidth}{@{\extracolsep{\fill}}llll}
\toprule
\textbf{Variable} & \textbf{Beta} & \textbf{95\% CI} & \textbf{p-value} \\ 
\midrule\addlinespace[2.5pt]
{\bfseries (Intercept)} & 87 & 69 to 105 & <0·0001 \\ 
{\bfseries Timepoint * treatment} &  &  &  \\ 
    Module 1 (post) * Intervention & 27 & 1·9 to 53 & 0·035 \\ 
    Module 2 (post) * Intervention & 44 & 3·3 to 85 & 0·034 \\ 
    Module 3 (post) * Intervention & 77 & 13 to 141 & 0·019 \\ 
    Module 4 (post) * Intervention & 113 & 30 to 195 & 0·0078 \\ 
    Module 5 (post) * Intervention & 126 & 30 to 223 & 0·011 \\ 
    Module 6 (post) * Intervention & 132 & 29 to 236 & 0·013 \\ 
{\bfseries Timepoint} &  &  &  \\ 
    Module 1 (post) & 0·00 & Ref. &  \\ 
    Module 2 (post) & 69 & 55 to 83 & <0·0001 \\ 
    Module 3 (post) & 149 & 118 to 180 & <0·0001 \\ 
    Module 4 (post) & 217 & 172 to 262 & <0·0001 \\ 
    Module 5 (post) & 289 & 235 to 344 & <0·0001 \\ 
    Module 6 (post) & 333 & 273 to 393 & <0·0001 \\ 
\bottomrule
\end{tabular*}
\begin{minipage}{\linewidth}
\vspace{.05em}
\parbox{\linewidth}{\raggedright {Abbreviations: CI = Confidence Interval, StGB = German penalty law\\
}}\end{minipage}

}

\end{table}%

\begin{figure}

\centering{

\includegraphics[width=1\linewidth,height=\textheight,keepaspectratio]{_figures/fig-time-to-post-module-estimated-marginal-means.png}

}

\caption{\label{fig-time-to-post-module-estimated-marginal-means}Estimated
marginal means of the MMRM for the cumulative time from baseline to
finish module. The cumulative time from baseline to finish module is
defined as time from beginning of module one to the end of the
respective module. Note, that the cumulative time from baseline to
finish module consist of the time of working on the content and time for
filling in questionnaires and time needed to revise guided tasks (and an
undefined error term for doing things not related to the WBI).}

\end{figure}%

\begin{landscape}

\begin{table}

\caption{\label{tbl-time-to-post-module-pairwise-comparisons}Pairwise
comparisons for the cumulative time from baseline to finish module is
defined as time from beginning of module one to the end of the
respective module. Note, that the cumulative time from baseline to
finish module consist of the time of working on the content and time for
filling in questionnaires and time needed to revise guided tasks (and an
undefined error term for doing things not related to the WBI).}

\centering{

\fontsize{8.0pt}{9.0pt}\selectfont
\begin{tabular*}{\linewidth}{@{\extracolsep{\fill}}lcccc}
\toprule
\textbf{Characteristic} & \textbf{Intervention}  N = 108 & \textbf{Placebo}  N = 113 & \textbf{p} & \textbf{q-value} \\ 
\midrule\addlinespace[2.5pt]
{\bfseries Module 1 (advised: 28 days)} &  &  & 0·053 & 0·24 \\ 
    Mean (SD) & 114 (110) & 87 (76) &  &  \\ 
    Median (Q1, Q3) & 69 (43, 151) & 58 (36, 112) &  &  \\ 
    Min, Max & 24, 691 & 23, 506 &  &  \\ 
    Missing & 2 & 7 &  &  \\ 
{\bfseries Module 2 (advised: 56 days)} &  &  & 0·25 & 0·30 \\ 
    Mean (SD) & 162 (126) & 145 (104) &  &  \\ 
    Median (Q1, Q3) & 116 (78, 193) & 103 (71, 192) &  &  \\ 
    Min, Max & 58, 704 & 56, 635 &  &  \\ 
    Missing & 23 & 22 &  &  \\ 
{\bfseries Module 3 (advised: 84 days)} &  &  & 0·10 & 0·30 \\ 
    Mean (SD) & 210 (135) & 185 (111) &  &  \\ 
    Median (Q1, Q3) & 165 (121, 250) & 152 (103, 229) &  &  \\ 
    Min, Max & 88, 938 & 84, 521 &  &  \\ 
    Missing & 38 & 37 &  &  \\ 
{\bfseries Module 4 (advised: 112 days)} &  &  & 0·048 & 0·24 \\ 
    Mean (SD) & 260 (138) & 223 (122) &  &  \\ 
    Median (Q1, Q3) & 226 (153, 336) & 189 (140, 248) &  &  \\ 
    Min, Max & 117, 890 & 113, 667 &  &  \\ 
    Missing & 46 & 44 &  &  \\ 
{\bfseries Module 5 (advised: 140 days)} &  &  & 0·028 & 0·17 \\ 
    Mean (SD) & 325 (161) & 258 (113) &  &  \\ 
    Median (Q1, Q3) & 290 (186, 418) & 228 (177, 279) &  &  \\ 
    Min, Max & 152, 919 & 141, 649 &  &  \\ 
    Missing & 58 & 58 &  &  \\ 
{\bfseries Module 6 (advised: 168 days)} &  &  & 0·13 & 0·30 \\ 
    Mean (SD) & 363 (181) & 309 (152) &  &  \\ 
    Median (Q1, Q3) & 329 (208, 441) & 259 (208, 347) &  &  \\ 
    Min, Max & 174, 965 & 162, 901 &  &  \\ 
    Missing & 66 & 58 &  &  \\ 
\bottomrule
\end{tabular*}
\begin{minipage}{\linewidth}
\vspace{.05em}
\parbox{\linewidth}{\raggedright {Abbreviation: Q1 = 25th percentile, Q3 = 75th percentile, p\textsubscript{adj} = Holm-Bonferroni adjusted p.\\
}}\end{minipage}

}

\end{table}%

\end{landscape}

\subsection{Time working on module}\label{time-working-on-module}

\begin{figure}

\centering{

\includegraphics[width=1\linewidth,height=\textheight,keepaspectratio]{_figures/fig-time-working-on-module-rainplot.png}

}

\caption{\label{fig-time-working-on-module-rainplot}Rainplot for time
working on content by module. The time working on content is defined as
the sum of the time from creating a data base entry for lection data (at
first visit of the lection page) to the time last change of database
entry for this lection data occured for each lection the participant
worked on in one module. This time difference can be interpreted as a
narrow estimator for the time needed to finish the content of a WBI
without the time needed for filling in questionnaires, reviews of guided
tasks, or waiting times. But it includes the time needed for first
answer of a guided exercise and times, in which the participant not
logged in or logged in but not worked on the lection.}

\end{figure}%

\begin{table}

\caption{\label{tbl-time-working-on-module-mmrm}Mixed model for repeated
measures (MMRM) for time working on content of one module of the WBI.
The time working on content is defined as the sum of the time from
creating a data base entry for lection data (at first visit of the
lection page) to the time last change of database entry for this lection
data occured for each lection the participant worked on in one module.
This time difference can be interpreted as a narrow estimator for the
time needed to finish the content of a WBI without the time needed for
filling in questionnaires, reviews of guided tasks, or waiting times.
But it includes the time needed for first answer of a guided exercise
and times, in which the participant not logged in or logged in but not
worked on the lection.}

\centering{

\fontsize{8.0pt}{9.0pt}\selectfont
\begin{tabular*}{\linewidth}{@{\extracolsep{\fill}}llll}
\toprule
\textbf{Variable} & \textbf{Beta} & \textbf{95\% CI} & \textbf{p-value} \\ 
\midrule\addlinespace[2.5pt]
{\bfseries (Intercept)} & 349 & 129 to 570 & 0·0020 \\ 
{\bfseries Timepoint * treatment} &  &  &  \\ 
    Module 1 (post) * Intervention & 360 & 49 to 671 & 0·024 \\ 
    Module 2 (post) * Intervention & 148 & -48 to 345 & 0·14 \\ 
    Module 3 (post) * Intervention & 188 & -96 to 473 & 0·19 \\ 
    Module 4 (post) * Intervention & 236 & 19 to 453 & 0·034 \\ 
    Module 5 (post) * Intervention & 324 & 6·1 to 641 & 0·046 \\ 
    Module 6 (post) * Intervention & 93 & -142 to 328 & 0·42 \\ 
{\bfseries Timepoint} &  &  &  \\ 
    Module 1 (post) & 0·00 & Ref. &  \\ 
    Module 2 (post) & -130 & -317 to 57 & 0·17 \\ 
    Module 3 (post) & -18 & -255 to 219 & 0·88 \\ 
    Module 4 (post) & -106 & -327 to 115 & 0·34 \\ 
    Module 5 (post) & -33 & -304 to 237 & 0·81 \\ 
    Module 6 (post) & -95 & -327 to 137 & 0·42 \\ 
\bottomrule
\end{tabular*}
\begin{minipage}{\linewidth}
\vspace{.05em}
\parbox{\linewidth}{\raggedright {Abbreviation: CI = Confidence Interval, StGB = German penalty law.\\
}}\end{minipage}

}

\end{table}%

\begin{figure}

\centering{

\includegraphics[width=1\linewidth,height=\textheight,keepaspectratio]{_figures/fig-time-working-on-module-estimated-marginal-means.png}

}

\caption{\label{fig-time-working-on-module-estimated-marginal-means}Estimated
marginal means for for time working on content of one module of the WBI.
The time working on content is defined as the sum of the time from
creating a data base entry for lection data (at first visit of the
lection page) to the time last change of database entry for this lection
data occured for each lection the participant worked on in one module.
This time difference can be interpreted as a narrow estimator for the
time needed to finish the content of a WBI without the time needed for
filling in questionnaires, reviews of guided tasks, or waiting times.
But it includes the time needed for first answer of a guided exercise
and times, in which the participant not logged in or logged in but not
worked on the lection.}

\end{figure}%

\begin{landscape}

\begin{table}

\caption{\label{tbl-time-working-on-module}Pairwise comparisons for time
working on content by module. The time working on content is defined as
the sum of the time from creating a data base entry for lection data (at
first visit of a lection page) to the time last change of database entry
for this lection data occured for each lection the participant worked on
in one module. This time difference can be interpreted as a narrow
estimator for the time needed to finish the content of a WBI without the
time needed for filling in questionnaires, reviews of guided tasks, or
waiting times. But it includes the time needed for first answer of a
guided exercise and times, in which the participant not logged in or
logged in but not worked on the lection.}

\centering{

\fontsize{8.0pt}{9.0pt}\selectfont
\begin{tabular*}{\linewidth}{@{\extracolsep{\fill}}lcccc}
\toprule
\textbf{Characteristic} & \textbf{Intervention}  N = 108 & \textbf{Placebo}  N = 113 & \textbf{p} & \textbf{q-value} \\ 
\midrule\addlinespace[2.5pt]
{\bfseries Module 1 (pre-tests: about 4 hours)} &  &  & 0·18 & >0·99 \\ 
    Mean (SD) & 709 (1 500) & 349 (629) &  &  \\ 
    Median (Q1, Q3) & 49 (2, 637) & 12 (2, 498) &  &  \\ 
    Min, Max & 1, 9 549 & 0, 3 242 &  &  \\ 
    Missing & 2 & 7 &  &  \\ 
{\bfseries Module 2 (pre-tests: about 4 hours)} &  &  & 0·77 & >0·99 \\ 
    Mean (SD) & 273 (668) & 218 (553) &  &  \\ 
    Median (Q1, Q3) & 11 (1, 193) & 4 (1, 157) &  &  \\ 
    Min, Max & 0, 3 825 & 0, 3 962 &  &  \\ 
    Missing & 24 & 22 &  &  \\ 
{\bfseries Module 3 (pre-tests: about 4 hours)} &  &  & 0·46 & >0·99 \\ 
    Mean (SD) & 283 (561) & 237 (680) &  &  \\ 
    Median (Q1, Q3) & 29 (2, 258) & 4 (2, 120) &  &  \\ 
    Min, Max & 1, 3 137 & 0, 4 762 &  &  \\ 
    Missing & 39 & 38 &  &  \\ 
{\bfseries Module 4 (pre-tests: about 4 hours)} &  &  & 0·89 & >0·99 \\ 
    Mean (SD) & 293 (620) & 145 (297) &  &  \\ 
    Median (Q1, Q3) & 4 (1, 173) & 4 (2, 89) &  &  \\ 
    Min, Max & 1, 3 586 & 0, 1 019 &  &  \\ 
    Missing & 47 & 45 &  &  \\ 
{\bfseries Module 5 (pre-tests: about 4 hours)} &  &  & 0·25 & >0·99 \\ 
    Mean (SD) & 398 (775) & 126 (297) &  &  \\ 
    Median (Q1, Q3) & 22 (2, 391) & 5 (3, 76) &  &  \\ 
    Min, Max & 1, 3 364 & 1, 1 464 &  &  \\ 
    Missing & 58 & 58 &  &  \\ 
{\bfseries Module 6 (pre-tests: about 4 hours)} &  &  & 0·95 & >0·99 \\ 
    Mean (SD) & 202 (603) & 139 (361) &  &  \\ 
    Median (Q1, Q3) & 2 (1, 93) & 2 (1, 40) &  &  \\ 
    Min, Max & 1, 3 564 & 0, 1 655 &  &  \\ 
    Missing & 66 & 60 &  &  \\ 
\bottomrule
\end{tabular*}
\begin{minipage}{\linewidth}
\vspace{.05em}
\parbox{\linewidth}{\raggedright {Abbreviation: Q1 = 25th percentile, Q3 = 75th percentile, p\textsubscript{adj} = Holm-Bonferroni adjusted p.\\
}}\end{minipage}

}

\end{table}%

\end{landscape}

\newpage{}

\section{Primary Outcomes}\label{primary-outcomes}

\subsection{CARES}\label{cares-1}

\begin{figure}

\centering{

\pandocbounded{\includegraphics[keepaspectratio]{_figures/fig-cares-rainplot.png}}

}

\caption{\label{fig-cares-rainplot}Rainplot of the primary endpoint
Child Sexual Abuse Risk Evaluation Self-Report (CARES total score).}

\end{figure}%

\begin{figure}

\centering{

\pandocbounded{\includegraphics[keepaspectratio]{_figures/fig-cares-estimated-marginal-means.png}}

}

\caption{\label{fig-cares-estimated-marginal-means}Estimated marginal
means of the MMRM for the primary endpoint Child Sexual Abuse Risk
Evaluation Self-Report (CARES total score).}

\end{figure}%

\begin{table}

\caption{\label{tbl-prim-cares-mmrm-time-since-baseline}Explorative MMRM
for the primary endpoint Child Sexual Abuse Risk Evaluation Self-Report
(CARES total score) with time since baseline as additional factor. Time
since baseline is defined as the time between the baseline measurement
and the respective consecutive measurement time-point.}

\centering{

\fontsize{8.0pt}{9.0pt}\selectfont
\begin{tabular*}{\linewidth}{@{\extracolsep{\fill}}llll}
\toprule
\textbf{Variable} & \textbf{Beta} & \textbf{95\% CI} & \textbf{p-value} \\ 
\midrule\addlinespace[2.5pt]
{\bfseries (Intercept)} & -0·09 & -1·7 to 1·5 & 0·91 \\ 
{\bfseries Type of supervision} &  &  &  \\ 
    community supervision & 0·00 & Ref. &  \\ 
    post-release supervision & 0·31 & -1·2 to 1·8 & 0·68 \\ 
{\bfseries Additional treatment} &  &  &  \\ 
    No & 0·00 & Ref. &  \\ 
    Yes & -1·0 & -2·2 to 0·19 & 0·10 \\ 
{\bfseries Offense type} &  &  &  \\ 
    Hands-off (only §184b StGB) & 0·00 & Ref. &  \\ 
    Hands-on (at least one conviction §176 ff StGB) & 0·99 & -0·24 to 2·2 & 0·11 \\ 
{\bfseries Baseline value} & 0·84 & 0·77 to 0·91 & <0·0001 \\ 
{\bfseries Static recidivism risk} & 0·40 & -0·11 to 0·90 & 0·12 \\ 
{\bfseries Timepoint * treatment} &  &  &  \\ 
    Module 1 (post) * Intervention & 0·47 & -0·75 to 1·7 & 0·45 \\ 
    Module 2 (post) * Intervention & -0·87 & -2·5 to 0·76 & 0·29 \\ 
    Module 3 (post) * Intervention & -0·19 & -1·8 to 1·4 & 0·82 \\ 
    Module 4 (post) * Intervention & -0·07 & -1·6 to 1·5 & 0·93 \\ 
    Module 5 (post) * Intervention & 0·07 & -1·8 to 2·0 & 0·94 \\ 
    Module 6 (post) * Intervention & -0·39 & -2·4 to 1·6 & 0·70 \\ 
{\bfseries Timepoint} &  &  &  \\ 
    Module 1 (post) & 0·00 & Ref. &  \\ 
    Module 2 (post) & 1·2 & 0·17 to 2·2 & 0·022 \\ 
    Module 3 (post) & -0·09 & -1·1 to 0·94 & 0·86 \\ 
    Module 4 (post) & -0·01 & -1·2 to 1·2 & 0·98 \\ 
    Module 5 (post) & 0·75 & -0·65 to 2·1 & 0·29 \\ 
    Module 6 (post) & 0·99 & -0·41 to 2·4 & 0·17 \\ 
{\bfseries Time since baseline (years)} & -1·2 & -2·4 to 0·01 & 0·051 \\ 
\bottomrule
\end{tabular*}
\begin{minipage}{\linewidth}
\vspace{.05em}
\parbox{\linewidth}{\raggedright {Abbreviations: CI = Confidence Interval, StGB = German penalty law\\
}}\end{minipage}

}

\end{table}%

\begin{figure}

\centering{

\pandocbounded{\includegraphics[keepaspectratio]{_figures/fig-cares-estimated-marginal-means-time-since-baseline.png}}

}

\caption{\label{fig-cares-estimated-marginal-means-time-since-baseline}Estimated
marginal means of the MMRM for the primary endpoint Child Sexual Abuse
Risk Evaluation Self-Report (CARES total score) with time since baseline
as additional factor.}

\end{figure}%

\begin{table}

\caption{\label{tbl-prim-cares-sensitivity-analysis}Sensitivity analysis
for primary endpoint Child Sexual Abuse Risk Evaluation Self-Report
(CARES total score). The table summarizes the results of a
reference-based multiple imputation followed by ANCOVA models for the
difference in the CARES compared to baseline.}

\centering{

\fontsize{8.0pt}{9.0pt}\selectfont
\begin{tabular*}{\linewidth}{@{\extracolsep{\fill}}lrrl}
\toprule
Parameter & Estimate & 95\% CI & p \\ 
\midrule\addlinespace[2.5pt]
Group differences Module 1 (post) & -0.90 & (-2.21, 1.06) & 0.385 \\ 
Least square mean [intervention] Module 1 (post) & 24.20 & (22.07, 25.21) & *< 0.001* \\ 
Least square mean [placebo] Module 1 (post) & 23.30 & (21.28, 24.98) & *< 0.001* \\ 
Group differences Module 2 (post) & 0.46 & (-1.35, 2.91) & 0.994 \\ 
Least square mean [intervention] Module 2 (post) & 22.89 & (20.56, 24.74) & *< 0.001* \\ 
Least square mean [placebo] Module 2 (post) & 23.35 & (20.23, 25.45) & *< 0.001* \\ 
Group differences Module 3 (post) & 0.88 & (-1.08, 3.53) & 0.169 \\ 
Least square mean [intervention] Module 3 (post) & 20.30 & (17.36, 22.49) & *< 0.001* \\ 
Least square mean [placebo] Module 3 (post) & 21.18 & (19.1, 23.89) & *< 0.001* \\ 
Group differences Module 4 (post) & 0.41 & (-1.6, 3.04) & 0.964 \\ 
Least square mean [intervention] Module 4 (post) & 20.09 & (17.56, 21.86) & *< 0.001* \\ 
Least square mean [placebo] Module 4 (post) & 20.50 & (18, 22.75) & *< 0.001* \\ 
Group differences Module 5 (post) & 0.36 & (-2.65, 3.61) & 0.832 \\ 
Least square mean [intervention] Module 5 (post) & 20.53 & (15.88, 22.31) & *< 0.001* \\ 
Least square mean [placebo] Module 5 (post) & 20.89 & (17.85, 23.46) & *< 0.001* \\ 
Group differences Module 6 (post) & 2.27 & (-0.81, 4.84) & 0.172 \\ 
Least square mean [intervention] Module 6 (post) & 18.09 & (15.81, 19.75) & *< 0.001* \\ 
Least square mean [placebo] Module 6 (post) & 20.35 & (18.22, 22.75) & *< 0.001* \\ 
\bottomrule
\end{tabular*}

}

\end{table}%

\begin{table}

\caption{\label{tbl-prim-cares-pairwise-intervention}Pairwise
comparisons of the primary endpoint Child Sexual Abuse Risk Evaluation
Self-Report (CARES total score) for the intervention arm.}

\centering{

\fontsize{8.0pt}{9.0pt}\selectfont
\begin{tabular*}{\linewidth}{@{\extracolsep{\fill}}lcccc}
\toprule
\textbf{Module} & \textbf{pre}  N = 106 & \textbf{post}  N = 106 & \textbf{p} & \textbf{q-value} \\ 
\midrule\addlinespace[2.5pt]
{\bfseries Module 1 (post)} &  &  & 0·0050 & 0·030 \\ 
    Mean (SD) & 14 (8) & 12 (9) &  &  \\ 
    Median (Q1, Q3) & 13 (8, 18) & 12 (6, 16) &  &  \\ 
    Min, Max & 1, 40 & 0, 40 &  &  \\ 
{\bfseries Module 2 (post)} &  &  & 0·58 & >0·99 \\ 
    Mean (SD) & 12 (8) & 12 (8) &  &  \\ 
    Median (Q1, Q3) & 12 (6, 15) & 11 (5, 17) &  &  \\ 
    Min, Max & 0, 40 & 0, 30 &  &  \\ 
    Missing & 22 & 22 &  &  \\ 
{\bfseries Module 3 (post)} &  &  & 0·010 & 0·050 \\ 
    Mean (SD) & 11 (8) & 10 (8) &  &  \\ 
    Median (Q1, Q3) & 11 (4, 17) & 9 (5, 16) &  &  \\ 
    Min, Max & 0, 28 & 0, 39 &  &  \\ 
    Missing & 38 & 38 &  &  \\ 
{\bfseries Module 4 (post)} &  &  & 0·89 & >0·99 \\ 
    Mean (SD) & 11 (8) & 11 (8) &  &  \\ 
    Median (Q1, Q3) & 10 (6, 16) & 10 (5, 16) &  &  \\ 
    Min, Max & 0, 39 & 0, 30 &  &  \\ 
    Missing & 46 & 46 &  &  \\ 
{\bfseries Module 5 (post)} &  &  & 0·44 & >0·99 \\ 
    Mean (SD) & 10 (7) & 11 (9) &  &  \\ 
    Median (Q1, Q3) & 9 (4, 14) & 10 (4, 15) &  &  \\ 
    Min, Max & 0, 30 & 0, 50 &  &  \\ 
    Missing & 56 & 56 &  &  \\ 
{\bfseries Module 6 (post)} &  &  & >0·99 & >0·99 \\ 
    Mean (SD) & 10 (7) & 9 (6) &  &  \\ 
    Median (Q1, Q3) & 8 (4, 15) & 9 (5, 13) &  &  \\ 
    Min, Max & 0, 25 & 0, 24 &  &  \\ 
    Missing & 64 & 64 &  &  \\ 
\bottomrule
\end{tabular*}
\begin{minipage}{\linewidth}
\vspace{.05em}
\parbox{\linewidth}{\raggedright {Abbreviation: Q1 = 25th percentile, Q3 = 75th percentile, p\textsubscript{adj} = Holm-Bonferroni adjusted p.\\
}}\end{minipage}

}

\end{table}%

\begin{table}

\caption{\label{tbl-prim-cares-pairwise-placebo}Pairwise comparisons of
the primary endpoint Child Sexual Abuse Risk Evaluation Self-Report
(CARES total score) for the placebo arm.}

\centering{

\fontsize{8.0pt}{9.0pt}\selectfont
\begin{tabular*}{\linewidth}{@{\extracolsep{\fill}}lcccc}
\toprule
\textbf{Module} & \textbf{pre}  N = 106 & \textbf{post}  N = 106 & \textbf{p} & \textbf{q-value} \\ 
\midrule\addlinespace[2.5pt]
{\bfseries Module 1 (post)} &  &  & <0·0001 & <0·0001 \\ 
    Mean (SD) & 12 (7) & 11 (8) &  &  \\ 
    Median (Q1, Q3) & 11 (7, 16) & 9 (4, 16) &  &  \\ 
    Min, Max & 0, 31 & 0, 35 &  &  \\ 
{\bfseries Module 2 (post)} &  &  & 0·050 & 0·21 \\ 
    Mean (SD) & 10 (8) & 11 (8) &  &  \\ 
    Median (Q1, Q3) & 8 (4, 16) & 10 (5, 14) &  &  \\ 
    Min, Max & 0, 35 & 0, 36 &  &  \\ 
    Missing & 15 & 15 &  &  \\ 
{\bfseries Module 3 (post)} &  &  & 0·041 & 0·21 \\ 
    Mean (SD) & 12 (9) & 10 (8) &  &  \\ 
    Median (Q1, Q3) & 10 (5, 18) & 8 (5, 14) &  &  \\ 
    Min, Max & 0, 36 & 0, 32 &  &  \\ 
    Missing & 31 & 31 &  &  \\ 
{\bfseries Module 4 (post)} &  &  & 0·98 & >0·99 \\ 
    Mean (SD) & 10 (7) & 10 (7) &  &  \\ 
    Median (Q1, Q3) & 8 (5, 14) & 8 (5, 14) &  &  \\ 
    Min, Max & 0, 29 & 0, 37 &  &  \\ 
    Missing & 38 & 38 &  &  \\ 
{\bfseries Module 5 (post)} &  &  & 0·78 & >0·99 \\ 
    Mean (SD) & 9 (8) & 10 (8) &  &  \\ 
    Median (Q1, Q3) & 7 (4, 13) & 7 (4, 14) &  &  \\ 
    Min, Max & 0, 37 & 1, 32 &  &  \\ 
    Missing & 51 & 51 &  &  \\ 
{\bfseries Module 6 (post)} &  &  & 0·62 & >0·99 \\ 
    Mean (SD) & 10 (8) & 10 (8) &  &  \\ 
    Median (Q1, Q3) & 7 (4, 14) & 7 (4, 13) &  &  \\ 
    Min, Max & 1, 32 & 1, 35 &  &  \\ 
    Missing & 53 & 53 &  &  \\ 
\bottomrule
\end{tabular*}
\begin{minipage}{\linewidth}
\vspace{.05em}
\parbox{\linewidth}{\raggedright {Abbreviation: Q1 = 25th percentile, Q3 = 75th percentile, p\textsubscript{adj} = Holm-Bonferroni adjusted p.\\
}}\end{minipage}

}

\end{table}%

\begin{table}

\caption{\label{tbl-prim-cares-diffscore}Pairwise comparisons of
difference from baseline of the primary endpoint Child Sexual Abuse Risk
Evaluation Self-Report (CARES total score).}

\centering{

\fontsize{8.0pt}{9.0pt}\selectfont
\begin{tabular*}{\linewidth}{@{\extracolsep{\fill}}lcccc}
\toprule
\textbf{Characteristic} & \textbf{Placebo}  N = 106 & \textbf{Intervention}  N = 106 & \textbf{p} & \textbf{q-value} \\ 
\midrule\addlinespace[2.5pt]
{\bfseries Module 1 (post)} &  &  & 0·68 & >0·99 \\ 
    Mean (SD) & -1·5 (4·2) & -1·3 (5·0) &  &  \\ 
    Median (Q1, Q3) & -1·0 (-4·0, 1·0) & -1·0 (-4·0, 2·0) &  &  \\ 
    Min, Max & -13·0, 12·0 & -21·0, 15·0 &  &  \\ 
{\bfseries Module 2 (post)} &  &  & 0·052 & 0·31 \\ 
    Mean (SD) & 1·0 (5·5) & -0·3 (3·9) &  &  \\ 
    Median (Q1, Q3) & 1·0 (-2·0, 4·0) & -1·0 (-2·0, 2·0) &  &  \\ 
    Min, Max & -17·0, 32·0 & -14·0, 8·0 &  &  \\ 
    Missing & 15 & 22 &  &  \\ 
{\bfseries Module 3 (post)} &  &  & 0·51 & >0·99 \\ 
    Mean (SD) & -1·5 (5·5) & -0·8 (4·1) &  &  \\ 
    Median (Q1, Q3) & -1·0 (-3·0, 1·0) & -1·0 (-3·0, 1·0) &  &  \\ 
    Min, Max & -34·0, 11·0 & -13·0, 12·0 &  &  \\ 
    Missing & 31 & 38 &  &  \\ 
{\bfseries Module 4 (post)} &  &  & 0·98 & >0·99 \\ 
    Mean (SD) & 0·0 (4·3) & -0·2 (3·5) &  &  \\ 
    Median (Q1, Q3) & 0·0 (-2·0, 2·0) & 0·0 (-2·0, 2·0) &  &  \\ 
    Min, Max & -11·0, 13·0 & -9·0, 9·0 &  &  \\ 
    Missing & 38 & 46 &  &  \\ 
{\bfseries Module 5 (post)} &  &  & 0·86 & >0·99 \\ 
    Mean (SD) & 0·6 (4·9) & 0·7 (4·8) &  &  \\ 
    Median (Q1, Q3) & 1·0 (-3·0, 3·0) & 0·0 (-2·0, 2·0) &  &  \\ 
    Min, Max & -14·0, 16·0 & -10·0, 21·0 &  &  \\ 
    Missing & 51 & 56 &  &  \\ 
{\bfseries Module 6 (post)} &  &  & 0·85 & >0·99 \\ 
    Mean (SD) & 0·21 (3·01) & -0·17 (3·88) &  &  \\ 
    Median (Q1, Q3) & 0·00 (-1·00, 2·00) & 0·00 (-2·00, 2·00) &  &  \\ 
    Min, Max & -9·00, 6·00 & -16·00, 9·00 &  &  \\ 
    Missing & 53 & 64 &  &  \\ 
\bottomrule
\end{tabular*}
\begin{minipage}{\linewidth}
\vspace{.05em}
\parbox{\linewidth}{\raggedright {Abbreviation: Q1 = 25th percentile, Q3 = 75th percentile, p\textsubscript{adj} = Holm-Bonferroni adjusted p.\\
}}\end{minipage}

}

\end{table}%

\begin{figure}

\centering{

\pandocbounded{\includegraphics[keepaspectratio]{_figures/fig-cares-rainplot-difference-from-baseline.png}}

}

\caption{\label{fig-cares-rainplot-difference-from-baseline}Rainplot of
primary endpoint Child Sexual Abuse Risk Evaluation Self-Report (CARES
difference from baseline scores).}

\end{figure}%

\newpage{}

\subsection{Index of Desistance (IoD)}\label{index-of-desistance-iod}

\begin{figure}

\centering{

\pandocbounded{\includegraphics[keepaspectratio]{_figures/fig-iod-rainplot.png}}

}

\caption{\label{fig-iod-rainplot}Rainplot of the Index of Desistance
(IoD total score). Note, that the IoD is the preliminary version of the
CARES and mentioned in the Statistical analysis protocol as primary
outcome.}

\end{figure}%

\begin{table}

\caption{\label{tbl-prim-iod-mmrm}Mixed model for repeated measures
(MMRM) for the Index of Desistance (IoD total score).}

\centering{

\fontsize{8.0pt}{9.0pt}\selectfont
\begin{tabular*}{\linewidth}{@{\extracolsep{\fill}}llll}
\toprule
\textbf{Variable} & \textbf{Beta} & \textbf{95\% CI} & \textbf{p-value} \\ 
\midrule\addlinespace[2.5pt]
{\bfseries (Intercept)} & 3·1 & 1·2 to 5·0 & 0·0017 \\ 
{\bfseries Type of supervision} &  &  &  \\ 
    community supervision & 0·00 & Ref. &  \\ 
    post-release supervision & 1·2 & -0·29 to 2·8 & 0·11 \\ 
{\bfseries Additional treatment} &  &  &  \\ 
    No & 0·00 & Ref. &  \\ 
    Yes & -1·3 & -2·5 to -0·10 & 0·034 \\ 
{\bfseries Offense type} &  &  &  \\ 
    Hands-off (only §184b StGB) & 0·00 & Ref. &  \\ 
    Hands-on (at least one conviction §176 ff StGB) & 1·2 & -0·08 to 2·4 & 0·065 \\ 
{\bfseries IoD\_baseline} & 0·88 & 0·83 to 0·92 & <0·0001 \\ 
{\bfseries Timepoint} &  &  &  \\ 
    Baseline & 0·00 & Ref. &  \\ 
    Module 1 (post) & -2·2 & -4·2 to -0·27 & 0·027 \\ 
    Module 2 (post) & -2·9 & -4·9 to -0·95 & 0·0040 \\ 
    Module 3 (post) & -5·0 & -7·0 to -2·9 & <0·0001 \\ 
    Module 4 (post) & -5·4 & -7·5 to -3·2 & <0·0001 \\ 
    Module 5 (post) & -3·9 & -6·2 to -1·6 & 0·0009 \\ 
    Module 6 (post) & -4·7 & -7·1 to -2·3 & 0·0001 \\ 
{\bfseries Treatment} &  &  &  \\ 
    Intervention & 0·00 & Ref. &  \\ 
    Placebo & -0·30 & -2·2 to 1·5 & 0·75 \\ 
{\bfseries Treatment * Timepoint} &  &  &  \\ 
    Placebo * Module 1 (post) & -0·50 & -3·3 to 2·3 & 0·71 \\ 
    Placebo * Module 2 (post) & 0·71 & -2·1 to 3·5 & 0·62 \\ 
    Placebo * Module 3 (post) & 0·41 & -2·5 to 3·3 & 0·78 \\ 
    Placebo * Module 4 (post) & 1·1 & -1·9 to 4·1 & 0·47 \\ 
    Placebo * Module 5 (post) & 0·84 & -2·4 to 4·0 & 0·61 \\ 
    Placebo * Module 6 (post) & 1·9 & -1·4 to 5·1 & 0·26 \\ 
\bottomrule
\end{tabular*}
\begin{minipage}{\linewidth}
\vspace{.05em}
\parbox{\linewidth}{\raggedright {Abbreviation: CI = Confidence Interval, StGB = German penalty law.\\
}}\end{minipage}

}

\end{table}%

\begin{figure}

\centering{

\pandocbounded{\includegraphics[keepaspectratio]{_figures/fig-iod-estimated-marginal-means.png}}

}

\caption{\label{fig-iod-estimated-marginal-means}Estimated marginal
means of the MMRM for the IoD total score.}

\end{figure}%

\begin{table}

\caption{\label{tbl-prim-iod-mmrm-time-since-baseline}Explorative MMRM
for Index of Desistance (IoD total score) with the additional factor
time since beginning of the first module. StGB = German penalty law.}

\centering{

\fontsize{8.0pt}{9.0pt}\selectfont
\begin{tabular*}{\linewidth}{@{\extracolsep{\fill}}llll}
\toprule
\textbf{Variable} & \textbf{Beta} & \textbf{95\% CI} & \textbf{p-value} \\ 
\midrule\addlinespace[2.5pt]
{\bfseries (Intercept)} & 0·35 & -2·5 to 3·2 & 0·81 \\ 
{\bfseries Type of supervision} &  &  &  \\ 
    community supervision & 0·00 & Ref. &  \\ 
    post-release supervision & 1·2 & -1·3 to 3·6 & 0·34 \\ 
{\bfseries Additional treatment} &  &  &  \\ 
    No & 0·00 & Ref. &  \\ 
    Yes & -1·7 & -3·6 to 0·27 & 0·092 \\ 
{\bfseries Offense type} &  &  &  \\ 
    Hands-off (only §184b StGB) & 0·00 & Ref. &  \\ 
    Hands-on (at least one conviction §176 ff StGB) & 1·8 & -0·17 to 3·8 & 0·073 \\ 
{\bfseries IoD\_baseline} & 0·83 & 0·75 to 0·90 & <0·0001 \\ 
{\bfseries Static recidivism risk} & 0·57 & -0·25 to 1·4 & 0·17 \\ 
{\bfseries Timepoint * treatment} &  &  &  \\ 
    Module 1 (post) * Intervention & 0·91 & -1·0 to 2·9 & 0·36 \\ 
    Module 2 (post) * Intervention & -0·26 & -3·0 to 2·4 & 0·85 \\ 
    Module 3 (post) * Intervention & 0·26 & -2·3 to 2·8 & 0·84 \\ 
    Module 4 (post) * Intervention & -0·55 & -3·1 to 2·0 & 0·67 \\ 
    Module 5 (post) * Intervention & -0·16 & -3·3 to 3·0 & 0·92 \\ 
    Module 6 (post) * Intervention & -1·6 & -4·6 to 1·4 & 0·30 \\ 
{\bfseries Timepoint} &  &  &  \\ 
    Module 1 (post) & 0·00 & Ref. &  \\ 
    Module 2 (post) & 0·63 & -0·99 to 2·3 & 0·44 \\ 
    Module 3 (post) & -1·6 & -3·3 to 0·08 & 0·061 \\ 
    Module 4 (post) & -1·3 & -3·3 to 0·63 & 0·18 \\ 
    Module 5 (post) & -0·33 & -2·6 to 2·0 & 0·78 \\ 
    Module 6 (post) & -0·01 & -2·3 to 2·3 & >0·99 \\ 
{\bfseries Time since baseline (years)} & -0·56 & -2·5 to 1·4 & 0·57 \\ 
\bottomrule
\end{tabular*}
\begin{minipage}{\linewidth}
\vspace{.05em}
\parbox{\linewidth}{\raggedright {Abbreviation: CI = Confidence Interval, StGB = German penalty law.\\
}}\end{minipage}

}

\end{table}%

\begin{figure}

\centering{

\pandocbounded{\includegraphics[keepaspectratio]{_figures/fig-iod-estimated-marginal-means-time-since-baseline.png}}

}

\caption{\label{fig-iod-estimated-marginal-means-time-since-baseline}Estimated
marginal means of the MMRM for the Index of Desistance (IoD total score)
with time since baseline as additional factor.}

\end{figure}%

\begin{table}

\caption{\label{tbl-prim-iod-sensitivity-analysis}Sensitivity analysis
for the Index of Desistance (IoD total score). Summarized are the
results of the reference-based multiple imputation followed by ANCOVA
models for the difference in the IoD (total score) compared to
baseline.}

\centering{

\fontsize{8.0pt}{9.0pt}\selectfont
\begin{tabular*}{\linewidth}{@{\extracolsep{\fill}}lrrl}
\toprule
Parameter & Estimate & 95\% CI & p \\ 
\midrule\addlinespace[2.5pt]
Group differences Module 1 (post) & -0.78 & (-2.52, 0.91) & 0.372 \\ 
Least square mean [intervention] Module 1 (post) & -2.09 & (-3.69, -1.49) & *< 0.001* \\ 
Least square mean [placebo] Module 1 (post) & -2.87 & (-4.4, -1.83) & *< 0.001* \\ 
Group differences Module 2 (post) & 0.08 & (-2.53, 2.96) & 0.964 \\ 
Least square mean [intervention] Module 2 (post) & -2.56 & (-4.53, -1.45) & *< 0.001* \\ 
Least square mean [placebo] Module 2 (post) & -2.47 & (-5.28, -0.12) & *< 0.001* \\ 
Group differences Module 3 (post) & -0.38 & (-2.5, 2.41) & 0.982 \\ 
Least square mean [intervention] Module 3 (post) & -4.40 & (-6.67, -2.51) & *< 0.001* \\ 
Least square mean [placebo] Module 3 (post) & -4.78 & (-6.45, -2.8) & *< 0.001* \\ 
Group differences Module 4 (post) & 0.46 & (-1.95, 3.43) & 0.836 \\ 
Least square mean [intervention] Module 4 (post) & -4.99 & (-6.59, -2.53) & *< 0.001* \\ 
Least square mean [placebo] Module 4 (post) & -4.54 & (-5.88, -2.89) & *< 0.001* \\ 
Group differences Module 5 (post) & 0.14 & (-2.34, 3.34) & 0.878 \\ 
Least square mean [intervention] Module 5 (post) & -3.75 & (-6.48, -2.2) & *< 0.001* \\ 
Least square mean [placebo] Module 5 (post) & -3.61 & (-6.38, -2.19) & *< 0.001* \\ 
Group differences Module 6 (post) & 1.52 & (-1.87, 3.49) & 0.186 \\ 
Least square mean [intervention] Module 6 (post) & -4.90 & (-6.01, -2.8) & *< 0.001* \\ 
Least square mean [placebo] Module 6 (post) & -3.38 & (-4.82, -1.99) & *< 0.001* \\ 
\bottomrule
\end{tabular*}

}

\end{table}%

\begin{table}

\caption{\label{tbl-prim-iod-pairwise-intervention}Explorative pairwise
comparison of the Index of Desistance (IoD total score) for the
intervention arm.}

\centering{

\fontsize{8.0pt}{9.0pt}\selectfont
\begin{tabular*}{\linewidth}{@{\extracolsep{\fill}}lcccc}
\toprule
\textbf{Module} & \textbf{pre}  N = 106 & \textbf{post}  N = 106 & \textbf{p} & \textbf{q-value} \\ 
\midrule\addlinespace[2.5pt]
{\bfseries Module 1 (post)} &  &  & 0·0050 & 0·025 \\ 
    Mean (SD) & 28 (13) & 25 (13) &  &  \\ 
    Median (Q1, Q3) & 26 (19, 33) & 24 (16, 30) &  &  \\ 
    Min, Max & 5, 70 & 2, 70 &  &  \\ 
{\bfseries Module 2 (post)} &  &  & 0·70 & >0·99 \\ 
    Mean (SD) & 24 (14) & 24 (13) &  &  \\ 
    Median (Q1, Q3) & 23 (15, 28) & 22 (14, 33) &  &  \\ 
    Min, Max & 2, 70 & 6, 63 &  &  \\ 
    Missing & 22 & 22 &  &  \\ 
{\bfseries Module 3 (post)} &  &  & 0·0010 & 0·0060 \\ 
    Mean (SD) & 23 (12) & 21 (13) &  &  \\ 
    Median (Q1, Q3) & 22 (13, 31) & 19 (11, 29) &  &  \\ 
    Min, Max & 6, 49 & 4, 69 &  &  \\ 
    Missing & 38 & 38 &  &  \\ 
{\bfseries Module 4 (post)} &  &  & 0·51 & >0·99 \\ 
    Mean (SD) & 21 (13) & 21 (11) &  &  \\ 
    Median (Q1, Q3) & 20 (12, 29) & 21 (11, 28) &  &  \\ 
    Min, Max & 4, 69 & 4, 57 &  &  \\ 
    Missing & 46 & 46 &  &  \\ 
{\bfseries Module 5 (post)} &  &  & 0·56 & >0·99 \\ 
    Mean (SD) & 20 (12) & 21 (14) &  &  \\ 
    Median (Q1, Q3) & 21 (10, 28) & 20 (11, 28) &  &  \\ 
    Min, Max & 5, 57 & 4, 86 &  &  \\ 
    Missing & 56 & 56 &  &  \\ 
{\bfseries Module 6 (post)} &  &  & 0·75 & >0·99 \\ 
    Mean (SD) & 19 (10) & 18 (8) &  &  \\ 
    Median (Q1, Q3) & 17 (10, 28) & 18 (11, 24) &  &  \\ 
    Min, Max & 4, 41 & 3, 34 &  &  \\ 
    Missing & 64 & 64 &  &  \\ 
\bottomrule
\end{tabular*}
\begin{minipage}{\linewidth}
\vspace{.05em}
\parbox{\linewidth}{\raggedright {Abbreviation: Q1 = 25th percentile, Q3 = 75th percentile, p\textsubscript{adj} = Holm-Bonferroni adjusted p.\\
}}\end{minipage}

}

\end{table}%

\begin{table}

\caption{\label{tbl-prim-iod-pairwise-placebo}Explorative pairwise
comparison of the Index of Desistance (IoD total score) for the placebo
arm.}

\centering{

\fontsize{8.0pt}{9.0pt}\selectfont
\begin{tabular*}{\linewidth}{@{\extracolsep{\fill}}lcccc}
\toprule
\textbf{Module} & \textbf{pre}  N = 106 & \textbf{post}  N = 106 & \textbf{p} & \textbf{q-value} \\ 
\midrule\addlinespace[2.5pt]
{\bfseries Module 1 (post)} &  &  & <0·0001 & <0·0001 \\ 
    Mean (SD) & 25 (12) & 22 (12) &  &  \\ 
    Median (Q1, Q3) & 22 (16, 32) & 19 (13, 27) &  &  \\ 
    Min, Max & 5, 58 & 5, 70 &  &  \\ 
{\bfseries Module 2 (post)} &  &  & 0·68 & >0·99 \\ 
    Mean (SD) & 22 (13) & 22 (13) &  &  \\ 
    Median (Q1, Q3) & 19 (12, 27) & 19 (14, 31) &  &  \\ 
    Min, Max & 5, 70 & 5, 66 &  &  \\ 
    Missing & 15 & 15 &  &  \\ 
{\bfseries Module 3 (post)} &  &  & 0·040 & 0·20 \\ 
    Mean (SD) & 23 (14) & 20 (12) &  &  \\ 
    Median (Q1, Q3) & 19 (14, 31) & 18 (11, 27) &  &  \\ 
    Min, Max & 5, 66 & 2, 58 &  &  \\ 
    Missing & 31 & 31 &  &  \\ 
{\bfseries Module 4 (post)} &  &  & >0·99 & >0·99 \\ 
    Mean (SD) & 20 (11) & 20 (12) &  &  \\ 
    Median (Q1, Q3) & 18 (11, 26) & 17 (12, 24) &  &  \\ 
    Min, Max & 2, 47 & 3, 64 &  &  \\ 
    Missing & 38 & 38 &  &  \\ 
{\bfseries Module 5 (post)} &  &  & 0·67 & >0·99 \\ 
    Mean (SD) & 19 (12) & 20 (13) &  &  \\ 
    Median (Q1, Q3) & 15 (11, 24) & 16 (10, 26) &  &  \\ 
    Min, Max & 3, 64 & 6, 63 &  &  \\ 
    Missing & 51 & 51 &  &  \\ 
{\bfseries Module 6 (post)} &  &  & 0·80 & >0·99 \\ 
    Mean (SD) & 20 (13) & 20 (13) &  &  \\ 
    Median (Q1, Q3) & 16 (10, 26) & 17 (10, 26) &  &  \\ 
    Min, Max & 6, 63 & 5, 65 &  &  \\ 
    Missing & 53 & 53 &  &  \\ 
\bottomrule
\end{tabular*}
\begin{minipage}{\linewidth}
\vspace{.05em}
\parbox{\linewidth}{\raggedright {Abbreviation: Q1 = 25th percentile, Q3 = 75th percentile, p\textsubscript{adj} = Holm-Bonferroni adjusted p.\\
}}\end{minipage}

}

\end{table}%

\begin{table}

\caption{\label{tbl-prim-iod-diffscore}Difference of the Index of
Desistance (IoD total score) between before (pre) and after (post) each
module.}

\centering{

\fontsize{8.0pt}{9.0pt}\selectfont
\begin{tabular*}{\linewidth}{@{\extracolsep{\fill}}lcccc}
\toprule
\textbf{Characteristic} & \textbf{Placebo}  N = 106 & \textbf{Intervention}  N = 106 & \textbf{p} & \textbf{q-value} \\ 
\midrule\addlinespace[2.5pt]
{\bfseries Module 1 (post)} &  &  & 0·45 & >0·99 \\ 
    Mean (SD) & -3 (7) & -2 (8) &  &  \\ 
    Median (Q1, Q3) & -3 (-6, 1) & -2 (-7, 3) &  &  \\ 
    Min, Max & -30, 17 & -37, 16 &  &  \\ 
{\bfseries Module 2 (post)} &  &  & 0·33 & >0·99 \\ 
    Mean (SD) & 1 (8) & 0 (7) &  &  \\ 
    Median (Q1, Q3) & 1 (-3, 5) & -1 (-4, 4) &  &  \\ 
    Min, Max & -33, 46 & -24, 18 &  &  \\ 
    Missing & 15 & 22 &  &  \\ 
{\bfseries Module 3 (post)} &  &  & 0·85 & >0·99 \\ 
    Mean (SD) & -3 (9) & -2 (7) &  &  \\ 
    Median (Q1, Q3) & -2 (-5, 1) & -2 (-6, 2) &  &  \\ 
    Min, Max & -55, 17 & -19, 24 &  &  \\ 
    Missing & 31 & 38 &  &  \\ 
{\bfseries Module 4 (post)} &  &  & 0·44 & >0·99 \\ 
    Mean (SD) & 0·3 (6·3) & -0·7 (5·4) &  &  \\ 
    Median (Q1, Q3) & 0·0 (-3·0, 4·0) & 0·0 (-4·5, 3·0) &  &  \\ 
    Min, Max & -19·0, 21·0 & -12·0, 11·0 &  &  \\ 
    Missing & 38 & 46 &  &  \\ 
{\bfseries Module 5 (post)} &  &  & 0·88 & >0·99 \\ 
    Mean (SD) & 1·0 (8·4) & 0·9 (7·1) &  &  \\ 
    Median (Q1, Q3) & 2·0 (-4·0, 4·0) & 0·0 (-3·0, 3·0) &  &  \\ 
    Min, Max & -24·0, 29·0 & -16·0, 32·0 &  &  \\ 
    Missing & 51 & 56 &  &  \\ 
{\bfseries Module 6 (post)} &  &  & 0·76 & >0·99 \\ 
    Mean (SD) & 0·3 (4·3) & -0·6 (5·5) &  &  \\ 
    Median (Q1, Q3) & -1·0 (-3·0, 3·0) & 0·0 (-4·0, 3·0) &  &  \\ 
    Min, Max & -7·0, 13·0 & -22·0, 9·0 &  &  \\ 
    Missing & 53 & 64 &  &  \\ 
\bottomrule
\end{tabular*}
\begin{minipage}{\linewidth}
\vspace{.05em}
\parbox{\linewidth}{\raggedright {Abbreviation: Q1 = 25th percentile, Q3 = 75th percentile, p\textsubscript{adj} = Holm-Bonferroni adjusted p.\\
}}\end{minipage}

}

\end{table}%

\begin{figure}

\centering{

\pandocbounded{\includegraphics[keepaspectratio]{_figures/fig-iod-rainplot-difference-from-baseline.png}}

}

\caption{\label{fig-iod-rainplot-difference-from-baseline}Rainplot of
the differences from baseline for the Index of Desistance (IoD total
score).}

\end{figure}%

\newpage{}

\section{Secondary Outcomes}\label{secondary-outcomes}

\subsection{Differences from baseline}\label{differences-from-baseline}

\begin{table}

\caption{\label{tbl-sec-diffscore}Difference between before (pre) and
after (post) for each secondary endpoint. Differences (from pre to post)
of secondary endpoints have been compared between the treatment groups
using the Brunner-Munzel test for unpaired data.}

\centering{

\fontsize{8.0pt}{9.0pt}\selectfont
\begin{tabular*}{\linewidth}{@{\extracolsep{\fill}}lcccc}
\toprule
\textbf{Characteristic} & \textbf{Placebo}  N = 113 & \textbf{Intervention}  N = 108 & \textbf{p} & \textbf{q-value} \\ 
\midrule\addlinespace[2.5pt]
{\bfseries CVTRQ (total score)} &  &  & 0·19 & >0·99 \\ 
    Mean (SD) & 0·5 (4·9) & 1·5 (5·7) &  &  \\ 
    Median (Q1, Q3) & 0·0 (-3·0, 3·0) & 1·0 (-2·0, 5·0) &  &  \\ 
    Min, Max & -11·0, 15·0 & -12·0, 28·0 &  &  \\ 
{\bfseries RCQ (total score)} &  &  & 0·21 & >0·99 \\ 
    Mean (SD) & 0·3 (4·5) & -0·5 (4·6) &  &  \\ 
    Median (Q1, Q3) & 0·0 (-3·0, 3·0) & -1·0 (-4·0, 3·0) &  &  \\ 
    Min, Max & -14·0, 14·0 & -12·0, 13·0 &  &  \\ 
{\bfseries F-Soz-U (total score)} &  &  & 0·22 & >0·99 \\ 
    Mean (SD) & 0·14 (0·46) & 0·01 (0·56) &  &  \\ 
    Median (Q1, Q3) & 0·14 (-0·14, 0·43) & 0·00 (-0·29, 0·43) &  &  \\ 
    Min, Max & -0·86, 1·14 & -2·29, 1·29 &  &  \\ 
    Missing & 19 & 17 &  &  \\ 
{\bfseries OQMPR (total score)} &  &  & 0·94 & >0·99 \\ 
    Mean (SD) & -0·9 (5·4) & -1·3 (5·9) &  &  \\ 
    Median (Q1, Q3) & -1·0 (-5·0, 2·0) & -1·0 (-4·0, 3·0) &  &  \\ 
    Min, Max & -15·0, 18·0 & -21·0, 20·0 &  &  \\ 
    Missing & 19 & 17 &  &  \\ 
{\bfseries UCLA (total score)} &  &  & 0·28 & >0·99 \\ 
    Mean (SD) & -0·11 (3·23) & -0·54 (2·98) &  &  \\ 
    Median (Q1, Q3) & 0·00 (-2·00, 2·00) & 0·00 (-2·00, 1·00) &  &  \\ 
    Min, Max & -13·00, 10·00 & -12·00, 11·00 &  &  \\ 
    Missing & 19 & 17 &  &  \\ 
{\bfseries BIS-15 (total score)} &  &  & 0·92 & >0·99 \\ 
    Mean (SD) & 0·0 (3·9) & 0·0 (3·4) &  &  \\ 
    Median (Q1, Q3) & -0·5 (-2·0, 2·0) & 0·0 (-2·0, 2·0) &  &  \\ 
    Min, Max & -19·0, 11·0 & -10·0, 10·0 &  &  \\ 
    Missing & 35 & 36 &  &  \\ 
{\bfseries CUSI (total score)} &  &  & 0·55 & >0·99 \\ 
    Mean (SD) & 0·8 (5·3) & 0·4 (6·5) &  &  \\ 
    Median (Q1, Q3) & 0·0 (-2·0, 2·0) & 0·0 (-3·0, 2·5) &  &  \\ 
    Min, Max & -8·0, 23·0 & -11·0, 27·0 &  &  \\ 
    Missing & 35 & 36 &  &  \\ 
{\bfseries DERS (subscale impulsivity)} &  &  & 0·024 & 0·36 \\ 
    Mean (SD) & -0·15 (1·24) & 0·49 (2·10) &  &  \\ 
    Median (Q1, Q3) & 0·00 (0·00, 0·00) & 0·00 (0·00, 1·00) &  &  \\ 
    Min, Max & -4·00, 5·00 & -5·00, 7·00 &  &  \\ 
    Missing & 35 & 36 &  &  \\ 
{\bfseries NARQ (subscale externalizing strategies)} &  &  & 0·30 & >0·99 \\ 
    Mean (SD) & -0·29 (2·53) & 0·17 (2·22) &  &  \\ 
    Median (Q1, Q3) & 0·00 (-1·00, 1·00) & 0·00 (0·00, 1·00) &  &  \\ 
    Min, Max & -16·00, 4·00 & -12·00, 5·00 &  &  \\ 
    Missing & 35 & 36 &  &  \\ 
{\bfseries SPSI-R (total score)} &  &  & 0·97 & >0·99 \\ 
    Mean (SD) & 0·02 (1·46) & 0·08 (1·66) &  &  \\ 
    Median (Q1, Q3) & 0·00 (-0·90, 0·80) & 0·00 (-1·00, 1·20) &  &  \\ 
    Min, Max & -4·20, 4·60 & -4·00, 4·40 &  &  \\ 
    Missing & 41 & 46 &  &  \\ 
{\bfseries BMS (total score)} &  &  & 0·14 & >0·99 \\ 
    Mean (SD) & -1 (7) & 2 (10) &  &  \\ 
    Median (Q1, Q3) & 0 (-4, 1) & 0 (-3, 5) &  &  \\ 
    Min, Max & -20, 23 & -12, 39 &  &  \\ 
    Missing & 54 & 55 &  &  \\ 
{\bfseries EKK-R (total score)} &  &  & 0·49 & >0·99 \\ 
    Mean (SD) & 0·3 (3·8) & -0·6 (4·8) &  &  \\ 
    Median (Q1, Q3) & 0·0 (-2·0, 3·0) & 0·0 (-2·0, 2·0) &  &  \\ 
    Min, Max & -9·0, 9·0 & -17·0, 9·0 &  &  \\ 
    Missing & 60 & 66 &  &  \\ 
{\bfseries ESIQ (total score)} &  &  & 0·0020 & 0·032 \\ 
    Mean (SD) & -0·1 (7·9) & -3·6 (5·4) &  &  \\ 
    Median (Q1, Q3) & 0·0 (-1·0, 2·0) & -3·0 (-8·0, 0·0) &  &  \\ 
    Min, Max & -22·0, 31·0 & -14·0, 11·0 &  &  \\ 
    Missing & 60 & 66 &  &  \\ 
{\bfseries HBI-19 (total score)} &  &  & 0·92 & >0·99 \\ 
    Mean (SD) & 0·0 (5·5) & -1·0 (6·9) &  &  \\ 
    Median (Q1, Q3) & 0·0 (-2·0, 2·0) & 0·0 (-4·0, 2·0) &  &  \\ 
    Min, Max & -11·0, 18·0 & -23·0, 18·0 &  &  \\ 
    Missing & 60 & 66 &  &  \\ 
{\bfseries SSIC (total score)} &  &  & 0·077 & >0·99 \\ 
    Mean (SD) & 0·1 (4·5) & 0·6 (3·6) &  &  \\ 
    Median (Q1, Q3) & 0·0 (-1·0, 0·0) & 0·0 (0·0, 2·0) &  &  \\ 
    Min, Max & -18·0, 20·0 & -12·0, 9·0 &  &  \\ 
    Missing & 60 & 66 &  &  \\ 
{\bfseries SOI (Item 2a)} &  &  & 0·061 & 0·85 \\ 
    Mean (SD) & 1 (6) & -1 (11) &  &  \\ 
    Median (Q1, Q3) & 0 (0, 0) & 0 (-1, 0) &  &  \\ 
    Min, Max & -19, 22 & -43, 45 &  &  \\ 
    Missing & 60 & 66 &  &  \\ 
\bottomrule
\end{tabular*}
\begin{minipage}{\linewidth}
\vspace{.05em}
\parbox{\linewidth}{\raggedright {Abbreviation: BIS-15 = Barratt Impulsiveness Scale-15, BMS = Bumby Molest Scale, CUSI = Coping Using Sex Inventory, CVTRQ = Corrections Victoria Treatment Readiness Questionnaire, DERS = Difficulties in Emotion Regulation Scale, EKK-R = Questionnaire on Emotional Congruence with Children-Revised, ESIQ = Explicit Sexual Interest Questionnaire, F-Soz-U = Seven-item short version of the Social Support Questionnaire, HBI-19 = Hypersexual Behavior Inventory-19, NARQ = Negative Affect Repair Questionnaire, OQMPR = Questionnaire for the Measurement of Psychological Reactance, Q1 = 25th percentile, Q3 = 75th percentile, RCQ = Readiness to Change Questionnaire - German version, SOI-R = Sexual Outlet Inventory revised, subscale desire for sexual activity with children, SPSI-R = Social Problem-Solving Inventory Revised, SSIC = Specific self-efficacy for modifying Sexual Interest in Children, UCLA = UCLA Loneliness Scale - German short version, p\textsubscript{adj} = Holm-Bonferroni adjusted p.\\
}}\end{minipage}

}

\end{table}%

\newpage{}

\subsection{CVTRQ}\label{cvtrq-1}

\begin{figure}

\centering{

\pandocbounded{\includegraphics[keepaspectratio]{_figures/fig-sec-rainplot-cvtrq-calc-total.png}}

}

\caption{\label{fig-sec-rainplot-cvtrq}Rainplot of the Corrections
Victoria Treatment Readiness Questionnaire (CVTRQ, total score).}

\end{figure}%

\begin{figure}

\centering{

\pandocbounded{\includegraphics[keepaspectratio]{_figures/fig-sec-diagnostic-plots-cvtrq-calc-total.png}}

}

\caption{\label{fig-sec-rainplot-cvtrq-calc-total}Diagnostic plots for
the Corrections Victoria Treatment Readiness Questionnaire (CVTRQ, total
score).}

\end{figure}%

\begin{table}

\caption{\label{tbl-sec-lm-cvtrq-calc-total}Linear regression model for
the Corrections Victoria Treatment Readiness Questionnaire CVTRQ (total
score); Number of observations: 221).}

\centering{

\fontsize{8.0pt}{9.0pt}\selectfont
\begin{tabular*}{\linewidth}{@{\extracolsep{\fill}}lccccc}
\toprule
\textbf{Variable} & \textbf{N} & \textbf{β} & \textbf{95\% CI} & \textbf{p-value} & \textbf{adjusted p value} \\ 
\midrule\addlinespace[2.5pt]
{\bfseries (Intercept)} & 221 & 18 & 9.1, 26 & <0.001 &  \\ 
{\bfseries Baseline value (pre)} & 221 & 0.81 & 0.71, 0.91 & <0.001 &  \\ 
{\bfseries Treatment} &  &  &  &  &  \\ 
    Placebo & 113 & 0.00 & Ref. &  &  \\ 
    Intervention & 108 & 0.89 & -0.49, 2.3 & 0.2 &  1.00 \\ 
{\bfseries Offense type} &  &  &  &  &  \\ 
    Hands-on (at least one conviction §176 ff StGB) & 89 & 0.00 & Ref. &  &  \\ 
    Hands-off (only §184b StGB) & 132 & -0.22 & -1.8, 1.3 & 0.8 &  1.00 \\ 
{\bfseries Type of supervision} &  &  &  &  &  \\ 
    community supervision & 176 & 0.00 & Ref. &  &  \\ 
    post-release supervision & 45 & -0.51 & -2.4, 1.4 & 0.6 &  1.00 \\ 
{\bfseries Additional treatment} &  &  &  &  &  \\ 
    No & 148 & 0.00 & Ref. &  &  \\ 
    Yes & 73 & 0.24 & -1.3, 1.7 & 0.7 &  1.00 \\ 
{\bfseries Static recidivism risk (baseline)} & 221 & -0.47 & -1.1, 0.17 & 0.2 &  1.00 \\ 
\bottomrule
\end{tabular*}
\begin{minipage}{\linewidth}
\vspace{.05em}
\parbox{\linewidth}{\raggedright {Abbreviation: CI = Confidence Interval\\
}}\end{minipage}

}

\end{table}%

\newpage{}

\subsection{RCQ}\label{rcq-1}

\begin{figure}

\centering{

\pandocbounded{\includegraphics[keepaspectratio]{_figures/fig-sec-rainplot-rcq-calc-total.png}}

}

\caption{\label{fig-sec-rainplot-cvtrq-calc-total}Rainplot of the
Readiness to Change Questionnaire (RCQ, total score).}

\end{figure}%

\begin{figure}

\centering{

\pandocbounded{\includegraphics[keepaspectratio]{_figures/fig-sec-diagnostic-plots-rcq-calc-total.png}}

}

\caption{\label{fig-sec-rainplot-rcq-calc-total}Diagnostic plots for the
Readiness to Change Questionnaire (RCQ, total score). The plots on the
left correspond to the model fitted with all data points, the plots on
the right correspond to the model after the removal of identified
outliers. Both models models showed similar results.}

\end{figure}%

\begin{table}

\caption{\label{tbl-sec-lm-rcq-calc-total}Linear regression model for
the Readiness to Change Questionnaire RCQ (total score); Number of
observations: 221).}

\centering{

\fontsize{8.0pt}{9.0pt}\selectfont
\begin{tabular*}{\linewidth}{@{\extracolsep{\fill}}lccccc}
\toprule
\textbf{Variable} & \textbf{N} & \textbf{β} & \textbf{95\% CI} & \textbf{p-value} & \textbf{adjusted p value} \\ 
\midrule\addlinespace[2.5pt]
{\bfseries (Intercept)} & 221 & 19 & 16, 23 & <0.001 &  \\ 
{\bfseries Baseline value (pre)} & 221 & 0.27 & 0.15, 0.40 & <0.001 &  \\ 
{\bfseries Treatment} &  &  &  &  &  \\ 
    Placebo & 113 & 0.00 & Ref. &  &  \\ 
    Intervention & 108 & -0.89 & -1.8, 0.07 & 0.069 & 0.826 \\ 
{\bfseries Offense type} &  &  &  &  &  \\ 
    Hands-on (at least one conviction §176 ff StGB) & 89 & 0.00 & Ref. &  &  \\ 
    Hands-off (only §184b StGB) & 132 & 0.12 & -0.95, 1.2 & 0.8 & 1.000 \\ 
{\bfseries Type of supervision} &  &  &  &  &  \\ 
    community supervision & 176 & 0.00 & Ref. &  &  \\ 
    post-release supervision & 45 & -0.84 & -2.2, 0.48 & 0.2 & 1.000 \\ 
{\bfseries Additional treatment} &  &  &  &  &  \\ 
    No & 148 & 0.00 & Ref. &  &  \\ 
    Yes & 73 & -0.40 & -1.4, 0.64 & 0.4 & 1.000 \\ 
{\bfseries Static recidivism risk (baseline)} & 221 & -0.08 & -0.53, 0.36 & 0.7 & 1.000 \\ 
\bottomrule
\end{tabular*}
\begin{minipage}{\linewidth}
\vspace{.05em}
\parbox{\linewidth}{\raggedright {Abbreviation: CI = Confidence Interval\\
}}\end{minipage}

}

\end{table}%

\newpage{}

\subsection{F-Soz-U}\label{f-soz-u}

\begin{figure}

\centering{

\pandocbounded{\includegraphics[keepaspectratio]{_figures/fig-sec-rainplot-fsozu-calc-total.png}}

}

\caption{\label{fig-sec-rainplot-cvtrq-calc-total}Rainplot of the Social
Support Questionnaire (F-Soz-U, total score).}

\end{figure}%

\begin{figure}

\centering{

\pandocbounded{\includegraphics[keepaspectratio]{_figures/fig-sec-diagnostic-plots-fsozu-calc-total.png}}

}

\caption{\label{fig-sec-rainplot-fsozu-calc-total}Diagnostic plots for
the Social Support Questionnaire (F-Soz-U, total score).}

\end{figure}%

\begin{table}

\caption{\label{tbl-sec-lm-fsozu-calc-total}Linear regression model for
the Social Support Questionnaire F-Soz-U (total score); Number of
observations: 185).}

\centering{

\fontsize{8.0pt}{9.0pt}\selectfont
\begin{tabular*}{\linewidth}{@{\extracolsep{\fill}}lccccc}
\toprule
\textbf{Variable} & \textbf{N} & \textbf{β} & \textbf{95\% CI} & \textbf{p-value} & \textbf{adjusted p value} \\ 
\midrule\addlinespace[2.5pt]
{\bfseries (Intercept)} & 185 & 0.50 & 0.12, 0.89 & 0.010 &  \\ 
{\bfseries Baseline value (pre)} & 185 & 0.94 & 0.85, 1.0 & <0.001 &  \\ 
{\bfseries Treatment} &  &  &  &  &  \\ 
    Placebo & 94 & 0.00 & Ref. &  &  \\ 
    Intervention & 91 & -0.13 & -0.28, 0.02 & 0.092 &  1.00 \\ 
{\bfseries Offense type} &  &  &  &  &  \\ 
    Hands-on (at least one conviction §176 ff StGB) & 72 & 0.00 & Ref. &  &  \\ 
    Hands-off (only §184b StGB) & 113 & -0.05 & -0.22, 0.12 & 0.5 &  1.00 \\ 
{\bfseries Type of supervision} &  &  &  &  &  \\ 
    community supervision & 151 & 0.00 & Ref. &  &  \\ 
    post-release supervision & 34 & -0.04 & -0.25, 0.17 & 0.7 &  1.00 \\ 
{\bfseries Additional treatment} &  &  &  &  &  \\ 
    No & 122 & 0.00 & Ref. &  &  \\ 
    Yes & 63 & -0.10 & -0.26, 0.06 & 0.2 &  1.00 \\ 
{\bfseries Static recidivism risk (baseline)} & 185 & -0.04 & -0.11, 0.03 & 0.3 &  1.00 \\ 
\bottomrule
\end{tabular*}
\begin{minipage}{\linewidth}
\vspace{.05em}
\parbox{\linewidth}{\raggedright {Abbreviation: CI = Confidence Interval\\
}}\end{minipage}

}

\end{table}%

\newpage{}

\subsection{OQMPR}\label{oqmpr-1}

\begin{figure}

\centering{

\pandocbounded{\includegraphics[keepaspectratio]{_figures/fig-sec-rainplot-ors-calc-total.png}}

}

\caption{\label{fig-sec-rainplot-ors-total-score}Rainplot of the
Questionnaire for the Measurement of Psychological Reactance (OQMPR,
total score).}

\end{figure}%

\begin{figure}

\centering{

\pandocbounded{\includegraphics[keepaspectratio]{_figures/fig-sec-diagnostic-plots-ors-calc-total.png}}

}

\caption{\label{fig-sec-rainplot-ors-calc-total}Diagnostic plots for the
Questionnaire for the Measurement of Psychological Reactance (OQMPR,
total score).}

\end{figure}%

\begin{table}

\caption{\label{tbl-sec-lm-ors-calc-total}Linear regression model for
the Questionnaire for the Measurement of Psychological Reactance OQMPR
(total score) with n = 185 observations.}

\centering{

\fontsize{8.0pt}{9.0pt}\selectfont
\begin{tabular*}{\linewidth}{@{\extracolsep{\fill}}lccccc}
\toprule
\textbf{Variable} & \textbf{N} & \textbf{β} & \textbf{95\% CI} & \textbf{p-value} & \textbf{adjusted p value} \\ 
\midrule\addlinespace[2.5pt]
{\bfseries (Intercept)} & 185 & 1.8 & -2.0, 5.6 & 0.3 &  \\ 
{\bfseries Baseline value (pre)} & 185 & 0.89 & 0.79, 1.0 & <0.001 &  \\ 
{\bfseries Treatment} &  &  &  &  &  \\ 
    Placebo & 94 & 0.00 & Ref. &  &  \\ 
    Intervention & 91 & -0.36 & -2.0, 1.3 & 0.7 &  1.00 \\ 
{\bfseries Offense type} &  &  &  &  &  \\ 
    Hands-on (at least one conviction §176 ff StGB) & 72 & 0.00 & Ref. &  &  \\ 
    Hands-off (only §184b StGB) & 113 & -1.1 & -2.9, 0.68 & 0.2 &  1.00 \\ 
{\bfseries Type of supervision} &  &  &  &  &  \\ 
    community supervision & 151 & 0.00 & Ref. &  &  \\ 
    post-release supervision & 34 & -0.44 & -2.7, 1.8 & 0.7 &  1.00 \\ 
{\bfseries Additional treatment} &  &  &  &  &  \\ 
    No & 122 & 0.00 & Ref. &  &  \\ 
    Yes & 63 & -0.13 & -1.9, 1.6 & 0.9 &  1.00 \\ 
{\bfseries Static recidivism risk (baseline)} & 185 & 0.68 & -0.09, 1.4 & 0.083 &  1.00 \\ 
\bottomrule
\end{tabular*}
\begin{minipage}{\linewidth}
\vspace{.05em}
\parbox{\linewidth}{\raggedright {Abbreviation: CI = Confidence Interval\\
}}\end{minipage}

}

\end{table}%

\newpage{}

\subsection{SOI-R}\label{soi-r-1}

\begin{figure}

\centering{

\pandocbounded{\includegraphics[keepaspectratio]{_figures/fig-sec-rainplot-soi-total-score.png}}

}

\caption{\label{fig-sec-rainplot-soi-total-score}Rainplot of the Sexual
Outlet Inventory revised (SOI-R, Item 2a).}

\end{figure}%

\begin{figure}

\centering{

\pandocbounded{\includegraphics[keepaspectratio]{_figures/fig-sec-diagnostic-plots-soi-total-score.png}}

}

\caption{\label{fig-sec-rainplot-soi-total-score}Diagnostic plots for
the Sexual Outlet Inventory revised (SOI-R, Item 2a).}

\end{figure}%

\newpage{}

\subsection{UCLA}\label{ucla-1}

\begin{figure}

\centering{

\pandocbounded{\includegraphics[keepaspectratio]{_figures/fig-sec-rainplot-ucla-calc-total.png}}

}

\caption{\label{fig-sec-rainplot-ucla-total-score}Rainplot of the UCLA
Loneliness Scale (UCLA, total score).}

\end{figure}%

\begin{figure}

\centering{

\pandocbounded{\includegraphics[keepaspectratio]{_figures/fig-sec-diagnostic-plots-ucla-calc-total.png}}

}

\caption{\label{fig-sec-rainplot-ucla-calc-total}Diagnostic plots for
the UCLA Loneliness Scale (UCLA, total score). The plots on the left
correspond to the model fitted with all data points, the plots on the
right correspond to the model after the removal of identified outliers.
Both models models showed similar results.}

\end{figure}%

\begin{table}

\caption{\label{tbl-sec-lm-ucla-calc-total}Linear regression model for
the UCLA Loneliness Scale UCLA (total score) with n = 185 observations.}

\centering{

\fontsize{8.0pt}{9.0pt}\selectfont
\begin{tabular*}{\linewidth}{@{\extracolsep{\fill}}lccccc}
\toprule
\textbf{Variable} & \textbf{N} & \textbf{β} & \textbf{95\% CI} & \textbf{p-value} & \textbf{adjusted p value} \\ 
\midrule\addlinespace[2.5pt]
{\bfseries (Intercept)} & 185 & 5.2 & 1.8, 8.6 & 0.003 &  \\ 
{\bfseries Baseline value (pre)} & 185 & 0.82 & 0.70, 0.94 & <0.001 &  \\ 
{\bfseries Treatment} &  &  &  &  &  \\ 
    Placebo & 94 & 0.00 & Ref. &  &  \\ 
    Intervention & 91 & -0.40 & -1.3, 0.49 & 0.4 &  1.00 \\ 
{\bfseries Offense type} &  &  &  &  &  \\ 
    Hands-on (at least one conviction §176 ff StGB) & 72 & 0.00 & Ref. &  &  \\ 
    Hands-off (only §184b StGB) & 113 & -0.62 & -1.6, 0.37 & 0.2 &  1.00 \\ 
{\bfseries Type of supervision} &  &  &  &  &  \\ 
    community supervision & 151 & 0.00 & Ref. &  &  \\ 
    post-release supervision & 34 & -0.47 & -1.7, 0.77 & 0.5 &  1.00 \\ 
{\bfseries Additional treatment} &  &  &  &  &  \\ 
    No & 122 & 0.00 & Ref. &  &  \\ 
    Yes & 63 & 0.08 & -0.88, 1.0 & 0.9 &  1.00 \\ 
{\bfseries Static recidivism risk (baseline)} & 185 & 0.11 & -0.31, 0.52 & 0.6 &  1.00 \\ 
\bottomrule
\end{tabular*}
\begin{minipage}{\linewidth}
\vspace{.05em}
\parbox{\linewidth}{\raggedright {Abbreviation: CI = Confidence Interval\\
}}\end{minipage}

}

\end{table}%

\newpage{}

\subsection{BIS-15}\label{bis-15-1}

\begin{figure}

\centering{

\pandocbounded{\includegraphics[keepaspectratio]{_figures/fig-sec-rainplot-bis-calc-total.png}}

}

\caption{\label{fig-sec-rainplot-bis-total-score}Rainplot of the Barratt
Impulsiveness Scale-15 (BIS-15, total score).}

\end{figure}%

\begin{figure}

\centering{

\pandocbounded{\includegraphics[keepaspectratio]{_figures/fig-sec-diagnostic-plots-bis-calc-total.png}}

}

\caption{\label{fig-sec-rainplot-bis-calc-total}Diagnostic plots for the
Barratt Impulsiveness Scale-15 (BIS-15, total score). The plots on the
left correspond to an ordinary linear model, the plots on the right
correspond to the model where the bis score was log-transformed.}

\end{figure}%

\begin{table}

\caption{\label{tbl-sec-lm-bis-calc-total}Linear regression model for
the Barratt Impulsiveness Scale-15 BIS-15 (total score) with n = 150
observations.}

\centering{

\fontsize{8.0pt}{9.0pt}\selectfont
\begin{tabular*}{\linewidth}{@{\extracolsep{\fill}}lccccc}
\toprule
\textbf{Variable} & \textbf{N} & \textbf{β} & \textbf{95\% CI} & \textbf{p-value} & \textbf{adjusted p value} \\ 
\midrule\addlinespace[2.5pt]
{\bfseries (Intercept)} & 150 & 3.2 & 0.41, 6.0 & 0.025 &  \\ 
{\bfseries Baseline value (pre)} & 150 & 0.88 & 0.80, 0.97 & <0.001 &  \\ 
{\bfseries Treatment} &  &  &  &  &  \\ 
    Placebo & 78 & 0.00 & Ref. &  &  \\ 
    Intervention & 72 & 0.15 & -1.0, 1.3 & 0.8 &  1.00 \\ 
{\bfseries Offense type} &  &  &  &  &  \\ 
    Hands-on (at least one conviction §176 ff StGB) & 58 & 0.00 & Ref. &  &  \\ 
    Hands-off (only §184b StGB) & 92 & 0.77 & -0.56, 2.1 & 0.3 &  1.00 \\ 
{\bfseries Type of supervision} &  &  &  &  &  \\ 
    community supervision & 120 & 0.00 & Ref. &  &  \\ 
    post-release supervision & 30 & 0.67 & -0.93, 2.3 & 0.4 &  1.00 \\ 
{\bfseries Additional treatment} &  &  &  &  &  \\ 
    No & 95 & 0.00 & Ref. &  &  \\ 
    Yes & 55 & 0.01 & -1.2, 1.2 & >0.9 &  1.00 \\ 
{\bfseries Static recidivism risk (baseline)} & 150 & -0.27 & -0.83, 0.28 & 0.3 &  1.00 \\ 
\bottomrule
\end{tabular*}
\begin{minipage}{\linewidth}
\vspace{.05em}
\parbox{\linewidth}{\raggedright {Abbreviation: CI = Confidence Interval\\
}}\end{minipage}

}

\end{table}%

\newpage{}

\subsection{CUSI}\label{cusi-1}

\begin{figure}

\centering{

\pandocbounded{\includegraphics[keepaspectratio]{_figures/fig-sec-rainplot-cusi-calc-total.png}}

}

\caption{\label{fig-sec-rainplot-cusi-total-score}Rainplot of the Coping
Using Sex Inventory (CUSI, total score).}

\end{figure}%

\begin{figure}

\centering{

\pandocbounded{\includegraphics[keepaspectratio]{_figures/fig-sec-diagnostic-plots-cusi-calc-total.png}}

}

\caption{\label{fig-sec-rainplot-cusi-calc-total}Diagnostic plots for
Coping Using Sex Inventory (CUSI, total score). The plots on the left
correspond to the model fitted with all data points, the plots on the
right correspond to the model after the removal of identified outliers.
Both models models showed similar results.}

\end{figure}%

\begin{table}

\caption{\label{tbl-sec-lm-cusi-calc-total}Linear regression model for
CUSI (total score) with n = 150 observations.}

\centering{

\fontsize{8.0pt}{9.0pt}\selectfont
\begin{tabular*}{\linewidth}{@{\extracolsep{\fill}}lccccc}
\toprule
\textbf{Variable} & \textbf{N} & \textbf{β} & \textbf{95\% CI} & \textbf{p-value} & \textbf{adjusted p value} \\ 
\midrule\addlinespace[2.5pt]
{\bfseries (Intercept)} & 150 & 3.5 & -0.24, 7.3 & 0.066 &  \\ 
{\bfseries Baseline value (pre)} & 150 & 0.82 & 0.71, 0.94 & <0.001 &  \\ 
{\bfseries Treatment} &  &  &  &  &  \\ 
    Placebo & 78 & 0.00 & Ref. &  &  \\ 
    Intervention & 72 & -0.09 & -2.0, 1.8 & >0.9 &  1.00 \\ 
{\bfseries Offense type} &  &  &  &  &  \\ 
    Hands-on (at least one conviction §176 ff StGB) & 58 & 0.00 & Ref. &  &  \\ 
    Hands-off (only §184b StGB) & 92 & 1.8 & -0.33, 4.0 & 0.10 &  1.00 \\ 
{\bfseries Type of supervision} &  &  &  &  &  \\ 
    community supervision & 120 & 0.00 & Ref. &  &  \\ 
    post-release supervision & 30 & 1.9 & -0.68, 4.4 & 0.15 &  1.00 \\ 
{\bfseries Additional treatment} &  &  &  &  &  \\ 
    No & 95 & 0.00 & Ref. &  &  \\ 
    Yes & 55 & 0.41 & -1.6, 2.4 & 0.7 &  1.00 \\ 
{\bfseries Static recidivism risk (baseline)} & 150 & 0.29 & -0.58, 1.2 & 0.5 &  1.00 \\ 
\bottomrule
\end{tabular*}
\begin{minipage}{\linewidth}
\vspace{.05em}
\parbox{\linewidth}{\raggedright {Abbreviation: CI = Confidence Interval\\
}}\end{minipage}

}

\end{table}%

\newpage{}

\subsection{SPSI-R}\label{spsi-r-1}

\begin{figure}

\centering{

\pandocbounded{\includegraphics[keepaspectratio]{_figures/fig-sec-rainplot-spsi-calc-total.png}}

}

\caption{\label{fig-sec-rainplot-spsi-total-score}Rainplot of the Social
Problem-Solving Inventory Revised (SPSI-R, total score).}

\end{figure}%

\begin{figure}

\centering{

\pandocbounded{\includegraphics[keepaspectratio]{_figures/fig-sec-diagnostic-plots-spsi-calc-total.png}}

}

\caption{\label{fig-sec-rainplot-spsi-calc-total}Diagnostic plots for
the Social Problem-Solving Inventory Revised (SPSI-R, total score).}

\end{figure}%

\begin{table}

\caption{\label{tbl-sec-lm-spsi-calc-total}Linear regression model for
Social Problem-Solving Inventory Revised SPSI-R (total score) with n =
134 observations.}

\centering{

\fontsize{8.0pt}{9.0pt}\selectfont
\begin{tabular*}{\linewidth}{@{\extracolsep{\fill}}lccccc}
\toprule
\textbf{Variable} & \textbf{N} & \textbf{β} & \textbf{95\% CI} & \textbf{p-value} & \textbf{adjusted p value} \\ 
\midrule\addlinespace[2.5pt]
{\bfseries (Intercept)} & 134 & 1.3 & -0.24, 2.8 & 0.10 &  \\ 
{\bfseries Baseline value (pre)} & 134 & 0.90 & 0.81, 1.0 & <0.001 &  \\ 
{\bfseries Treatment} &  &  &  &  &  \\ 
    Placebo & 72 & 0.00 & Ref. &  &  \\ 
    Intervention & 62 & 0.03 & -0.51, 0.56 & >0.9 &  1.00 \\ 
{\bfseries Offense type} &  &  &  &  &  \\ 
    Hands-on (at least one conviction §176 ff StGB) & 53 & 0.00 & Ref. &  &  \\ 
    Hands-off (only §184b StGB) & 81 & 0.08 & -0.52, 0.67 & 0.8 &  1.00 \\ 
{\bfseries Type of supervision} &  &  &  &  &  \\ 
    community supervision & 109 & 0.00 & Ref. &  &  \\ 
    post-release supervision & 25 & -0.44 & -1.2, 0.30 & 0.2 &  1.00 \\ 
{\bfseries Additional treatment} &  &  &  &  &  \\ 
    No & 88 & 0.00 & Ref. &  &  \\ 
    Yes & 46 & 0.23 & -0.34, 0.81 & 0.4 &  1.00 \\ 
{\bfseries Static recidivism risk (baseline)} & 134 & 0.01 & -0.23, 0.25 & >0.9 &  1.00 \\ 
\bottomrule
\end{tabular*}
\begin{minipage}{\linewidth}
\vspace{.05em}
\parbox{\linewidth}{\raggedright {Abbreviation: CI = Confidence Interval\\
}}\end{minipage}

}

\end{table}%

\newpage{}

\subsection{EKK-R}\label{ekk-r-1}

\begin{figure}

\centering{

\pandocbounded{\includegraphics[keepaspectratio]{_figures/fig-sec-rainplot-ekk-calc-total.png}}

}

\caption{\label{fig-sec-rainplot-ekk-total-score}Rainplot of the
Emotional Congruence with Children-Revised (EKK-R, total score).}

\end{figure}%

\begin{figure}

\centering{

\pandocbounded{\includegraphics[keepaspectratio]{_figures/fig-sec-diagnostic-plots-ekk-calc-total.png}}

}

\caption{\label{fig-sec-rainplot-ekk-calc-total}Diagnostic plots for
Emotional Congruence with Children-Revised (EKK-R, total score).}

\end{figure}%

\begin{table}

\caption{\label{tbl-sec-lm-ekk-calc-total}Linear regression model for
EKK-R (total score) with n = 95 observations.}

\centering{

\fontsize{8.0pt}{9.0pt}\selectfont
\begin{tabular*}{\linewidth}{@{\extracolsep{\fill}}lccccc}
\toprule
\textbf{Variable} & \textbf{N} & \textbf{β} & \textbf{95\% CI} & \textbf{p-value} & \textbf{adjusted p value} \\ 
\midrule\addlinespace[2.5pt]
{\bfseries (Intercept)} & 95 & 5.4 & 1.6, 9.3 & 0.006 &  \\ 
{\bfseries Baseline value (pre)} & 95 & 0.81 & 0.70, 0.92 & <0.001 &  \\ 
{\bfseries Treatment} &  &  &  &  &  \\ 
    Placebo & 53 & 0.00 & Ref. &  &  \\ 
    Intervention & 42 & -0.90 & -2.6, 0.81 & 0.3 &  1.00 \\ 
{\bfseries Offense type} &  &  &  &  &  \\ 
    Hands-on (at least one conviction §176 ff StGB) & 38 & 0.00 & Ref. &  &  \\ 
    Hands-off (only §184b StGB) & 57 & 0.27 & -1.7, 2.2 & 0.8 &  1.00 \\ 
{\bfseries Type of supervision} &  &  &  &  &  \\ 
    community supervision & 77 & 0.00 & Ref. &  &  \\ 
    post-release supervision & 18 & 0.40 & -2.0, 2.8 & 0.7 &  1.00 \\ 
{\bfseries Additional treatment} &  &  &  &  &  \\ 
    No & 59 & 0.00 & Ref. &  &  \\ 
    Yes & 36 & 0.57 & -1.3, 2.5 & 0.6 &  1.00 \\ 
{\bfseries Static recidivism risk (baseline)} & 95 & 0.04 & -0.78, 0.86 & >0.9 &  1.00 \\ 
\bottomrule
\end{tabular*}
\begin{minipage}{\linewidth}
\vspace{.05em}
\parbox{\linewidth}{\raggedright {Abbreviation: CI = Confidence Interval\\
}}\end{minipage}

}

\end{table}%

\newpage{}

\subsection{DERS}\label{ders-1}

\begin{figure}

\centering{

\pandocbounded{\includegraphics[keepaspectratio]{_figures/fig-sec-rainplot-ders-calc-imp.png}}

}

\caption{\label{fig-sec-rainplot-ders-imp-score}Rainplot of the
Difficulties in Emotion Regulation Scale (DERS, subscale impulsivity).
Note, that for the dependent variable, values greater than 9 were merged
into a single category.}

\end{figure}%

\begin{figure}

\centering{

\pandocbounded{\includegraphics[keepaspectratio]{_figures/fig-sec-diagnostic-plots-ders-calc-imp.png}}

}

\caption{\label{fig-sec-rainplot-ders-calc-imp}Diagnostic plots for
Difficulties in Emotion Regulation Scale (DERS, subscale impulsivity).
Note, that for the dependent variable, values greater than 9 were merged
into a single category..}

\end{figure}%

\begin{table}

\caption{\label{tbl-sec-lm-ders-calc-imp}Ordinal regression model for
the Difficulties in Emotion Regulation Scale DERS (subscore impulsivity)
with n = 150 observations. For the dependent variable, values greater
than 9 were merged into a single category.}

\centering{

\fontsize{8.0pt}{9.0pt}\selectfont
\begin{tabular*}{\linewidth}{@{\extracolsep{\fill}}lccccc}
\toprule
\textbf{Variable} & \textbf{N} & \textbf{OR} & \textbf{95\% CI} & \textbf{p-value} & \textbf{adjusted p value} \\ 
\midrule\addlinespace[2.5pt]
{\bfseries Baseline value (pre)} & 150 & 2·08 & 1·70, 2·59 & < 0.001 &  \\ 
{\bfseries treatment} &  &  &  &  &  \\ 
    Placebo & 78 & 1·00 & Ref. &  &  \\ 
    Intervention & 72 & 2·35 & 1·22, 4·61 & 0.012 & 0.176 \\ 
{\bfseries Offense type} &  &  &  &  &  \\ 
    Hands-on (at least one conviction §176 ff StGB) & 58 & 1·00 & Ref. &  &  \\ 
    Hands-off (only §184b StGB) & 92 & 1·02 & 0·47, 2·29 & 0.955 & 1.000 \\ 
{\bfseries Type of supervision} &  &  &  &  &  \\ 
    community supervision & 120 & 1·00 & Ref. &  &  \\ 
    post-release supervision & 30 & 1·87 & 0·74, 4·75 & 0.182 & 1.000 \\ 
{\bfseries Additional treatment} &  &  &  &  &  \\ 
    No & 95 & 1·00 & Ref. &  &  \\ 
    Yes & 55 & 0·69 & 0·33, 1·41 & 0.313 & 1.000 \\ 
{\bfseries Static recidivsm risk (baseline)} & 150 & 0·89 & 0·64, 1·21 & 0.459 & 1.000 \\ 
\bottomrule
\end{tabular*}
\begin{minipage}{\linewidth}
\vspace{.05em}
\parbox{\linewidth}{\raggedright {Abbreviations: CI = Confidence Interval, OR = Odds Ratio\\
}}\end{minipage}

}

\end{table}%

\newpage{}

\subsection{NARQ}\label{narq-1}

\begin{figure}

\centering{

\pandocbounded{\includegraphics[keepaspectratio]{_figures/fig-sec-rainplot-narq-calc-ris.png}}

}

\caption{\label{fig-sec-rainplot-narq-ris-score}Rainplot of the Negative
Affect Repair Questionnaire (NARQ, subscale externalizing strategies).
Note, that for the dependent variable, values greater than 4 were merged
into a single category.}

\end{figure}%

\begin{figure}

\centering{

\pandocbounded{\includegraphics[keepaspectratio]{_figures/fig-sec-diagnostic-plots-narq-calc-ris.png}}

}

\caption{\label{fig-sec-rainplot-narq-calc-ris}Diagnostic plots for the
Negative Affect Repair Questionnaire (NARQ, subscale externalizing
strategies). Note, that for the dependent variable, values greater than
4 were merged into a single category}

\end{figure}%

\begin{table}

\caption{\label{tbl-sec-lm-narq-calc-ris}Ordinal regression model for
the Negative Affect Repair Questionnaire NARQ (subscale externalizing
strategies) with n = 150 observations. Note, that for the dependent
variable, values greater than 4 were merged into a single category.}

\centering{

\fontsize{8.0pt}{9.0pt}\selectfont
\begin{tabular*}{\linewidth}{@{\extracolsep{\fill}}lccccc}
\toprule
\textbf{Variable} & \textbf{N} & \textbf{OR} & \textbf{95\% CI} & \textbf{p-value} & \textbf{adjusted p value} \\ 
\midrule\addlinespace[2.5pt]
{\bfseries Baseline value (pre)} & 150 & 1·69 & 1·44, 2·02 & < 0.001 &  \\ 
{\bfseries treatment} &  &  &  &  &  \\ 
    Placebo & 78 & 1·00 & Ref. &  &  \\ 
    Intervention & 72 & 1·46 & 0·81, 2·66 & 0.212 &  1.00 \\ 
{\bfseries Offense type} &  &  &  &  &  \\ 
    Hands-on (at least one conviction §176 ff StGB) & 58 & 1·00 & Ref. &  &  \\ 
    Hands-off (only §184b StGB) & 92 & 1·35 & 0·68, 2·71 & 0.395 &  1.00 \\ 
{\bfseries Type of supervision} &  &  &  &  &  \\ 
    community supervision & 120 & 1·00 & Ref. &  &  \\ 
    post-release supervision & 30 & 1·04 & 0·47, 2·32 & 0.922 &  1.00 \\ 
{\bfseries Additional treatment} &  &  &  &  &  \\ 
    No & 95 & 1·00 & Ref. &  &  \\ 
    Yes & 55 & 0·98 & 0·52, 1·84 & 0.952 &  1.00 \\ 
{\bfseries Static recidivsm risk (baseline)} & 150 & 0·85 & 0·63, 1·13 & 0.273 &  1.00 \\ 
\bottomrule
\end{tabular*}
\begin{minipage}{\linewidth}
\vspace{.05em}
\parbox{\linewidth}{\raggedright {Abbreviations: CI = Confidence Interval, OR = Odds Ratio\\
}}\end{minipage}

}

\end{table}%

\newpage{}

\subsection{ESIQ}\label{esiq-1}

\begin{figure}

\centering{

\pandocbounded{\includegraphics[keepaspectratio]{_figures/fig-sec-rainplot-esiq-calc-total.png}}

}

\caption{\label{fig-sec-rainplot-esiq-calc-total}Rainplot of the
Explicit Sexual Interest Questionnaire (ESIQ, total score). Note, that
the dependent variable has 3 ordered categories: worsening (increase),
no change, improvement (decrease).}

\end{figure}%

\begin{figure}

\centering{

\pandocbounded{\includegraphics[keepaspectratio]{_figures/fig-sec-diagnostic-plots-esiq-calc-total.png}}

}

\caption{\label{fig-sec-rainplot-esiq-calc-total}Diagnostic plots for
the Explicit Sexual Interest Questionnaire (ESIQ, total score). Note,
that the dependent variable has 3 ordered categories: worsening
(increase), no change, improvement (decrease)}

\end{figure}%

\begin{table}

\caption{\label{tbl-sec-lm-esiq-calc-total}Ordinal regression model for
the Explicit Sexual Interest Questionnaire ESIQ (total score) with n =
95 observations. Note, that the dependent variable has 3 ordered
categories: worsening (increase), no change, improvement (decrease).}

\centering{

\fontsize{8.0pt}{9.0pt}\selectfont
\begin{tabular*}{\linewidth}{@{\extracolsep{\fill}}lccccc}
\toprule
\textbf{Variable} & \textbf{N} & \textbf{OR} & \textbf{95\% CI} & \textbf{p-value} & \textbf{adjusted p value} \\ 
\midrule\addlinespace[2.5pt]
{\bfseries treatment} &  &  &  &  &  \\ 
    Placebo & 53 & 1·00 & Ref. &  &  \\ 
    Intervention & 42 & 4·36 & 1·93, 10·3 & < 0.001 & 0.009 \\ 
{\bfseries Offense type} &  &  &  &  &  \\ 
    Hands-on (at least one conviction §176 ff StGB) & 38 & 1·00 & Ref. &  &  \\ 
    Hands-off (only §184b StGB) & 57 & 0·87 & 0·34, 2·19 & 0.764 & 1.000 \\ 
{\bfseries Type of supervision} &  &  &  &  &  \\ 
    community supervision & 77 & 1·00 & Ref. &  &  \\ 
    post-release supervision & 18 & 0·88 & 0·29, 2·75 & 0.822 & 1.000 \\ 
{\bfseries Additional treatment} &  &  &  &  &  \\ 
    No & 59 & 1·00 & Ref. &  &  \\ 
    Yes & 36 & 1·19 & 0·49, 2·95 & 0.701 & 1.000 \\ 
{\bfseries Static recidivsm risk (baseline)} & 95 & 0·98 & 0·66, 1·45 & 0.904 & 1.000 \\ 
\bottomrule
\end{tabular*}
\begin{minipage}{\linewidth}
\vspace{.05em}
\parbox{\linewidth}{\raggedright {Abbreviations: CI = Confidence Interval, OR = Odds Ratio\\
}}\end{minipage}

}

\end{table}%

\newpage{}

\subsection{BMS}\label{bms-1}

\begin{figure}

\centering{

\pandocbounded{\includegraphics[keepaspectratio]{_figures/fig-sec-rainplot-kvm-score.png}}

}

\caption{\label{fig-sec-rainplot-kvm-score}Rainplot of the Bumby Molest
Scale (BMS, total score). Note, that for the dependent variable, values
were dichotomized based on the cut-off greater or equal to 43.}

\end{figure}%

\begin{figure}

\centering{

\pandocbounded{\includegraphics[keepaspectratio]{_figures/fig-sec-diagnostic-plots-kvm-score.png}}

}

\caption{\label{fig-sec-rainplot-kvm-score}Diagnostic plots for the
Bumby Molest Scale (BMS, total score). Note, that for the dependent
variable, values were dichotomized based on the cut-off greater or equal
to 43.}

\end{figure}%

\begin{table}

\caption{\label{tbl-sec-lm-kvm-score}Logistic regression model for the
Bumby Molest Scale BMS (total score) with n = 112 observations. Note,
that for the dependent variable, values were dichotomized based on the
cut-off greater or equal to 43.}

\centering{

\fontsize{8.0pt}{9.0pt}\selectfont
\begin{tabular*}{\linewidth}{@{\extracolsep{\fill}}lcccccc}
\toprule
\textbf{Variable} & \textbf{N} & \textbf{Event N} & \textbf{OR} & \textbf{95\% CI} & \textbf{p-value} & \textbf{adjusted p value} \\ 
\midrule\addlinespace[2.5pt]
{\bfseries Baseline value (pre)} &  &  &  &  &  &  \\ 
    FALSE & 48 & 6 & 1·00 & Ref. &  &  \\ 
    TRUE & 64 & 51 & 31·2 & 10·7, 111 & < 0.001 &  \\ 
{\bfseries Treatment} &  &  &  &  &  &  \\ 
    Placebo & 59 & 24 & 1·00 & Ref. &  &  \\ 
    Intervention & 53 & 33 & 3·51 & 1·22, 11·3 & 0.025 & 0.348 \\ 
{\bfseries Offense type} &  &  &  &  &  &  \\ 
    Hands-on (at least one conviction §176 ff StGB) & 43 & 27 & 1·00 & Ref. &  &  \\ 
    Hands-off (only §184b StGB) & 69 & 30 & 0·47 & 0·13, 1·60 & 0.229 & 1.000 \\ 
{\bfseries Type of supervision} &  &  &  &  &  &  \\ 
    community supervision & 91 & 45 & 1·00 & Ref. &  &  \\ 
    post-release supervision & 21 & 12 & 1·27 & 0·28, 6·28 & 0.764 & 1.000 \\ 
{\bfseries Additional treatment} &  &  &  &  &  &  \\ 
    No & 70 & 32 & 1·00 & Ref. &  &  \\ 
    Yes & 42 & 25 & 1·60 & 0·52, 5·08 & 0.415 & 1.000 \\ 
{\bfseries Static recidivsm risk (baseline)} & 112 & 57 & 1·05 & 0·61, 1·77 & 0.866 & 1.000 \\ 
\bottomrule
\end{tabular*}
\begin{minipage}{\linewidth}
\vspace{.05em}
\parbox{\linewidth}{\raggedright {Abbreviations: CI = Confidence Interval, OR = Odds Ratio\\
}}\end{minipage}

}

\end{table}%

\newpage{}

\subsection{HBI-19}\label{hbi-19-1}

\begin{figure}

\centering{

\pandocbounded{\includegraphics[keepaspectratio]{_figures/fig-sec-rainplot-hbi-calc-total.png}}

}

\caption{\label{fig-sec-rainplot-hbi-total-score}Rainplot of the
Hypersexual Behavior Inventory-19 (HBI-19, total score). Note, that for
the dependent variable, values were dichotomized based on the cut-off
greater or equal to 24.}

\end{figure}%

\begin{figure}

\centering{

\pandocbounded{\includegraphics[keepaspectratio]{_figures/fig-sec-diagnostic-plots-hbi-calc-total.png}}

}

\caption{\label{fig-sec-rainplot-hbi-calc-total}Diagnostic plots for the
Hypersexual Behavior Inventory-19 (HBI-19, total score). Note, that for
the dependent variable, values were dichotomized based on the cut-off
greater or equal to 24.}

\end{figure}%

\begin{table}

\caption{\label{tbl-sec-lm-hbi-calc-total}Logistic regression model for
the Hypersexual Behavior Inventory-19 HBI-19 (total score) with n = 95
observations. Note, that for the dependent variable, values were
dichotomized based on the cut-off greater or equal to 24.}

\centering{

\fontsize{8.0pt}{9.0pt}\selectfont
\begin{tabular*}{\linewidth}{@{\extracolsep{\fill}}lcccccc}
\toprule
\textbf{Variable} & \textbf{N} & \textbf{Event N} & \textbf{OR} & \textbf{95\% CI} & \textbf{p-value} & \textbf{adjusted p value} \\ 
\midrule\addlinespace[2.5pt]
{\bfseries Baseline value (pre)} &  &  &  &  &  &  \\ 
    FALSE & 50 & 13 & 1·00 & Ref. &  &  \\ 
    TRUE & 45 & 38 & 16·9 & 5·95, 55·5 & < 0.001 &  \\ 
{\bfseries Treatment} &  &  &  &  &  &  \\ 
    Placebo & 53 & 24 & 1·00 & Ref. &  &  \\ 
    Intervention & 42 & 27 & 2·84 & 0·98, 8·84 & 0.060 & 0.782 \\ 
{\bfseries Offense type} &  &  &  &  &  &  \\ 
    Hands-on (at least one conviction §176 ff StGB) & 38 & 18 & 1·00 & Ref. &  &  \\ 
    Hands-off (only §184b StGB) & 57 & 33 & 2·48 & 0·74, 8·92 & 0.148 & 1.000 \\ 
{\bfseries Type of supervision} &  &  &  &  &  &  \\ 
    community supervision & 77 & 42 & 1·00 & Ref. &  &  \\ 
    post-release supervision & 18 & 9 & 0·65 & 0·14, 2·97 & 0.572 & 1.000 \\ 
{\bfseries Additional treatment} &  &  &  &  &  &  \\ 
    No & 59 & 29 & 1·00 & Ref. &  &  \\ 
    Yes & 36 & 22 & 2·63 & 0·85, 8·73 & 0.102 & 1.000 \\ 
{\bfseries Static recidivsm risk (baseline)} & 95 & 51 & 0·72 & 0·43, 1·17 & 0.190 & 1.000 \\ 
\bottomrule
\end{tabular*}
\begin{minipage}{\linewidth}
\vspace{.05em}
\parbox{\linewidth}{\raggedright {Abbreviations: CI = Confidence Interval, OR = Odds Ratio\\
}}\end{minipage}

}

\end{table}%

\newpage{}

\subsection{SSIC}\label{ssic-1}

\begin{figure}

\centering{

\pandocbounded{\includegraphics[keepaspectratio]{_figures/fig-sec-rainplot-ssik-calc-total.png}}

}

\caption{\label{fig-sec-rainplot-ssik-calc-total}Rainplot of the
Specific self-efficacy for modifying Sexual Interest in Children (SSIC,
total score). Note, that for the dependent variable, values were
dichotomized based on the cut-off greater or equal to 30.}

\end{figure}%

\begin{figure}

\centering{

\pandocbounded{\includegraphics[keepaspectratio]{_figures/fig-sec-diagnostic-plots-ssik-calc-total.png}}

}

\caption{\label{fig-sec-rainplot-ssik-calc-total}Diagnostic plots for
the Specific self-efficacy for modifying Sexual Interest in Children
(SSIC, total score). Note, that for the dependent variable, values were
dichotomized based on the cut-off greater or equal to 30.}

\end{figure}%

\begin{table}

\caption{\label{tbl-sec-lm-ssik-calc-total}Logistic regression model for
the Specific self-efficacy for modifying Sexual Interest in Children
SSIC (total score) with n = 95 observations. Note, that for the
dependent variable, values were dichotomized based on the cut-off
greater or equal to 30.}

\centering{

\fontsize{8.0pt}{9.0pt}\selectfont
\begin{tabular*}{\linewidth}{@{\extracolsep{\fill}}lcccccc}
\toprule
\textbf{Variable} & \textbf{N} & \textbf{Event N} & \textbf{OR} & \textbf{95\% CI} & \textbf{p-value} & \textbf{adjusted p value} \\ 
\midrule\addlinespace[2.5pt]
{\bfseries Baseline value (pre)} &  &  &  &  &  &  \\ 
    FALSE & 51 & 18 & 1·00 & Ref. &  &  \\ 
    TRUE & 44 & 36 & 10·4 & 3·77, 33·0 & < 0.001 &  \\ 
{\bfseries Treatment} &  &  &  &  &  &  \\ 
    Placebo & 53 & 31 & 1·00 & Ref. &  &  \\ 
    Intervention & 42 & 23 & 1·51 & 0·57, 4·24 & 0.411 &  1.00 \\ 
{\bfseries Offense type} &  &  &  &  &  &  \\ 
    Hands-on (at least one conviction §176 ff StGB) & 38 & 23 & 1·00 & Ref. &  &  \\ 
    Hands-off (only §184b StGB) & 57 & 31 & 0·51 & 0·16, 1·54 & 0.239 &  1.00 \\ 
{\bfseries Type of supervision} &  &  &  &  &  &  \\ 
    community supervision & 77 & 42 & 1·00 & Ref. &  &  \\ 
    post-release supervision & 18 & 12 & 0·86 & 0·21, 3·46 & 0.834 &  1.00 \\ 
{\bfseries Additional treatment} &  &  &  &  &  &  \\ 
    No & 59 & 31 & 1·00 & Ref. &  &  \\ 
    Yes & 36 & 23 & 1·15 & 0·40, 3·25 & 0.797 &  1.00 \\ 
{\bfseries Static recidivsm risk (baseline)} & 95 & 54 & 1·04 & 0·66, 1·64 & 0.868 &  1.00 \\ 
\bottomrule
\end{tabular*}
\begin{minipage}{\linewidth}
\vspace{.05em}
\parbox{\linewidth}{\raggedright {Abbreviations: CI = Confidence Interval, OR = Odds Ratio\\
}}\end{minipage}

}

\end{table}%

\newpage{}

\section{Adverse events}\label{adverse-events}

\subsection{Mood and Risk Questionnaire
(MRQ)}\label{mood-and-risk-questionnaire-mrq}

\begin{landscape}

\begin{table}

\caption{\label{tbl-mrq-frequency-highest-value-post}Frequencies of
adverse events during working on the WBI. The MRQ Post questionnaire
explicitly asks for the time during working on one session. Shown are
the frequency of participants reported at least once the highest
possible score (= adverse event) on one of the measurement time-points
(after each session of the WBI).}

\centering{

\fontsize{8.0pt}{9.0pt}\selectfont
\begin{tabular*}{\linewidth}{@{\extracolsep{\fill}}lccccccc}
\toprule
\textbf{MRQ subscale} & \textbf{Overall}  N = 221\textsuperscript{\textit{1}} & \textbf{Intervention}  N = 108\textsuperscript{\textit{1}} & \textbf{Placebo}  N = 113\textsuperscript{\textit{1}} & \textbf{Cramer's V} & \textbf{p} & add\_stat\_1 & \textbf{q-value} \\ 
\midrule\addlinespace[2.5pt]
{\bfseries Contact planning} & 2 (0·9\%) & 1 (0·9\%) & 1 (0·9\%) & 0·002 & >0·99 &  & >0·99 \\ 
{\bfseries Contact preparation} & 0 (0\%) & 0 (0\%) & 0 (0\%) &  &  &  &  \\ 
{\bfseries Urge for CSA} & 0 (0\%) & 0 (0\%) & 0 (0\%) &  &  &  &  \\ 
{\bfseries Urge for CSAM} & 0 (0\%) & 0 (0\%) & 0 (0\%) &  &  &  &  \\ 
{\bfseries Sexual tension} & 2 (0·9\%) & 2 (1·9\%) & 0 (0\%) & 0·098 & 0·24 &  & >0·99 \\ 
{\bfseries Control of sexual thoughts and activity} & 1 (0·5\%) & 0 (0\%) & 1 (0·9\%) & 0·066 & >0·99 &  & >0·99 \\ 
{\bfseries Unable to cope with the mental burden} & 9 (4·1\%) & 7 (6·5\%) & 2 (1·8\%) & 0·119 & 0·10 &  & 0·57 \\ 
{\bfseries Suicidal ideations} & 2 (0·9\%) & 2 (1·9\%) & 0 (0\%) & 0·098 & 0·24 &  & >0·99 \\ 
{\bfseries Mental crisis} & 15 (6·8\%) & 11 (10\%) & 4 (3·5\%) & 0·132 & 0·062 &  & 0·44 \\ 
{\bfseries Very bad current mood} & 12 (5·4\%) & 8 (7·4\%) & 4 (3·5\%) & 0·085 & 0·24 &  & >0·99 \\ 
\bottomrule
\end{tabular*}
\begin{minipage}{\linewidth}
\vspace{.05em}
\parbox{\linewidth}{\raggedright {\textsuperscript{\textit{1}} n (\%)}}\\
\parbox{\linewidth}{\raggedright {Abbreviation: p\textsubscript{adj} = Holm-Bonferroni adjusted p.\\
}}\end{minipage}

}

\end{table}%

\end{landscape}

\begin{landscape}

\begin{table}

\caption{\label{tbl-mrq-frequency-highest-value-pre}Frequencies of
adverse events between working on sessions of the WBI. The MRQ pre
questionnaire explicitly asks asks for the last week before beginning a
new session. Shown are the frequency of participants reported at least
once the highest possible score (= adverse event) on one of the
measurement time-points (after each session of the WBI).}

\centering{

\fontsize{8.0pt}{9.0pt}\selectfont
\begin{tabular*}{\linewidth}{@{\extracolsep{\fill}}lccccccc}
\toprule
\textbf{MRQ subscale} & \textbf{Overall}  N = 221\textsuperscript{\textit{1}} & \textbf{Intervention}  N = 108\textsuperscript{\textit{1}} & \textbf{Placebo}  N = 113\textsuperscript{\textit{1}} & \textbf{Cramer's V} & \textbf{p} & add\_stat\_1 & \textbf{q-value} \\ 
\midrule\addlinespace[2.5pt]
{\bfseries Contact planning} & 3 (1·4\%) & 3 (2·8\%) & 0 (0\%) & 0·120 & 0·12 &  & >0·99 \\ 
{\bfseries Contact preparation} & 1 (0·5\%) & 1 (0·9\%) & 0 (0\%) & 0·069 & 0·49 &  & >0·99 \\ 
{\bfseries Urge for CSA} & 0 (0\%) & 0 (0\%) & 0 (0\%) &  &  &  &  \\ 
{\bfseries Urge for CSAM} & 2 (0·9\%) & 1 (0·9\%) & 1 (0·9\%) & 0·002 & >0·99 &  & >0·99 \\ 
{\bfseries Sexual tension} & 8 (3·6\%) & 6 (5·6\%) & 2 (1·8\%) & 0·101 & 0·16 &  & >0·99 \\ 
{\bfseries Control of sexual thoughts and activity} & 2 (0·9\%) & 2 (1·9\%) & 0 (0\%) & 0·098 & 0·24 &  & >0·99 \\ 
{\bfseries Unable to cope with the mental burden} & 20 (9·0\%) & 11 (10\%) & 9 (8·0\%) & 0·039 & 0·64 &  & >0·99 \\ 
{\bfseries Suicidal ideations} & 4 (1·8\%) & 1 (0·9\%) & 3 (2·7\%) & 0·065 & 0·62 &  & >0·99 \\ 
{\bfseries Mental crisis} & 16 (7·2\%) & 11 (10\%) & 5 (4·4\%) & 0·111 & 0·12 &  & >0·99 \\ 
{\bfseries Very bad current mood} & 8 (3·6\%) & 5 (4·6\%) & 3 (2·7\%) & 0·053 & 0·49 &  & >0·99 \\ 
\bottomrule
\end{tabular*}
\begin{minipage}{\linewidth}
\vspace{.05em}
\parbox{\linewidth}{\raggedright {\textsuperscript{\textit{1}} n (\%)}}\\
\parbox{\linewidth}{\raggedright {Abbreviation: p\textsubscript{adj} = Holm-Bonferroni adjusted p.\\
}}\end{minipage}

}

\end{table}%

\end{landscape}

\newpage{}

\begin{landscape}

\subsection{Exclusion reasons}\label{exclusion-reasons}

\begin{table}

\caption{\label{tbl-exclusion-questionnaire}Frequencies of exclusion
reasons reported by the supervision officers. Exclusion reasons have
been assessed in a structured manner by the Exclusion Questionnaire
(EQ).}

\centering{

\fontsize{8.0pt}{9.0pt}\selectfont
\begin{tabular*}{\linewidth}{@{\extracolsep{\fill}}lcccccc}
\toprule
\textbf{Exclusion reason} & \textbf{Overall}  N = 124\textsuperscript{\textit{1}} & \textbf{Intervention}  N = 66\textsuperscript{\textit{1}} & \textbf{Placebo}  N = 58\textsuperscript{\textit{1}} & \textbf{Cramer's V} & \textbf{p} & \textbf{q-value} \\ 
\midrule\addlinespace[2.5pt]
{\bfseries Other reasons} & 17 (14\%) & 10 (15\%) & 7 (12\%) & 0·040 & 0·79 & >0·99 \\ 
{\bfseries Withdrawal of informed consent} & 7 (5·6\%) & 4 (6·1\%) & 3 (5·2\%) & 0·020 & >0·99 & >0·99 \\ 
{\bfseries Probationary supervision expired} & 22 (18\%) & 8 (12\%) & 14 (24\%) & 0·160 & 0·10 & 0·91 \\ 
{\bfseries CSA offense, Other offense (not CSA or CSEM)} & 1 (0·8\%) & 0 (0\%) & 1 (1·7\%) & 0·100 & 0·47 & >0·99 \\ 
{\bfseries CSA and CSEM offense} & 2 (1·6\%) & 1 (1·5\%) & 1 (1·7\%) & 0·010 & >0·99 & >0·99 \\ 
{\bfseries Other offense (not CSA or CSEM)} & 3 (2·4\%) & 3 (4·5\%) & 0 (0\%) & 0·150 & 0·25 & >0·99 \\ 
{\bfseries CSA offense} & 1 (0·8\%) & 0 (0\%) & 1 (1·7\%) & 0·100 & 0·47 & >0·99 \\ 
{\bfseries CSEM offense} & 1 (0·8\%) & 1 (1·5\%) & 0 (0\%) & 0·080 & >0·99 & >0·99 \\ 
{\bfseries End of RCT} & 70 (56\%) & 39 (59\%) & 31 (53\%) & 0·060 & 0·59 & >0·99 \\ 
\bottomrule
\end{tabular*}
\begin{minipage}{\linewidth}
\vspace{.05em}
\parbox{\linewidth}{\raggedright {\textsuperscript{\textit{1}} n (\%)}}\\
\parbox{\linewidth}{\raggedright {Abbreviation: p\textsubscript{adj} = Holm-Bonferroni adjusted p.\\
}}\end{minipage}

}

\end{table}%

\end{landscape}

\newpage{}

\section{Ancillary Outcomes}\label{ancillary-outcomes}

\subsection{WHO-5}\label{who-5-1}

\begin{landscape}

\begin{figure}

\centering{

\pandocbounded{\includegraphics[keepaspectratio]{_figures/fig-anc-who-5-rainplot.png}}

}

\caption{\label{fig-anc-who-5-rainplot}Rainplot for the WHO-5 Well-Being
Index (WHO-5, total score).}

\end{figure}%

\end{landscape}

\begin{table}

\caption{\label{tbl-anc-who-mmrm}Mixed model for repeated measures
(MMRM) for the WHO-5 Well-Being Index (WHO-5, total score).}

\centering{

\fontsize{8.0pt}{9.0pt}\selectfont
\begin{tabular*}{\linewidth}{@{\extracolsep{\fill}}lccc}
\toprule
\textbf{Variable} & \textbf{Beta} & \textbf{95\% CI} & \textbf{p-value} \\ 
\midrule\addlinespace[2.5pt]
{\bfseries (Intercept)} & 22 & 14 to 29 & <0·0001 \\ 
{\bfseries Type of supervision} &  &  &  \\ 
    community supervision & 0·00 & Ref. &  \\ 
    post-release supervision & -2·9 & -8·2 to 2·4 & 0·29 \\ 
{\bfseries Additional treatment} &  &  &  \\ 
    No & 0·00 & Ref. &  \\ 
    Yes & 0·25 & -3·9 to 4·4 & 0·91 \\ 
{\bfseries Offense type} &  &  &  \\ 
    Hands-off (only §184b StGB) & 0·00 & Ref. &  \\ 
    Hands-on (at least one conviction §176 ff StGB) & 1·9 & -2·5 to 6·2 & 0·40 \\ 
{\bfseries Static recidivism risk (baseline)} & -0·94 & -2·7 to 0·84 & 0·30 \\ 
{\bfseries Timepoint * treatment} &  &  &  \\ 
    Module 1 (post) * Intervention & 1·7 & -3·1 to 6·5 & 0·49 \\ 
    Module 2 (before) * Intervention & -1·8 & -6·4 to 2·7 & 0·43 \\ 
    Module 2 (post) * Intervention & -1·6 & -6·6 to 3·3 & 0·52 \\ 
    Module 3 (before) * Intervention & -0·86 & -6·3 to 4·6 & 0·76 \\ 
    Module 3 (post) * Intervention & -4·5 & -10 to 1·1 & 0·11 \\ 
    Module 4 (before) * Intervention & 0·10 & -5·9 to 6·1 & 0·97 \\ 
    Module 4 (post) * Intervention & -4·3 & -10 to 1·6 & 0·15 \\ 
    Module 5 (before) * Intervention & -2·4 & -8·8 to 4·1 & 0·47 \\ 
    Module 5 (post) * Intervention & -2·9 & -10 to 4·4 & 0·43 \\ 
    Module 6 (before) * Intervention & -1·9 & -9·2 to 5·3 & 0·60 \\ 
    Module 6 (post) * Intervention & -0·15 & -7·6 to 7·3 & 0·97 \\ 
{\bfseries Timepoint} &  &  &  \\ 
    Module 1 (post) & 0·00 & Ref. &  \\ 
    Module 2 (before) & -0·22 & -3·1 to 2·6 & 0·88 \\ 
    Module 2 (post) & 0·72 & -2·6 to 4·0 & 0·67 \\ 
    Module 3 (before) & -0·02 & -3·9 to 3·9 & >0·99 \\ 
    Module 3 (post) & 1·8 & -2·0 to 5·5 & 0·35 \\ 
    Module 4 (before) & -2·4 & -6·4 to 1·6 & 0·24 \\ 
    Module 4 (post) & 1·1 & -2·8 to 5·0 & 0·57 \\ 
    Module 5 (before) & 0·90 & -3·4 to 5·2 & 0·68 \\ 
    Module 5 (post) & -0·31 & -5·2 to 4·6 & 0·90 \\ 
    Module 6 (before) & 1·4 & -3·7 to 6·5 & 0·58 \\ 
    Module 6 (post) & 3·1 & -2·2 to 8·5 & 0·24 \\ 
{\bfseries Baseline value (WHO-5)} & 0·68 & 0·60 to 0·77 & <0·0001 \\ 
\bottomrule
\end{tabular*}
\begin{minipage}{\linewidth}
\vspace{.05em}
\parbox{\linewidth}{\raggedright {Abbreviations: CI = Confidence Interval, CI = Confidence interval, StGB = German penalty law.\\
}}\end{minipage}

}

\end{table}%

\begin{landscape}

\begin{figure}

\centering{

\pandocbounded{\includegraphics[keepaspectratio]{_figures/fig-anc-who-estimated-marginal-means.png}}

}

\caption{\label{fig-anc-who-estimated-marginal-means}Estimated marginal
means of the MMRM for the WHO-5 Well-Being Index (WHO-5 total score).}

\end{figure}%

\end{landscape}

\begin{table}

\caption{\label{tbl-anc-who-pre-post-pairwise-intervention}Pairwise
comparison of the WHO-5 Well-Being Index (WHO-5, total score, pre-post
module) for the intervention arm.}

\centering{

\fontsize{8.0pt}{9.0pt}\selectfont
\begin{tabular*}{\linewidth}{@{\extracolsep{\fill}}lcccc}
\toprule
\textbf{Module} & \textbf{pre}  N = 106 & \textbf{post}  N = 106 & \textbf{p} & \textbf{q-value} \\ 
\midrule\addlinespace[2.5pt]
{\bfseries Module 1} &  &  & 0·054 & 0·27 \\ 
    Mean (SD) & 58 (23) & 62 (24) &  &  \\ 
    Median (Q1, Q3) & 64 (40, 76) & 68 (40, 80) &  &  \\ 
    Min, Max & 8, 100 & 12, 100 &  &  \\ 
{\bfseries Module 2} &  &  & 0·52 & >0·99 \\ 
    Mean (SD) & 63 (23) & 62 (23) &  &  \\ 
    Median (Q1, Q3) & 70 (44, 80) & 64 (44, 80) &  &  \\ 
    Min, Max & 12, 100 & 0, 100 &  &  \\ 
    Missing & 22 & 22 &  &  \\ 
{\bfseries Module 3} &  &  & 0·041 & 0·25 \\ 
    Mean (SD) & 63 (24) & 60 (23) &  &  \\ 
    Median (Q1, Q3) & 68 (44, 80) & 62 (42, 80) &  &  \\ 
    Min, Max & 0, 100 & 0, 100 &  &  \\ 
    Missing & 38 & 38 &  &  \\ 
{\bfseries Module 4} &  &  & 0·77 & >0·99 \\ 
    Mean (SD) & 60 (24) & 58 (27) &  &  \\ 
    Median (Q1, Q3) & 62 (38, 80) & 64 (36, 80) &  &  \\ 
    Min, Max & 0, 100 & 0, 100 &  &  \\ 
    Missing & 46 & 46 &  &  \\ 
{\bfseries Module 5} &  &  & 0·48 & >0·99 \\ 
    Mean (SD) & 59 (27) & 60 (28) &  &  \\ 
    Median (Q1, Q3) & 62 (36, 80) & 66 (32, 80) &  &  \\ 
    Min, Max & 0, 100 & 0, 100 &  &  \\ 
    Missing & 56 & 56 &  &  \\ 
{\bfseries Module 6} &  &  & 0·13 & 0·50 \\ 
    Mean (SD) & 61 (29) & 66 (26) &  &  \\ 
    Median (Q1, Q3) & 70 (32, 80) & 72 (52, 84) &  &  \\ 
    Min, Max & 0, 100 & 12, 100 &  &  \\ 
    Missing & 64 & 64 &  &  \\ 
\bottomrule
\end{tabular*}
\begin{minipage}{\linewidth}
\vspace{.05em}
\parbox{\linewidth}{\raggedright {Abbreviation: Q1 = 25th percentile, Q3 = 75th percentile, p\textsubscript{adj} = Holm-Bonferroni adjusted p.\\
}}\end{minipage}

}

\end{table}%

\begin{table}

\caption{\label{tbl-anc-who-pre-post-pairwise-placebo}Pairwise
comparison of the WHO-5 Well-Being Index (WHO-5, total score, pre-post
module) for the placebo arm.}

\centering{

\fontsize{8.0pt}{9.0pt}\selectfont
\begin{tabular*}{\linewidth}{@{\extracolsep{\fill}}lcccc}
\toprule
\textbf{Module} & \textbf{pre}  N = 106 & \textbf{post}  N = 106 & \textbf{p} & \textbf{q-value} \\ 
\midrule\addlinespace[2.5pt]
{\bfseries Module 1} &  &  & 0·054 & 0·27 \\ 
    Mean (SD) & 58 (23) & 62 (24) &  &  \\ 
    Median (Q1, Q3) & 64 (40, 76) & 68 (40, 80) &  &  \\ 
    Min, Max & 8, 100 & 12, 100 &  &  \\ 
{\bfseries Module 2} &  &  & 0·52 & >0·99 \\ 
    Mean (SD) & 63 (23) & 62 (23) &  &  \\ 
    Median (Q1, Q3) & 70 (44, 80) & 64 (44, 80) &  &  \\ 
    Min, Max & 12, 100 & 0, 100 &  &  \\ 
    Missing & 22 & 22 &  &  \\ 
{\bfseries Module 3} &  &  & 0·041 & 0·25 \\ 
    Mean (SD) & 63 (24) & 60 (23) &  &  \\ 
    Median (Q1, Q3) & 68 (44, 80) & 62 (42, 80) &  &  \\ 
    Min, Max & 0, 100 & 0, 100 &  &  \\ 
    Missing & 38 & 38 &  &  \\ 
{\bfseries Module 4} &  &  & 0·77 & >0·99 \\ 
    Mean (SD) & 60 (24) & 58 (27) &  &  \\ 
    Median (Q1, Q3) & 62 (38, 80) & 64 (36, 80) &  &  \\ 
    Min, Max & 0, 100 & 0, 100 &  &  \\ 
    Missing & 46 & 46 &  &  \\ 
{\bfseries Module 5} &  &  & 0·48 & >0·99 \\ 
    Mean (SD) & 59 (27) & 60 (28) &  &  \\ 
    Median (Q1, Q3) & 62 (36, 80) & 66 (32, 80) &  &  \\ 
    Min, Max & 0, 100 & 0, 100 &  &  \\ 
    Missing & 56 & 56 &  &  \\ 
{\bfseries Module 6} &  &  & 0·13 & 0·50 \\ 
    Mean (SD) & 61 (29) & 66 (26) &  &  \\ 
    Median (Q1, Q3) & 70 (32, 80) & 72 (52, 84) &  &  \\ 
    Min, Max & 0, 100 & 12, 100 &  &  \\ 
    Missing & 64 & 64 &  &  \\ 
\bottomrule
\end{tabular*}
\begin{minipage}{\linewidth}
\vspace{.05em}
\parbox{\linewidth}{\raggedright {Abbreviation: Q1 = 25th percentile, Q3 = 75th percentile, p\textsubscript{adj} = Holm-Bonferroni adjusted p.\\
}}\end{minipage}

}

\end{table}%

\begin{table}

\caption{\label{tbl-anc-who-difference-from-baseline}Pairwise
comparisons of the WHO-5 Well-Being Index (WHO-5, total score,
difference score).}

\centering{

\fontsize{8.0pt}{9.0pt}\selectfont
\begin{tabular*}{\linewidth}{@{\extracolsep{\fill}}lcccc}
\toprule
\textbf{Characteristic} & \textbf{Placebo}  N = 113 & \textbf{Intervention}  N = 108 & \textbf{p} & \textbf{q-value} \\ 
\midrule\addlinespace[2.5pt]
{\bfseries Module 1} &  &  & 0·31 & >0·99 \\ 
    Mean (SD) & 1 (18) & 4 (20) &  &  \\ 
    Median (Q1, Q3) & 0 (-8, 12) & 4 (-4, 12) &  &  \\ 
    Min, Max & -68, 64 & -60, 76 &  &  \\ 
{\bfseries Module 2} &  &  & 0·86 & >0·99 \\ 
    Mean (SD) & 1 (12) & 0 (16) &  &  \\ 
    Median (Q1, Q3) & 0 (-4, 8) & 0 (-8, 8) &  &  \\ 
    Min, Max & -40, 40 & -48, 36 &  &  \\ 
    Missing & 19 & 17 &  &  \\ 
{\bfseries Module 3} &  &  & 0·26 & >0·99 \\ 
    Mean (SD) & 1 (19) & -2 (19) &  &  \\ 
    Median (Q1, Q3) & 0 (-8, 12) & 0 (-12, 6) &  &  \\ 
    Min, Max & -48, 72 & -100, 48 &  &  \\ 
    Missing & 35 & 36 &  &  \\ 
{\bfseries Module 4} &  &  & 0·15 & 0·89 \\ 
    Mean (SD) & 4 (18) & -1 (17) &  &  \\ 
    Median (Q1, Q3) & 0 (-2, 8) & 0 (-12, 8) &  &  \\ 
    Min, Max & -44, 64 & -36, 48 &  &  \\ 
    Missing & 41 & 46 &  &  \\ 
{\bfseries Module 5} &  &  & 0·95 & >0·99 \\ 
    Mean (SD) & -1 (18) & -2 (15) &  &  \\ 
    Median (Q1, Q3) & 0 (-4, 4) & 0 (-8, 8) &  &  \\ 
    Min, Max & -100, 40 & -60, 28 &  &  \\ 
    Missing & 54 & 55 &  &  \\ 
{\bfseries Module 6} &  &  & 0·63 & >0·99 \\ 
    Mean (SD) & 2 (12) & 2 (13) &  &  \\ 
    Median (Q1, Q3) & 0 (-4, 8) & 0 (0, 8) &  &  \\ 
    Min, Max & -24, 32 & -40, 28 &  &  \\ 
    Missing & 60 & 66 &  &  \\ 
\bottomrule
\end{tabular*}
\begin{minipage}{\linewidth}
\vspace{.05em}
\parbox{\linewidth}{\raggedright {Abbreviation: Q1 = 25th percentile, Q3 = 75th percentile, p\textsubscript{adj} = Holm-Bonferroni adjusted p.\\
}}\end{minipage}

}

\end{table}%

\begin{figure}

\centering{

\pandocbounded{\includegraphics[keepaspectratio]{_figures/fig-anc-who-5-difference-from-baseline-rainplot.png}}

}

\caption{\label{fig-anc-who-5-difference-from-baseline-rainplot}Rainplot
for the WHO-5 Well-Being Index (WHO-5 difference from baseline).}

\end{figure}%

\newpage{}

\subsection{WAI-SR: Online-Coach}\label{wai-sr-online-coach}

\begin{table}

\caption{\label{tbl-anc-wai-coach-mmrm}Mixed model for repeated measures
(MMRM) for the Working Alliance Inventory-Short Revised (WAI-SR COACH).}

\centering{

\fontsize{8.0pt}{9.0pt}\selectfont
\begin{tabular*}{\linewidth}{@{\extracolsep{\fill}}lccc}
\toprule
\textbf{Variable} & \textbf{Beta} & \textbf{95\% CI} & \textbf{p-value} \\ 
\midrule\addlinespace[2.5pt]
{\bfseries (Intercept)} & 15 & 11 to 20 & <0·0001 \\ 
{\bfseries Type of supervision} &  &  &  \\ 
    community supervision & 0·00 & Ref. &  \\ 
    post-release supervision & 1·2 & -1·6 to 4·0 & 0·39 \\ 
{\bfseries Additional treatment} &  &  &  \\ 
    No & 0·00 & Ref. &  \\ 
    Yes & 1·1 & -1·1 to 3·3 & 0·34 \\ 
{\bfseries Offense type} &  &  &  \\ 
    Hands-off (only §184b StGB) & 0·00 & Ref. &  \\ 
    Hands-on (at least one conviction §176 ff StGB) & -1·2 & -3·4 to 1·1 & 0·31 \\ 
{\bfseries Static recidivism risk (baseline)} & -0·59 & -1·5 to 0·35 & 0·22 \\ 
{\bfseries Timepoint * treatment} &  &  &  \\ 
    Module 1 * Intervention & 2·4 & 0·30 to 4·5 & 0·025 \\ 
    Module 2 * Intervention & 3·8 & 1·2 to 6·5 & 0·0044 \\ 
    Module 3 * Intervention & 3·4 & 0·96 to 5·9 & 0·0067 \\ 
    Module 4 * Intervention & 2·1 & -0·63 to 4·9 & 0·13 \\ 
    Module 5 * Intervention & 4·3 & 1·3 to 7·3 & 0·0051 \\ 
{\bfseries Timepoint} &  &  &  \\ 
    Module 1 & 0·00 & Ref. &  \\ 
    Module 2 & -0·96 & -2·3 to 0·35 & 0·15 \\ 
    Module 3 & 0·39 & -0·92 to 1·7 & 0·56 \\ 
    Module 4 & 0·46 & -1·1 to 2·0 & 0·55 \\ 
    Module 5 & 1·1 & -0·68 to 2·9 & 0·22 \\ 
{\bfseries Baseline value (WAI-SR COACH (total score))} & 0·67 & 0·58 to 0·77 & <0·0001 \\ 
\bottomrule
\end{tabular*}
\begin{minipage}{\linewidth}
\vspace{.05em}
\parbox{\linewidth}{\raggedright {Abbreviations: CI = Confidence Interval, CI = Confidence interval, StGB = German penalty law.\\
}}\end{minipage}

}

\end{table}%

\begin{figure}

\centering{

\pandocbounded{\includegraphics[keepaspectratio]{_figures/fig-anc-wai-coach-estimated-marginal-means.png}}

}

\caption{\label{fig-anc-wai-coach-estimated-marginal-means}Estimated
marginal means from the MMRM of the Working Alliance Inventory-Short
Revised (WAI-SR COACH).}

\end{figure}%

\begin{table}

\caption{\label{tbl-anc-wai-coach-pre-post-pairwise-intervention}Pairwise
comparison of the Working Alliance Inventory-Short Revised (WAI-SR COACH
total score) for the intervention arm.}

\centering{

\fontsize{8.0pt}{9.0pt}\selectfont
\begin{tabular*}{\linewidth}{@{\extracolsep{\fill}}lcccc}
\toprule
\textbf{Module} & \textbf{pre}  N = 97 & \textbf{post}  N = 97 & \textbf{p} & \textbf{q-value} \\ 
\midrule\addlinespace[2.5pt]
{\bfseries Module 1 (post)} &  &  & 0·0020 & 0·010 \\ 
    Mean (SD) & 42 (11) & 44 (9) &  &  \\ 
    Median (Q1, Q3) & 43 (34, 49) & 47 (39, 51) &  &  \\ 
    Min, Max & 14, 60 & 22, 60 &  &  \\ 
{\bfseries Module 2 (post)} &  &  & 0·21 & 0·62 \\ 
    Mean (SD) & 45 (9) & 46 (10) &  &  \\ 
    Median (Q1, Q3) & 47 (41, 52) & 48 (39, 54) &  &  \\ 
    Min, Max & 23, 60 & 12, 60 &  &  \\ 
    Missing & 15 & 15 &  &  \\ 
{\bfseries Module 3 (post)} &  &  & 0·61 & 0·62 \\ 
    Mean (SD) & 47 (9) & 48 (9) &  &  \\ 
    Median (Q1, Q3) & 48 (40, 55) & 48 (42, 55) &  &  \\ 
    Min, Max & 27, 60 & 24, 60 &  &  \\ 
    Missing & 29 & 29 &  &  \\ 
{\bfseries Module 4 (post)} &  &  & 0·24 & 0·62 \\ 
    Mean (SD) & 47 (9) & 46 (10) &  &  \\ 
    Median (Q1, Q3) & 48 (42, 55) & 47 (39, 56) &  &  \\ 
    Min, Max & 24, 60 & 23, 60 &  &  \\ 
    Missing & 39 & 39 &  &  \\ 
{\bfseries Module 5 (post)} &  &  & 0·0040 & 0·016 \\ 
    Mean (SD) & 46 (11) & 49 (9) &  &  \\ 
    Median (Q1, Q3) & 47 (37, 57) & 49 (41, 57) &  &  \\ 
    Min, Max & 23, 60 & 23, 60 &  &  \\ 
    Missing & 50 & 50 &  &  \\ 
\bottomrule
\end{tabular*}
\begin{minipage}{\linewidth}
\vspace{.05em}
\parbox{\linewidth}{\raggedright {Abbreviation: Q1 = 25th percentile, Q3 = 75th percentile, p\textsubscript{adj} = Holm-Bonferroni adjusted p.\\
}}\end{minipage}

}

\end{table}%

\begin{table}

\caption{\label{tbl-anc-wai-coach-pre-post-pairwise-placebo}Pairwise
comparison of the Working Alliance Inventory-Short Revised (WAI-SR
COACH) for the placebo arm.}

\centering{

\fontsize{8.0pt}{9.0pt}\selectfont
\begin{tabular*}{\linewidth}{@{\extracolsep{\fill}}lcccc}
\toprule
\textbf{Module} & \textbf{pre}  N = 98 & \textbf{post}  N = 98 & \textbf{p} & \textbf{q-value} \\ 
\midrule\addlinespace[2.5pt]
{\bfseries Module 1 (post)} &  &  & 0·59 & >0·99 \\ 
    Mean (SD) & 42 (11) & 42 (11) &  &  \\ 
    Median (Q1, Q3) & 43 (34, 51) & 44 (34, 50) &  &  \\ 
    Min, Max & 12, 60 & 13, 60 &  &  \\ 
{\bfseries Module 2 (post)} &  &  & 0·52 & >0·99 \\ 
    Mean (SD) & 42 (12) & 41 (13) &  &  \\ 
    Median (Q1, Q3) & 44 (34, 50) & 42 (30, 53) &  &  \\ 
    Min, Max & 13, 60 & 12, 60 &  &  \\ 
    Missing & 16 & 16 &  &  \\ 
{\bfseries Module 3 (post)} &  &  & 0·28 & >0·99 \\ 
    Mean (SD) & 42 (14) & 43 (13) &  &  \\ 
    Median (Q1, Q3) & 44 (30, 53) & 46 (35, 53) &  &  \\ 
    Min, Max & 12, 60 & 12, 60 &  &  \\ 
    Missing & 24 & 24 &  &  \\ 
{\bfseries Module 4 (post)} &  &  & 0·56 & >0·99 \\ 
    Mean (SD) & 43 (13) & 43 (13) &  &  \\ 
    Median (Q1, Q3) & 46 (35, 53) & 47 (34, 54) &  &  \\ 
    Min, Max & 12, 60 & 12, 60 &  &  \\ 
    Missing & 32 & 32 &  &  \\ 
{\bfseries Module 5 (post)} &  &  & 0·40 & >0·99 \\ 
    Mean (SD) & 44 (13) & 44 (13) &  &  \\ 
    Median (Q1, Q3) & 48 (34, 54) & 48 (34, 55) &  &  \\ 
    Min, Max & 12, 60 & 14, 60 &  &  \\ 
    Missing & 43 & 43 &  &  \\ 
\bottomrule
\end{tabular*}
\begin{minipage}{\linewidth}
\vspace{.05em}
\parbox{\linewidth}{\raggedright {Abbreviation: Q1 = 25th percentile, Q3 = 75th percentile, p\textsubscript{adj} = Holm-Bonferroni adjusted p.\\
}}\end{minipage}

}

\end{table}%

\begin{table}

\caption{\label{tbl-anc-wai-coach-difference-from-baseline}Pairwise
comparisons of the pre-post differences of the Working Alliance
Inventory-Short Revised (WAI-SR COACH).}

\centering{

\fontsize{8.0pt}{9.0pt}\selectfont
\begin{tabular*}{\linewidth}{@{\extracolsep{\fill}}lcccc}
\toprule
\textbf{Characteristic} & \textbf{Placebo}  N = 98 & \textbf{Intervention}  N = 97 & \textbf{p} & \textbf{q-value} \\ 
\midrule\addlinespace[2.5pt]
{\bfseries Module 1 (post)} &  &  & 0·025 & 0·13 \\ 
    Median (Q1 – Q3) & 1 (-3 – 5) & 3 (-2 – 8) &  &  \\ 
{\bfseries Module 2 (post)} &  &  & 0·059 & 0·18 \\ 
    Median (Q1 – Q3) & 0 (-4 – 3) & 1 (-2 – 4) &  &  \\ 
    Missing & 16 & 15 &  &  \\ 
{\bfseries Module 3 (post)} &  &  & 0·39 & 0·39 \\ 
    Median (Q1 – Q3) & 0·0 (-2·0 – 4·0) & 0·0 (-3·0 – 2·0) &  &  \\ 
    Missing & 24 & 29 &  &  \\ 
{\bfseries Module 4 (post)} &  &  & 0·19 & 0·39 \\ 
    Median (Q1 – Q3) & 0·0 (-2·0 – 4·0) & -1·0 (-5·0 – 2·0) &  &  \\ 
    Missing & 32 & 39 &  &  \\ 
{\bfseries Module 5 (post)} &  &  & 0·036 & 0·14 \\ 
    Median (Q1 – Q3) & 0·0 (-2·0 – 4·0) & 2·0 (0·0 – 6·0) &  &  \\ 
    Missing & 43 & 50 &  &  \\ 
\bottomrule
\end{tabular*}
\begin{minipage}{\linewidth}
\vspace{.05em}
\parbox{\linewidth}{\raggedright {Abbreviation: Q1 = 25th percentile, Q3 = 75th percentile, SD = Standard deviance, p\textsubscript{adj} = Holm-Bonferroni adjusted p.\\
}}\end{minipage}

}

\end{table}%

\newpage{}

\subsection{WAI-SR: Community
supervisor}\label{wai-sr-community-supervisor}

\begin{table}

\caption{\label{tbl-anc-wai-supervisor-mmrm}Mixed model for repeated
measures (MMRM) for the Working Alliance Inventory-Short Revised (WAI-SR
SUPERVISOR).}

\centering{

\fontsize{8.0pt}{9.0pt}\selectfont
\begin{tabular*}{\linewidth}{@{\extracolsep{\fill}}lccc}
\toprule
\textbf{Variable} & \textbf{Beta} & \textbf{95\% CI} & \textbf{p-value} \\ 
\midrule\addlinespace[2.5pt]
{\bfseries (Intercept)} & 15 & 10 to 19 & <0·0001 \\ 
{\bfseries Type of supervision} &  &  &  \\ 
    community supervision & 0·00 & Ref. &  \\ 
    post-release supervision & 1·6 & -0·61 to 3·9 & 0·15 \\ 
{\bfseries Additional treatment} &  &  &  \\ 
    No & 0·00 & Ref. &  \\ 
    Yes & 1·0 & -0·75 to 2·8 & 0·26 \\ 
{\bfseries Offense type} &  &  &  \\ 
    Hands-off (only §184b StGB) & 0·00 & Ref. &  \\ 
    Hands-on (at least one conviction §176 ff StGB) & -1·2 & -3·0 to 0·62 & 0·20 \\ 
{\bfseries Static recidivism risk (baseline)} & -0·66 & -1·4 to 0·09 & 0·086 \\ 
{\bfseries Timepoint * treatment} &  &  &  \\ 
    Module 1 * Intervention & 2·3 & 0·64 to 4·1 & 0·0073 \\ 
    Module 2 * Intervention & 3·0 & 0·87 to 5·2 & 0·0061 \\ 
    Module 3 * Intervention & 3·2 & 1·2 to 5·3 & 0·0022 \\ 
    Module 4 * Intervention & 2·1 & -0·16 to 4·3 & 0·068 \\ 
    Module 5 * Intervention & 3·6 & 1·2 to 6·0 & 0·0038 \\ 
{\bfseries Timepoint} &  &  &  \\ 
    Module 1 & 0·00 & Ref. &  \\ 
    Module 2 & -0·14 & -1·3 to 0·96 & 0·80 \\ 
    Module 3 & 0·72 & -0·36 to 1·8 & 0·19 \\ 
    Module 4 & 1·1 & -0·04 to 2·3 & 0·059 \\ 
    Module 5 & 1·8 & 0·30 to 3·3 & 0·019 \\ 
{\bfseries Baseline value (WAI-SR SUPERVISOR (total score))} & 0·70 & 0·62 to 0·79 & <0·0001 \\ 
\bottomrule
\end{tabular*}
\begin{minipage}{\linewidth}
\vspace{.05em}
\parbox{\linewidth}{\raggedright {Abbreviations: CI = Confidence Interval, CI = Confidence interval, StGB = German penalty law.\\
}}\end{minipage}

}

\end{table}%

\begin{figure}

{\centering \pandocbounded{\includegraphics[keepaspectratio]{_figures/fig-anc-wai-supervisor-estimated-marginal-means.png}}

}

\caption{Estimated marginal means from the MMRM of Working Alliance
Inventory-Short Revised (WAI-SR SUPERVISOR).}

\end{figure}%

\begin{table}

\caption{\label{tbl-anc-wai-supervisor-pre-post-pairwise-intervention}Pairwise
comparison of the Working Alliance Inventory-Short Revised (WAI-SR
SUPERVISOR) for the intervention arm.}

\centering{

\fontsize{8.0pt}{9.0pt}\selectfont
\begin{tabular*}{\linewidth}{@{\extracolsep{\fill}}lcccc}
\toprule
\textbf{Module} & \textbf{pre}  N = 97 & \textbf{post}  N = 97 & \textbf{p} & \textbf{q-value} \\ 
\midrule\addlinespace[2.5pt]
{\bfseries Module 1 (post)} &  &  & <0·0001 & <0·0001 \\ 
    Mean (SD) & 43 (9) & 46 (8) &  &  \\ 
    Median (Q1, Q3) & 44 (36, 49) & 47 (42, 53) &  &  \\ 
    Min, Max & 22, 60 & 24, 60 &  &  \\ 
{\bfseries Module 2 (post)} &  &  & 0·14 & 0·35 \\ 
    Mean (SD) & 47 (8) & 47 (9) &  &  \\ 
    Median (Q1, Q3) & 48 (42, 53) & 48 (42, 54) &  &  \\ 
    Min, Max & 24, 60 & 12, 60 &  &  \\ 
    Missing & 15 & 15 &  &  \\ 
{\bfseries Module 3 (post)} &  &  & 0·12 & 0·35 \\ 
    Mean (SD) & 49 (8) & 50 (8) &  &  \\ 
    Median (Q1, Q3) & 51 (43, 55) & 51 (46, 57) &  &  \\ 
    Min, Max & 27, 60 & 27, 60 &  &  \\ 
    Missing & 29 & 29 &  &  \\ 
{\bfseries Module 4 (post)} &  &  & 0·44 & 0·44 \\ 
    Mean (SD) & 49 (8) & 49 (9) &  &  \\ 
    Median (Q1, Q3) & 51 (46, 57) & 49 (41, 57) &  &  \\ 
    Min, Max & 27, 60 & 31, 60 &  &  \\ 
    Missing & 39 & 39 &  &  \\ 
{\bfseries Module 5 (post)} &  &  & 0·046 & 0·18 \\ 
    Mean (SD) & 49 (9) & 50 (8) &  &  \\ 
    Median (Q1, Q3) & 49 (41, 57) & 52 (46, 57) &  &  \\ 
    Min, Max & 31, 60 & 33, 60 &  &  \\ 
    Missing & 50 & 50 &  &  \\ 
\bottomrule
\end{tabular*}
\begin{minipage}{\linewidth}
\vspace{.05em}
\parbox{\linewidth}{\raggedright {Abbreviation: Q1 = 25th percentile, Q3 = 75th percentile, p\textsubscript{adj} = Holm-Bonferroni adjusted p.\\
}}\end{minipage}

}

\end{table}%

\begin{table}

\caption{\label{tbl-anc-wai-supervisor-pre-post-pairwise-placebo}Pairwise
comparisons of the Working Alliance Inventory-Short Revised (WAI-SR
SUPERVISOR) for the placebo arm.}

\centering{

\fontsize{8.0pt}{9.0pt}\selectfont
\begin{tabular*}{\linewidth}{@{\extracolsep{\fill}}lcccc}
\toprule
\textbf{Module} & \textbf{pre}  N = 98 & \textbf{post}  N = 98 & \textbf{p} & \textbf{q-value} \\ 
\midrule\addlinespace[2.5pt]
{\bfseries Module 1 (post)} &  &  & 0·32 & >0·99 \\ 
    Mean (SD) & 44 (10) & 45 (10) &  &  \\ 
    Median (Q1, Q3) & 45 (38, 52) & 47 (37, 51) &  &  \\ 
    Min, Max & 16, 60 & 19, 60 &  &  \\ 
{\bfseries Module 2 (post)} &  &  & 0·72 & >0·99 \\ 
    Mean (SD) & 44 (10) & 44 (11) &  &  \\ 
    Median (Q1, Q3) & 47 (37, 52) & 47 (37, 53) &  &  \\ 
    Min, Max & 19, 60 & 14, 60 &  &  \\ 
    Missing & 16 & 16 &  &  \\ 
{\bfseries Module 3 (post)} &  &  & 0·21 & >0·99 \\ 
    Mean (SD) & 45 (11) & 45 (11) &  &  \\ 
    Median (Q1, Q3) & 48 (37, 53) & 48 (36, 55) &  &  \\ 
    Min, Max & 14, 60 & 16, 60 &  &  \\ 
    Missing & 24 & 24 &  &  \\ 
{\bfseries Module 4 (post)} &  &  & 0·45 & >0·99 \\ 
    Mean (SD) & 46 (11) & 46 (11) &  &  \\ 
    Median (Q1, Q3) & 48 (38, 54) & 49 (39, 54) &  &  \\ 
    Min, Max & 16, 60 & 16, 60 &  &  \\ 
    Missing & 32 & 32 &  &  \\ 
{\bfseries Module 5 (post)} &  &  & 0·59 & >0·99 \\ 
    Mean (SD) & 47 (10) & 47 (10) &  &  \\ 
    Median (Q1, Q3) & 49 (40, 55) & 48 (40, 57) &  &  \\ 
    Min, Max & 20, 60 & 23, 60 &  &  \\ 
    Missing & 43 & 43 &  &  \\ 
\bottomrule
\end{tabular*}
\begin{minipage}{\linewidth}
\vspace{.05em}
\parbox{\linewidth}{\raggedright {Abbreviation: Q1 = 25th percentile, Q3 = 75th percentile, p\textsubscript{adj} = Holm-Bonferroni adjusted p.\\
}}\end{minipage}

}

\end{table}%

\begin{table}

\caption{\label{tbl-anc-wai-supervisor-difference-from-baseline}Differences
from baseline of the Working Alliance Inventory-Short Revised (WAI-SR
SUPERVISOR).}

\centering{

\fontsize{8.0pt}{9.0pt}\selectfont
\begin{tabular*}{\linewidth}{@{\extracolsep{\fill}}lcccc}
\toprule
\textbf{Characteristic} & \textbf{Placebo}  N = 98 & \textbf{Intervention}  N = 97 & \textbf{p} & \textbf{q-value} \\ 
\midrule\addlinespace[2.5pt]
{\bfseries Module 1 (post)} &  &  & 0·0090 & 0·045 \\ 
    Median (Q1 – Q3) & 1 (-3 – 4) & 4 (-1 – 6) &  &  \\ 
{\bfseries Module 2 (post)} &  &  & 0·19 & 0·40 \\ 
    Median (Q1 – Q3) & 0·0 (-3·0 – 2·0) & 1·0 (-1·0 – 4·0) &  &  \\ 
    Missing & 16 & 15 &  &  \\ 
{\bfseries Module 3 (post)} &  &  & 0·51 & 0·51 \\ 
    Median (Q1 – Q3) & 1·0 (-2·0 – 3·0) & 0·0 (-2·0 – 3·0) &  &  \\ 
    Missing & 24 & 29 &  &  \\ 
{\bfseries Module 4 (post)} &  &  & 0·13 & 0·40 \\ 
    Median (Q1 – Q3) & 0·0 (-2·0 – 3·0) & -0·5 (-4·0 – 2·0) &  &  \\ 
    Missing & 32 & 39 &  &  \\ 
{\bfseries Module 5 (post)} &  &  & 0·061 & 0·24 \\ 
    Median (Q1 – Q3) & 0·0 (-2·0 – 2·0) & 1·0 (0·0 – 4·0) &  &  \\ 
    Missing & 43 & 50 &  &  \\ 
\bottomrule
\end{tabular*}
\begin{minipage}{\linewidth}
\vspace{.05em}
\parbox{\linewidth}{\raggedright {Abbreviation: Q1 = 25th percentile, Q3 = 75th percentile, SD = Standard deviance, p\textsubscript{adj} = Holm-Bonferroni adjusted p.\\
}}\end{minipage}

}

\end{table}%

\newpage{}

\section{CONSORT checklist}\label{consort-checklist}

\begin{table}

\caption{\label{tbl-consort-checklist}CONSORT 2025 checklist. Adapated
from\textsuperscript{74}.}

\centering{

\fontsize{8.0pt}{9.0pt}\selectfont
\begin{tabular*}{\linewidth}{@{\extracolsep{\fill}}>{\raggedright\arraybackslash}p{\dimexpr 60.00pt -2\tabcolsep-1.5\arrayrulewidth}>{\raggedright\arraybackslash}p{\dimexpr 60.00pt -2\tabcolsep-1.5\arrayrulewidth}>{\raggedright\arraybackslash}p{\dimexpr 60.00pt -2\tabcolsep-1.5\arrayrulewidth}>{\raggedright\arraybackslash}p{\dimexpr 112.50pt -2\tabcolsep-1.5\arrayrulewidth}>{\raggedright\arraybackslash}p{\dimexpr 60.00pt -2\tabcolsep-1.5\arrayrulewidth}}
\toprule
{\bfseries Section} & {\bfseries Subsection} & {\bfseries Number} & {\bfseries Description} & {\bfseries Reference} \\ 
\midrule\addlinespace[2.5pt]
{\bfseries Title and abstract} & {\bfseries } & {\bfseries } & {\bfseries } & {\bfseries } \\ 
 & Title and structured abstract & 1a & Identification as a randomised trial & p. 2 (Summary) \\ 
 &  & 1b & Structured summary of the trial design, methods, results, and conclusions & p. 2 (Summary) \\ 
{\bfseries Open science} & {\bfseries } & {\bfseries } & {\bfseries } & {\bfseries } \\ 
 & Trial registration & 2 & Name of trial registry, identifying number (with URL) and date of registration & p. 5 (Section 2.1) \\ 
 & Protocol and statistical analysis plan & 3 & Where the trial protocol and statistical analysis plan can be accessed & appendix section 1 (trial protocol); appendix section 11 (SAP) \\ 
 & Data sharing & 4 & Where and how the individual de-identified participant data (including data dictionary), statistical code, and any other materials can be accessed & p. 13 (Data sharing) \\ 
 & Funding and conflicts of interest & 5a & Sources of funding and other support (eg, supply of drugs), and role of funders in the design, conduct, analysis, and reporting of the trial & Summary; p. 8 (Section 2.6); p. 13 (Acknowledgements) \\ 
 &  & 5b & Financial and other conflicts of interest of the manuscript authors & p. 13 (Declaration of interests) \\ 
{\bfseries Introduction} & {\bfseries } & {\bfseries } & {\bfseries } & {\bfseries } \\ 
 & Background and rationale & 6 & Scientific background and rationale & p. 4-5 (Section 1) \\ 
 & Objectives & 7 & Specific objectives related to benefits and harms & p. 5 (Section 1) \\ 
{\bfseries Methods} & {\bfseries } & {\bfseries } & {\bfseries } & {\bfseries } \\ 
 & Patient and public involvement & 8 & Details of patient or public involvement in the design, conduct, and reporting of the trial & p. 5 Section 2.1) \\ 
 & Trial design & 9 & Description of trial design including type of trial (eg, parallel group, crossover), allocation ratio, and framework (eg, superiority, equivalence, non-inferiority, or exploratory) & p. 5 (Section 2.1) \\ 
 & Changes to trial protocol & 10 & Important changes to the trial after it commenced including any outcomes or analyses that were not prespecified, with reason & p. 6 (Section 2.4) \\ 
 & Trial setting & 11 & Settings (eg, community, hospital) and locations (eg, countries, sites) where the trial was conducted & p. 5 (Section 2.1) \\ 
 & Eligibility criteria & 12a & Eligibility criteria for participants & p. 5 (Section 2.1) \\ 
 &  & 12b & If applicable, eligibility criteria for sites and for individuals delivering the interventions (eg, surgeons, physiotherapists) & not applicable \\ 
 & Intervention and comparator & 13 & Intervention and comparator with sufficient details to allow replication. If relevant, where additional materials describing the intervention and comparator (eg, intervention manual) can be accessed & p. 5 (Section 2.3) \\ 
 & Outcomes & 14 & Prespecified primary and secondary outcomes, including the specific measurement variable (eg, systolic blood pressure), analysis metric (eg, change from baseline, final value, time to event), method of aggregation (eg, median, proportion), and timepoint for each outcome & p. 6-7 (Section 2.4); Table 2; appendix Section 3 \\ 
 & Harms & 15 & How harms were defined and assessed (eg, systematically, non-systematically) & p. 6 (Section 2.4) \\ 
 & Sample size & 16a & How sample size was determined, including all assumptions supporting the sample size calculation & p. 7 (Section 2.5) \\ 
 &  & 16b & Explanation of any interim analyses and stopping guidelines & p. 7 (Section 2.5) \\ 
{\bfseries Randomisation} & {\bfseries } & {\bfseries } & {\bfseries } & {\bfseries } \\ 
 & Sequence generation & 17a & Who generated the random allocation sequence and the method used & p. 5 (Section 2.2) \\ 
 &  & 17b & Type of randomisation and details of any restriction (eg, stratification, blocking and block size) & p. 5 (Section 2.2) \\ 
 & Allocation concealment mechanism & 18 & Mechanism used to implement the random allocation sequence (eg, central computer/telephone; sequentially numbered, opaque, sealed containers), describing any steps to conceal the sequence until interventions were assigned & p. 5 (Section 2.2) \\ 
 & Implementation & 19 & Whether the personnel who enrolled and those who assigned participants to the interventions had access to the random allocation sequence & p. 5 (Section 2.2) \\ 
 & Blinding & 20a & Who was blinded after assignment to interventions (eg, participants, care providers, outcome assessors, data analysts) & p. 5 (Section 2.2) \\ 
 &  & 20b & If blinded, how blinding was achieved and description of the similarity of interventions & p. 5 (Section 2.2) \\ 
 & Statistical methods & 21a & Statistical methods used to compare groups for primary and secondary outcomes, including harms & p. 7-8 (Section 2.5) \\ 
 &  & 21b & Definition of who is included in each analysis (eg, all randomised participants), and in which group & p. 7-8 (Section 2.5) \\ 
 &  & 21c & How missing data were handled in the analysis & p. 7-8 (Section 2.5) \\ 
 &  & 21d & Methods for any additional analyses (eg, subgroup and sensitivity analyses), distinguishing prespecified from post hoc & p. 7-8 (Section 2.5) \\ 
{\bfseries Results} & {\bfseries } & {\bfseries } & {\bfseries } & {\bfseries } \\ 
 & Participant flow, including flow diagram & 22a & For each group, the numbers of participants who were randomly assigned, received intended intervention, and were analysed for the primary outcome & p. 8 (Section 3) and Figure 1 \\ 
 &  & 22b & For each group, losses and exclusions after randomisation, together with reasons & Figure 1 \\ 
 & Recruitment & 23a & Dates defining the periods of recruitment and follow-up for outcomes of benefits and harms & p. 8 (Section 3) \\ 
 &  & 23b & If relevant, why the trial ended or was stopped & p. 8 (Section 3) \\ 
 & Intervention and comparator delivery & 24a & Intervention and comparator as they were actually administered (eg, where appropriate, who delivered the intervention/comparator, how participants adhered, whether they were delivered as intended [fidelity]) & Table 1; Figure 1; p. 5-6 (Section 2.3) \\ 
 &  & 24b & Concomitant care received during the trial for each group & p. 8 (Section 3) \\ 
 & Baseline data & 25 & A table showing baseline demographic and clinical characteristics for each group & Table 3 \\ 
 & Numbers analysed, outcomes and estimation & 26 & For each primary and secondary outcome, by group: •the number of participants included in the analysis •the number of participants with available data at the outcome timepoint •result for each group, and the estimated effect size and its precision (such as 95\% confidence interval) •for binary outcomes, presentation of both absolute and relative effect size & p. 8-11 (Section 3); appendix \\ 
 & Harms & 27 & All harms or unintended events in each group & p. 10 (Section 3) \\ 
 & Ancillary analyses & 28 & Any other analyses performed, including subgroup and sensitivity analyses, distinguishing prespecified from post hoc & pp. 10-11 (Section 3); appendix \\ 
{\bfseries Discussion} & {\bfseries } & {\bfseries } & {\bfseries } & {\bfseries } \\ 
 & Interpretation & 29 & Interpretation consistent with results, balancing benefits and harms, and considering other relevant evidence & p. 11-12 (Section 4) \\ 
 & Limitations & 30 & Trial limitations, addressing sources of potential bias, imprecision, generalisability, and, if relevant, multiplicity of analyses & p. 11-12 (Section 4) \\ 
\bottomrule
\end{tabular*}

}

\end{table}%

\newpage{}

\section{Statistical analysis plan}\label{statistical-analysis-plan}

\includepdf[pages=-, fitpaper=true]{resources/myTabu_SAP_signed.pdf}

\newpage{}

\section{References}\label{references}

\phantomsection\label{refs}
\begin{CSLReferences}{0}{1}
\bibitem[\citeproctext]{ref-frombergerMyTabuAPlaceboControlled2021}
\CSLLeftMargin{1 }%
\CSLRightInline{Fromberger P, Schröder S, Bauer L, \emph{et al.}
\href{https://doi.org/10.3389/fpsyt.2020.575464}{@{myTabu}---{A Placebo
Controlled Randomized Trial} of a {Guided Web-Based Intervention} for
{Individuals Who Sexually Abused Children} and {Individuals Who Consumed
Child Sexual Exploitation Material}: {A Clinical Study Protocol}}.
\emph{Frontiers in Psychiatry} 2021; \textbf{11}: 575464.}

\bibitem[\citeproctext]{ref-brikenPraeventionSexuellenKindesmissbrauchs2017}
\CSLLeftMargin{2 }%
\CSLRightInline{Briken P, Berner W, Flöter A, Jückstock V, von Franqué
F. \href{https://doi.org/10.1055/s-0043-100462}{{Prävention sexuellen
Kindesmissbrauchs im Dunkelfeld -- das Hamburger Modell}}. \emph{PSYCH
up2date} 2017; \textbf{11}: 243--62.}

\bibitem[\citeproctext]{ref-wildPreventionSexualChild2020}
\CSLLeftMargin{3 }%
\CSLRightInline{Wild TSN, Müller I, Fromberger P, Jordan K, Klein L,
Müller JL. \href{https://doi.org/10.3389/fpsyt.2020.00088}{Prevention of
{Sexual Child Abuse}: {Preliminary Results From} an {Outpatient Therapy
Program}}. \emph{Frontiers in Psychiatry} 2020; \textbf{11}: 88.}

\bibitem[\citeproctext]{ref-mannAssessingRiskSexual2010}
\CSLLeftMargin{4 }%
\CSLRightInline{Mann RE, Hanson RK, Thornton D.
\href{https://doi.org/10.1177/1079063210366039}{Assessing {Risk} for
{Sexual Recidivism}: {Some Proposals} on the {Nature} of
{Psychologically Meaningful Risk Factors}}. \emph{Sexual Abuse} 2010;
\textbf{22}: 191--217.}

\bibitem[\citeproctext]{ref-setoEmpiricallybasedDynamicRisk2023}
\CSLLeftMargin{5 }%
\CSLRightInline{Seto MC, Augustyn C, Roche KM, Hilkes G.
\href{https://doi.org/10.1016/j.cpr.2023.102355}{Empirically-based
dynamic risk and protective factors for sexual offending}.
\emph{Clinical Psychology Review} 2023; \textbf{106}: 102355.}

\bibitem[\citeproctext]{ref-babchishinChildSexualExploitation2018}
\CSLLeftMargin{6 }%
\CSLRightInline{Babchishin KM, Merdian HL, Bartels RM, Perkins D.
\href{https://doi.org/10.1027/1016-9040/a000326}{Child {Sexual
Exploitation Materials Offenders}: {A Review}}. \emph{European
Psychologist} 2018; \textbf{23}: 130--43.}

\bibitem[\citeproctext]{ref-reichelReviewRiskFactors2024}
\CSLLeftMargin{7 }%
\CSLRightInline{Reichel R, Daser A, Gnielka FM, Schmidt AF, Blokland A,
Lehmann RJB. \href{https://doi.org/10.1080/13552600.2024.2418100}{A
review of risk factors for online and mixed child sexual abuse material
offending: What is being researched?} \emph{Journal of Sexual
Aggression} 2024; : 1--44.}

\bibitem[\citeproctext]{ref-setoMotivationFacilitationModelSexual2019}
\CSLLeftMargin{8 }%
\CSLRightInline{Seto MC.
\href{https://doi.org/10.1177/1079063217720919}{The
{Motivation-Facilitation Model} of {Sexual Offending}}. \emph{Sexual
Abuse} 2019; \textbf{31}: 3--24.}

\bibitem[\citeproctext]{ref-rollnickMotivationalInterviewingTreatment2015}
\CSLLeftMargin{9 }%
\CSLRightInline{Rollnick S, Miller WR, Arkowitz H. Motivational
{Interviewing} in the {Treatment} of {Psychological Problems}, 2nd edn.
New York, NY: Guilford Press, 2015.}

\bibitem[\citeproctext]{ref-wardMultifactorOffenderReadiness2004}
\CSLLeftMargin{10 }%
\CSLRightInline{Ward T, Day A, Howells K, Birgden A. The {Multifactor
Offender Readiness Model}. 2004; \textbf{9}.}

\bibitem[\citeproctext]{ref-terryMotivationSexOffender2001}
\CSLLeftMargin{11 }%
\CSLRightInline{Terry KJ, Mitchell EW.
\href{https://doi.org/10.1177/0306624X01456003}{Motivation and sex
offender treatment efficacy: Leading a horse to water and making it
drink? {Int J Offender Ther Compar Criminol}}. 2001; \textbf{45}:
663--72.}

\bibitem[\citeproctext]{ref-postelEffectivenessWebbasedIntervention2010}
\CSLLeftMargin{12 }%
\CSLRightInline{Postel MG, Haan HA, Ter Huurne ED, Becker ES, Jong CA.
\href{https://doi.org/10.2196/jmir.1642}{Effectiveness of a web-based
intervention for problem drinkers and reasons for dropout: Randomized
controlled trial}. \emph{J Med Internet Res} 2010; : 12 68.}

\bibitem[\citeproctext]{ref-melvilleDropoutInternetbasedTreatment2010}
\CSLLeftMargin{13 }%
\CSLRightInline{Melville KM, Casey LM, Kavanagh DJ.
\href{https://doi.org/10.1348/014466509X472138}{Dropout from
internet-based treatment for psychological disorders}. \emph{Br J Clin
Psychol} 2010; \textbf{49}: 455--71.}

\bibitem[\citeproctext]{ref-andrewsPsychologyCriminalConduct2010}
\CSLLeftMargin{14 }%
\CSLRightInline{Andrews DA, Bonta J. The {Psychology} of {Criminal
Conduct}, 5th edn. New Providence, NJ: Lexis Nexis, 2010.}

\bibitem[\citeproctext]{ref-hansonPredictorsSexualRecidivsm2004}
\CSLLeftMargin{15 }%
\CSLRightInline{Hanson RK, Morton-Bourgon KE. Predictors of {Sexual
Recidivsm}: {An Updated Meta-Analysis}. Ottawa, ON: Public Safety
Canada, 2004.}

\bibitem[\citeproctext]{ref-hansonAssessingRiskSexual2007}
\CSLLeftMargin{16 }%
\CSLRightInline{Hanson RK, Harris AJ, Scott TL, Helmus L. Assessing the
{Risk} of {Sexual Offenders} on {Community Supervision}: {The Dynamic
Supervision Project}. Ottawa, ON: Public Safety Canada, 2007.}

\bibitem[\citeproctext]{ref-knightEvaluatingImprovingRisk2007}
\CSLLeftMargin{17 }%
\CSLRightInline{Knight RA, Thornton D. Evaluating and {Improving Risk
Assessment Schemes} for {Sexual Recidivism}: {A Long-Term Follow-up} of
{Convicted Sexual Offenders}. Washington, DC: U. S. Department of
Justice, 2007.}

\bibitem[\citeproctext]{ref-linehanCourseEvolutionDialectical2015}
\CSLLeftMargin{18 }%
\CSLRightInline{Linehan MM, Wilks CR.
\href{https://doi.org/10.1176/appi.psychotherapy.2015.69.2.97}{The
course and evolution of dialectical behavior therapy}. \emph{Am J
Psychother} 2015; \textbf{69}: 97--110.}

\bibitem[\citeproctext]{ref-hayesAcceptanceCommitmentTherapy1999}
\CSLLeftMargin{19 }%
\CSLRightInline{Hayes SC, Strosahl K, Wilson KG. Acceptance and
{Commitment Therapy}: {An Experiential Approach} to {Behavior Change}.
New York, NY: Guilford Press, 1999.}

\bibitem[\citeproctext]{ref-berzinsDevelopmentImplementationDialectical2004}
\CSLLeftMargin{20 }%
\CSLRightInline{Berzins LG, Trestman RL.
\href{https://doi.org/10.1080/14999013.2004.10471199}{The development
and implementation of dialectical behavior therapy in forensic
settings}. \emph{Int J Forensic Mental Health} 2004; \textbf{3}:
93--103.}

\bibitem[\citeproctext]{ref-spijkermanEffectivenessOnlineMindfulnessbased2016}
\CSLLeftMargin{21 }%
\CSLRightInline{Spijkerman MPJ, Pots WTM, Bohlmeijer ET.
\href{https://doi.org/10.1016/j.cpr.2016.03.009}{Effectiveness of online
mindfulness-based interventions in improving mental health: A review and
meta-analysis of randomised controlled trials}. \emph{Clin Psychol Rev}
2016; \textbf{45}: 102--14.}

\bibitem[\citeproctext]{ref-jayawardeneEffectsPreventiveOnline2016}
\CSLLeftMargin{22 }%
\CSLRightInline{Jayawardene WP, Lohrmann DK, Erbe RG, Torabi MR.
\href{https://doi.org/10.1016/j.pmedr.2016.11.013}{Effects of preventive
online mindfulness interventions on stress and mindfulness: A
meta-analysis of randomized controlled trials}. \emph{Prev Med Rep}
2016; \textbf{5}: 150--9.}

\bibitem[\citeproctext]{ref-lanzaAcceptanceCommitmentTherapy2014}
\CSLLeftMargin{23 }%
\CSLRightInline{Lanza PV, Garcia PF, Lamelas FR, González-Menéndez A.
\href{https://doi.org/10.1002/jclp.22060}{Acceptance and commitment
therapy versus cognitive behavioral therapy in the treatment of
substance use disorder with incarcerated women}. \emph{J Clin Psychol}
2014; \textbf{70}: 644--57.}

\bibitem[\citeproctext]{ref-dzurillaSocialProblemSolving2004}
\CSLLeftMargin{24 }%
\CSLRightInline{D'Zurilla TJ, Nezu AM, Maydeu-Olivares A.
\href{https://doi.org/10.1037/10805-001}{Social {Problem Solving}:
{Theory} and {Assessment}.} In: Chang EC, D'Zurilla TJ, Sanna LJ, eds.
Social problem solving: {Theory}, research, and training. Washington:
American Psychological Association, 2004: 11--27.}

\bibitem[\citeproctext]{ref-dzurillaProblemSolvingBehavior1971}
\CSLLeftMargin{25 }%
\CSLRightInline{D'Zurilla TJ, Godfried MR.
\href{https://doi.org/10.1037/h0031360}{Problem solving and behavior
modification}. \emph{J Abnorm Psychol} 1971; \textbf{78}: 107--26.}

\bibitem[\citeproctext]{ref-nezuProblemSolvingTherapyTreatment2013}
\CSLLeftMargin{26 }%
\CSLRightInline{Nezu AM, Nezu CM, D'Zurilla T. Problem-{Solving
Therapy}: {A Treatment Manual}. New York, NY: Springer Pub. Co, 2013.}

\bibitem[\citeproctext]{ref-wakelingPsychometricValidationSocial2007}
\CSLLeftMargin{27 }%
\CSLRightInline{Wakeling HC.
\href{https://doi.org/10.1177/107906320701900304}{The psychometric
validation of the social problem-solving inventory-revised with {UK}
incarcerated sexual offenders}. \emph{Sexual Abuse} 2007; \textbf{19}:
217--36.}

\bibitem[\citeproctext]{ref-mcguireReviewEffectiveInterventions2008}
\CSLLeftMargin{28 }%
\CSLRightInline{McGuire J.
\href{https://doi.org/10.1098/rstb.2008.0035}{A review of effective
interventions for reducing aggression and violence}. \emph{Philosophical
Transactions of the Royal Society B: Biological Sciences} 2008;
\textbf{363}: 2577--97.}

\bibitem[\citeproctext]{ref-mannTreatingCognitiveComponents2016}
\CSLLeftMargin{29 }%
\CSLRightInline{Mann RE, Barnett GD.
\href{https://doi.org/10.1002/9781118574003.wattso066}{Treating
cognitive components of sexual offending}. In: Marshall WL, Marshall LE,
eds. The {Wiley Handbook} on the {Theories}, {Assessment}, and
{Treatment} of {Sexual Offending}. Wiley, 2016: 1385--401.}

\bibitem[\citeproctext]{ref-helmusAttitudesSupportiveSexual2013}
\CSLLeftMargin{30 }%
\CSLRightInline{Helmus L, Hanson RK, Babchishin KM, Mann RE.
\href{https://doi.org/10.1177/1524838012462244}{Attitudes supportive of
sexual offending predict recidivism: A meta-analysis}. \emph{Trauma
Violence Abuse} 2013; \textbf{14}: 34--53.}

\bibitem[\citeproctext]{ref-beechAssessmentTreatmentDistorted2013}
\CSLLeftMargin{31 }%
\CSLRightInline{Beech AR, Bartels RM, Dixon L.
\href{https://doi.org/10.1177/1524838012463970}{Assessment and
{Treatment} of {Distorted Schemas} in {Sexual Offenders}}. \emph{Trauma,
Violence, \& Abuse} 2013; \textbf{14}: 54--66.}

\bibitem[\citeproctext]{ref-lalumiereTestMateDeprivation1996}
\CSLLeftMargin{32 }%
\CSLRightInline{Lalumière ML, Chalmers LJ, Quinsey VL, Seto MC.
\href{https://doi.org/10.1016/S0162-3095(96)00076-3}{A test of the mate
deprivation hypothesis of sexual coercion}. \emph{Ethol Sociobiol} 1996;
\textbf{17}: 299--318.}

\bibitem[\citeproctext]{ref-matthesDeutscheVersionStable20072008}
\CSLLeftMargin{33 }%
\CSLRightInline{Matthes A, Rettenberger M. {Die Deutsche Version Des
Stable-2007}. Wien: Institut für Gewaltforschung und Prävention, 2008.}

\bibitem[\citeproctext]{ref-wardIntegratedTheorySexual2006}
\CSLLeftMargin{34 }%
\CSLRightInline{Ward T, Beech A.
\href{https://doi.org/10.1016/j.avb.2005.05.002}{An integrated theory of
sexual offending}. \emph{Aggress Violent Behav} 2006; \textbf{11}:
44--63.}

\bibitem[\citeproctext]{ref-wardDynamicRiskFactors2016}
\CSLLeftMargin{35 }%
\CSLRightInline{Ward T.
\href{https://doi.org/10.1080/1068316X.2015.1109094}{Dynamic risk
factors: Scientific kinds or predictive constructs}. \emph{Psychology,
Crime \& Law} 2016; \textbf{22}: 2--16.}

\bibitem[\citeproctext]{ref-olverEvaluatingChangeMen2020}
\CSLLeftMargin{36 }%
\CSLRightInline{Olver M, Stockdale K. Evaluating change in men who have
sexually offended: Linkages to risk assessment and management.
\emph{Curr Psychiatry Rep} 2020; \textbf{22}.
DOI:\href{https://doi.org/10.1007/s11920-020-01146-3}{10.1007/s11920-020-01146-3}.}

\bibitem[\citeproctext]{ref-franquefWhichTechniquesAre2015}
\CSLLeftMargin{37 }%
\CSLRightInline{Franqué F, V K, P B.
\href{https://doi.org/10.1002/smrj.34}{Which techniques are used in
psychotherapeutic interventions for nonparaphilic hypersexual behavior?}
\emph{Sexual Med Rev} 2015; \textbf{3}: 3--10.}

\bibitem[\citeproctext]{ref-marshall2011}
\CSLLeftMargin{38 }%
\CSLRightInline{Marshall WL, Marshall LE, Serran GA, O'Brien MD.
Washington, DC: American Psychological Association, 2011.}

\bibitem[\citeproctext]{ref-guseInterventionsUsingNew2012}
\CSLLeftMargin{39 }%
\CSLRightInline{Guse K, Levine D, Martins S, Lira A, Gaarde J,
Westmorland W.
\href{https://doi.org/10.1016/j.jadohealth.2012.03.014}{Interventions
using new digital media to improve adolescent sexual health: A
systematic review}. \emph{J Adolesc Health} 2012; \textbf{51}: 535--43.}

\bibitem[\citeproctext]{ref-baileyInternetStudyMen2016}
\CSLLeftMargin{40 }%
\CSLRightInline{Bailey JM, Hsu KJ, Bernhard PA.
\href{https://doi.org/10.1037/abn0000212}{An {Internet} study of men
sexually attracted to children: {Sexual} attraction patterns.}
\emph{Journal of Abnormal Psychology} 2016; \textbf{125}: 976--88.}

\bibitem[\citeproctext]{ref-loeweGesundheitsfragebogenFuerPatienten2002}
\CSLLeftMargin{41 }%
\CSLRightInline{Löwe B, Spitzer R, Zipfel S, Herzog W.
{Gesundheitsfragebogen Für Patienten (PHQ-D}. \emph{Komplett Kurzform
Testmappe Manual Fragebögen Schablonen} 2002; \textbf{2}: 5--7.}

\bibitem[\citeproctext]{ref-graefeScreeningPsychischerStoerungen2004}
\CSLLeftMargin{42 }%
\CSLRightInline{Gräfe K, Zipfel S, Herzog W, Löwe B.
\href{https://doi.org/10.1026/0012-1924.50.4.171}{{Screening psychischer
störungen mit dem {`Gesundheitsfragebogen für Patienten (PHQ-D)'}}}.
\emph{Diagnostica} 2004; \textbf{50}: 171--81.}

\bibitem[\citeproctext]{ref-setoRevisedScreeningScale2017a}
\CSLLeftMargin{43 }%
\CSLRightInline{Seto MC, Stephens S, Lalumière ML, Cantor JM.
\href{https://doi.org/10.1177/1079063215612444}{The {Revised Screening
Scale} for {Pedophilic Interests} ({SSPI}--2): {Development} and
{Criterion-Related Validation}}. \emph{Sexual Abuse} 2017; \textbf{29}:
619--35.}

\bibitem[\citeproctext]{ref-screenerInstituteSexResearch2019}
\CSLLeftMargin{44 }%
\CSLRightInline{Screener BPICDS. Institute for {Sex Research}, {Sexual
Medicine}, and {Forensic Psychiatry}. Hamburg: University Medical Center
Hamburg-Eppendorf, 2019.}

\bibitem[\citeproctext]{ref-sonjaetzlerDevelopmentInitialValidationsubmitted}
\CSLLeftMargin{45 }%
\CSLRightInline{Sonja Etzler, Katharina Nitsche, Ann-Sophie Tröger,
Peter Fromberger, Martin Rettenberger. Development and {Initial
Validation} of the {Child Sexual Abuse Risk Evaluation Self-Report}
({CARES}) {Scales}: {Assessment} of {Acute-} and {Stable-Dynamic Risk
Factors} via {Self-Report} in {Individuals Convicted} of {Sexual
Offenses Against Children}. submitted; published online submitted.}

\bibitem[\citeproctext]{ref-caseyAssessingSuitabilityOffender2007}
\CSLLeftMargin{46 }%
\CSLRightInline{Casey S, Day A, Howells K, Ward T.
\href{https://doi.org/10.1177/0093854807305827}{Assessing {Suitability}
for {Offender Rehabilitation}: {Development} and {Validation} of the
{Treatment Readiness Questionnaire}}. \emph{Criminal Justice and
Behavior} 2007; \textbf{34}: 1427--40.}

\bibitem[\citeproctext]{ref-rollnickDevelopmentShortReadiness1992}
\CSLLeftMargin{47 }%
\CSLRightInline{Rollnick S, Heather N, Gold R, Hall W.
\href{https://doi.org/10.1111/j.1360-0443.1992.tb02720.x}{Development of
a short {`readiness to change'} questionnaire for use in brief,
opportunistic interventions among excessive drinkers}. \emph{Br J
Addict} 1992; \textbf{87}: 743--54.}

\bibitem[\citeproctext]{ref-hannoverReadinessChangeQuestionnaire2002}
\CSLLeftMargin{48 }%
\CSLRightInline{Hannöver W, Thyrian H JR, U R, HJ M, C J, U.
\href{https://doi.org/10.1093/alcalc/37.4.362}{The readiness to change
questionnaire in subjects with hazardous alcohol consumption, alcohol
misuse and dependence in a general population survey}. \emph{Alcohol
Alcohol} 2002; \textbf{37}: 362--9.}

\bibitem[\citeproctext]{ref-merzQuestionnaireMeasurementPsychological1983}
\CSLLeftMargin{49 }%
\CSLRightInline{Merz J. {A Questionnaire for the measurement of
psychological reactance}. \emph{Diagnostica} 1983; \textbf{29}: 75--82.}

\bibitem[\citeproctext]{ref-herzbergZurPsychometrischenOptimierung2002}
\CSLLeftMargin{50 }%
\CSLRightInline{Herzberg PY.
\href{https://doi.org/10.1026//0012-1924.48.4.163}{{Zur psychometrischen
Optimierung einer Reaktanzskala mittels klassischer und IRT-basierter
Analysemethoden}}. \emph{Diagnostica} 2002; \textbf{48}: 163--71.}

\bibitem[\citeproctext]{ref-dunkelEvaluationShortformSocial2005}
\CSLLeftMargin{51 }%
\CSLRightInline{Dunkel D, Antretter E, Fröhlich-Walser S, Haring C.
\href{https://doi.org/10.1055/s-2004-834746}{Evaluation of the
short-form social support questionnaire ({SOZU-K-22}) in clinical and
non-clinical samples}. \emph{Psychother Psychosom Med Psychol} 2005;
\textbf{55}: 266--77.}

\bibitem[\citeproctext]{ref-fydrichFragebogenZurSozialen}
\CSLLeftMargin{52 }%
\CSLRightInline{Fydrich T, Sommer G, Tydecks S, Brähler E. {Fragebogen
zur sozialen Unterstützung (F-SozU): Normierung der Kurzform (K-14)}. ;
: 6.}

\bibitem[\citeproctext]{ref-russellUCLALonelinessScale1996}
\CSLLeftMargin{53 }%
\CSLRightInline{Russell DW.
\href{https://doi.org/10.1207/s15327752jpa6601_2}{{UCLA} loneliness
scale (version 3): Reliability, validity, and factor structure}. \emph{J
Pers Assess} 1996; \textbf{66}: 20--40.}

\bibitem[\citeproctext]{ref-doringPsychometrischeEinsamkeitsforschungDeutsche1993}
\CSLLeftMargin{54 }%
\CSLRightInline{Döring N, Bortz J. Psychometrische
{Einsamkeitsforschung}: {Deutsche Neukonstruktion} der {UCLA Loneliness
Scale}. \emph{Diagnostica} 1993; \textbf{39}: 224--39.}

\bibitem[\citeproctext]{ref-neutzeUndetectedDetectedChild2012}
\CSLLeftMargin{55 }%
\CSLRightInline{Neutze J, Grundmann D, Scherner G, Beier KM.
\href{https://doi.org/10.1016/j.ijlp.2012.02.004}{Undetected and
detected child sexual abuse and child pornography offenders}.
\emph{International Journal of Law and Psychiatry} 2012; \textbf{35}:
168--75.}

\bibitem[\citeproctext]{ref-gratzMultidimensionalAssessmentEmotion2004}
\CSLLeftMargin{56 }%
\CSLRightInline{Gratz KL, Roemer L.
\href{https://doi.org/10.1023/B:JOBA.0000007455.08539.94}{Multidimensional
assessment of emotion regulation and dysregulation: Development, factor
structure, and initial validation of the difficulties in emotion
regulation scale}. \emph{J Psychopathol Behav Assess} 2004; \textbf{26}:
41--54.}

\bibitem[\citeproctext]{ref-ehringCharacteristicsEmotionRegulation2008}
\CSLLeftMargin{57 }%
\CSLRightInline{Ehring T, Fischer S, Schnülle J, Bösterling A,
Tuschen-Caffier B.
\href{https://doi.org/10.1016/j.paid.2008.01.013}{Characteristics of
emotion regulation in recovered depressed versus never depressed
individuals}. \emph{Pers Indiv Differ} 2008; \textbf{44}: 1574--84.}

\bibitem[\citeproctext]{ref-schererNegativeAffectRepair2013}
\CSLLeftMargin{58 }%
\CSLRightInline{Scherer A, Eberle N, Boecker M, Vögele C, Gauggel S,
Forkmann T. \href{https://doi.org/10.1186/1471-244X-13-16}{The negative
affect repair questionnaire: Factor analysis and psychometric evaluation
in three samples}. \emph{BMC Psychiatry} 2013; \textbf{13}: 16.}

\bibitem[\citeproctext]{ref-pattonFactorStructureBarratt1995}
\CSLLeftMargin{59 }%
\CSLRightInline{Patton JH, Stanford MS, Barratt ES. Factor structure of
the {Barratt} impulsiveness scale. \emph{J Clin Psychol} 1995;
\textbf{51}: 768--74.}

\bibitem[\citeproctext]{ref-meulePsychometrischeEvaluationDeutschen2011}
\CSLLeftMargin{60 }%
\CSLRightInline{Meule A, Vögele C, Kübler A.
\href{https://doi.org/10.1037/t63911-000}{{Psychometrische Evaluation
Der Deutschen Barratt Impulsiveness Scale--Kurzversion (BIS-15).
Diagnostica}}. 2011; \textbf{57}: 126--33.}

\bibitem[\citeproctext]{ref-cortoniSexCopingStrategy2001}
\CSLLeftMargin{61 }%
\CSLRightInline{Cortoni F, Marshall WL.
\href{https://doi.org/10.1177/107906320101300104}{Sex {As} a {Coping
Strategy} and {Its Relationship} to {Juvenile Sexual History} and
{Intimacy} in {Sexual Offenders}}. \emph{Sexual Abuse} 2001;
\textbf{13}: 27--43.}

\bibitem[\citeproctext]{ref-grafPsychometrischeUeberpruefungDeutschsprachigen2003}
\CSLLeftMargin{62 }%
\CSLRightInline{Graf A.
\href{https://doi.org/10.1024/0170-1789.24.4.277}{{Psychometrische
Überprüfung einer deutschsprachigen Übersetzung des SPSI-R. Z Differ
Diagn Psychol}}. 2003; \textbf{24}: 277--91.}

\bibitem[\citeproctext]{ref-bumbyAssessingCognitiveDistortions}
\CSLLeftMargin{63 }%
\CSLRightInline{Bumby KM. Assessing the cognitive distortions of child
molesters and rapists: {Development} and validation of the {MOLEST} and
{RAPE} scales. ; : 18.}

\bibitem[\citeproctext]{ref-feelgoodKVMSkalaZurErfassung2009}
\CSLLeftMargin{64 }%
\CSLRightInline{Feelgood S, Schaefer GA, Hoyer J. {KV-M-Skala} zur
{Erfassung} kognitiver {Verzerrungen} bei {Missbrauchern}. 2009.}

\bibitem[\citeproctext]{ref-kleinValidierungsstudieDeutschenVersion2013}
\CSLLeftMargin{65 }%
\CSLRightInline{Klein V, Rettenberger M, Boom K-D, Briken P.
\href{https://doi.org/10.1055/s-0033-1357133}{{Eine Validierungsstudie
der deutschen Version des Hypersexual Behavior Inventory (HBI)}}.
\emph{PPmP - Psychotherapie {\(\cdot\)} Psychosomatik {\(\cdot\)}
Medizinische Psychologie} 2013; \textbf{64}: 136--40.}

\bibitem[\citeproctext]{ref-tozdanSpezifischeSelbstwirksamkeitZur2015}
\CSLLeftMargin{66 }%
\CSLRightInline{Tozdan S, Jakob C, Schuhmann P, Budde M, Briken P.
\href{https://doi.org/10.1055/s-0035-1548842}{{Spezifische
Selbstwirksamkeit zur Beeinflussung des sexuellen Interesses an Kindern
(SSIK): Konstruktion und Validierung eines Messinstruments}}. \emph{PPmP
- Psychotherapie {\(\cdot\)} Psychosomatik {\(\cdot\)} Medizinische
Psychologie} 2015; \textbf{65}: 345--52.}

\bibitem[\citeproctext]{ref-brikenSexualOutletInventory2010}
\CSLLeftMargin{67 }%
\CSLRightInline{Briken P. Sexual {Outlet Inventory Revised}. {Institute}
for {Sex Research} and {Forensic Psychiatry}. Hamburg, Germany:
University Medical Center Hamburg-Eppendorf, 2010.}

\bibitem[\citeproctext]{ref-mackEmotionaleKongruenzMit2012}
\CSLLeftMargin{68 }%
\CSLRightInline{Mack C, Yundina E. {Emotionale Kongruenz mit der
Kinderwelt als mögliches diagnostisches Merkmal von Pädophilie?} In:
Müller JM, Rösler M, Briken P, Fromberger P, Jordan K, eds. {EFPP
Jahrbuch 2012 -- Empirische Forschung in der Forensischen Psychiatrie,
Psychologie und Psychotherapie}. 2012: 71--81.}

\bibitem[\citeproctext]{ref-banseIndirectMeasuresSexual2010}
\CSLLeftMargin{69 }%
\CSLRightInline{Banse R, Schmidt AF, Clarbour J.
\href{https://doi.org/10.1177/0093854809357598}{Indirect measures of
sexual interest in child sex offenders: A multimethod approach}.
\emph{Criminal Justice Behav} 2010; \textbf{37}: 319--35.}

\bibitem[\citeproctext]{ref-bechMeasuringDimensionPsychological2004}
\CSLLeftMargin{70 }%
\CSLRightInline{Bech P. Measuring the {Dimension} of {Psychological
General Well-Being} by the {WHO-5}. \emph{Quality of Life Newsletter}
2004; \textbf{32}: 15--6.}

\bibitem[\citeproctext]{ref-toppWHO5WellbeingIndex2015}
\CSLLeftMargin{71 }%
\CSLRightInline{Topp CW, Østergaard SD, Søndergaard S, Bech P.
\href{https://doi.org/10.1159/000376585}{The {WHO-5} well-being index: A
systematic review of the literature}. \emph{Psychother Psychosom} 2015;
\textbf{84}: 167--76.}

\bibitem[\citeproctext]{ref-wilmersDeutschsprachigeVersionWorking2008}
\CSLLeftMargin{72 }%
\CSLRightInline{Wilmers F, Munder T, Leonhart R, Herzog T, Plassmann R,
Barth J. \href{https://doi.org/10.7892/boris.27962}{{Die
deutschsprachige Version des Working Alliance Inventory-Short Revised
(WAI-SR)-Ein schulenübergreifendes, ökonomisches und empirisch
validiertes Instrument zur Erfassung der therapeutischen Allianz}}.
\emph{Klin Diagn Eval} 2008; \textbf{1}: 343--58.}

\bibitem[\citeproctext]{ref-horvathDevelopmentWorkingAlliance1986}
\CSLLeftMargin{73 }%
\CSLRightInline{Horvath A. The development of the working alliance
inventory. New York, NY: Guilford Press, 1986.}

\bibitem[\citeproctext]{ref-hopewellCONSORT2025Statement2025}
\CSLLeftMargin{74 }%
\CSLRightInline{Hopewell S, Chan A-W, Collins GS, \emph{et al.}
\href{https://doi.org/10.1016/S0140-6736(25)00672-5}{{CONSORT} 2025
statement: Updated guideline for reporting randomised trials}. \emph{The
Lancet} 2025; \textbf{405}: 1633--40.}

\end{CSLReferences}




\end{document}
